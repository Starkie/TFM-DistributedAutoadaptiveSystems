\chapter{Instrucciones de ejecución}
\label{anx:ejecucion}

En este anexo se describirán las instrucciones para desplegar y operar el prototipo. El código se encuentra en un repositorio público\footnote{Página del repositorio: \url{https://github.com/Starkie/TFM-DistributedAutoadaptiveSystems/tree/main/src/AutoAdaptativeSystem}}. Para seguir estas instrucciones, deberemos clonarlo con \texttt{git} o descargarlo como un fichero zip.

\section{Requisitos}

Para poder ejecutar el prototipo tenemos los siguientes requisitos. Todos ellos son compatibles con las principales plataformas (Windows, Linux y Mac).

\begin{itemize}
  \item SDK de .NET v6.0 o superior\footnote{.NET SDK: \url{https://dotnet.microsoft.com/en-us/download/dotnet/6.0}}.
  \item Docker Compose v2.0 o superior \footnote{Docker compose: \url{https://docs.docker.com/compose/install/}}.
  \item Powershell v5 o Powershell Core\footnote{\url{https://docs.microsoft.com/es-es/powershell/scripting/install/install-other-linux?view=powershell-7.2\#install-as-a-net-global-tool}} (para ejecutar el \foreign{english}{script} de compilación).
  \item Python 3.7 o superior\footnote{Python 3.10: \url{https://www.python.org/downloads/release/python-3106/}} (para generar los API clients).
\end{itemize}

\section{Generar API Clients}

Si se hiciera algún cambio sobre los \foreign{english}{endpoints} que exponen los servicios, tendremos que regenerar los API Clients. Para ello, contamos con un \foreign{english}{script} escrito en Python. Este compila todos los proyectos, genera las especificaciones OpenAPI y, a partir de ellas, genera los clientes.

El \foreign{english}{script} se encuentra en la ruta \texttt{src/AutoAdaptativeSystem/GenerateApiClient.py}. Para ejecutarlo, basta con invocarlo con Python: \texttt{python3 GenerateApiClient.py}. Para que funcione correctamente, todos los proyectos deben encontrarse en un estado compilable.

Si hubiera algún cambio incompatible que haga fallar la compilación, tendremos que corregirlo antes de poder continuar. Por ejemplo, acceder al servicio que falla y solucionar los errores. Hecho esto, podremos ejecutar de nuevo el \foreign{english}{script}. Tendremos que hacer esto hasta que se generen todos los clientes correctamente.

\section{Despliegue}

Para compilar y ejecutar la solución contamos con un \foreign{english}{script} de Powershell. Este se encuentra en la ruta \texttt{src/AutoAdaptativeSystem/build.ps1}. Se encarga de:
\begin{enumerate}
  \item Compilar todos los proyectos de la solución.
  \item Publicar todos los proyectos a una carpeta común.
  \item Crear los contenedores de Docker y levantar la solución con Docker Compose.
\end{enumerate}

Para ejecutarlo haremos 'pwsh build.ps1'. Si todo ha ido bien, veremos en la consola:

fragmento de docker compose.

En caso de querer parar la ejecución tendremos que acceder a la carpeta \texttt{src/AutoAdaptativeSystem/publish} y ejecutar el comando \texttt{docker-compose down -f docker-compose.yml}.


\subsection{Servicios disponibles}

Servicios disponibles: (para más info, ver el fichero docker-compose.yml)
grafana: localhost:3000

\section{Paneles de monitorización en Grafana}

En la dirección \url{htttp://localhost:3000/d/N0ZSfeUnz/adaptionloop?orgId=1&refresh=10s} tendremos disponible el panel de monitorización del bucle de adaptación.Podremos interactuar con él de diversas manears. Para más información, acceder a la documentación de grafana.

Añadir foto del panel de monitorización.

Si queremos hacer cambios y persistirlos, tendremos que exportar el panel y guardar su contenido en el fichero
