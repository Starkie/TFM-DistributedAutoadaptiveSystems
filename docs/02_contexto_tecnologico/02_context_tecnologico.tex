\chapter{Contexto Tecnológico}
\label{chap:contexto_tecnologico}

\section{Computación autónoma y bucles de control}

Según \cite{ibmcorporationArchitecturalBlueprintAutonomic2006}, la \textbf{computación autónoma} tiene como objetivo dotar a los sistemas de \textbf{autonomía} en su operación; capacidades para gestionarse a si mismos. Es decir, deberán adaptarse a los distintos escenarios que puedan darse durante su ejecución. Con esto, buscamos alcanzar una reducción en el coste de operación y hacer más gestionable la complejidad de los sistemas.

Estas adaptaciones se realizan en base a directivas de alto nivel proporcionadas por un humano: el humano fija los objetivos que el sistema debe alcanzar; y este deberá adaptarse para lograrlo, si es posible. Siguiendo con el ejemplo de la página web, el operador humano podría definir un máximo de carga por cada instancia. Entonces, cuando se supera el umbral, el sistema podría decidir que se requiere una acción correctiva que consista en desplegar nuevas instancias del servicio cuando haya muchos accesos concurrentes. Cuando la carga de los servicios baje, podemos eliminarlas.

Para implementar estas capacidades de adaptación, recurriremos a la teoría de control y el \textbf{bucle de control} (o \emph{feedback loop}). \cite{brunEngineeringSelfAdaptiveSystems2009} Se trata de un proceso iterativo compuesto por cuatro actividades (figura \ref{fig:bucle-control}):

\begin{figure}[h]
  \centering
  \includegraphics[scale=0.065]{01_introduccion/images/feedback-loop}
  \caption[Un bucle de control genérico. Consta de cuatro actividades: Recopilar información, analizarla, decidir y actuar si procede.]{Un bucle de control genérico. Consta de cuatro actividades: Recopilar información, analizarla, decidir y actuar si procede. Obtenida de \cite{dobsonSurveyAutonomicCommunications2006}.}
  \label{fig:bucle-control}
\end{figure}

\begin{itemize}
  \item \textbf{Recopilar información}: El bucle \textbf{monitoriza} el estado del sistema a través de \textbf{sondas}. Estas reportan información del sistema y del entorno de ejecución. Pueden ser métricas del rendimiento del sistema, estado de los componentes, etc.

  Estos datos en bruto deben ser limpiados, filtrados y agregados. Si se considera que son relevantes, se almacenan para informar las siguientes etapas del bucle.

  \item \textbf{Analizar}: A partir de las propiedades de adaptación, la etapa de análisis debe identificar \textbf{síntomas}: indicadores de una situación que requiera de nuestra atención. Puede ser mediante heurísticas que hayamos configurado, análisis estadístico \textcolor{red}{y cosas así}. Un ejemplo de síntoma sería ''uso de CPU elevado'', ''número elevado de mensajes encolados en un sistema de mensajería'' \textcolor{red}{entre otras}.

  \item \textbf{Decidir}: A partir de los síntomas, el bucle debe determinar si es necesario tomar alguna acción correctiva. \textbf{Planifica} las acciones que se llevarán a cabo para que el sistema se adapte y alcance una configuración deseable. Por ejemplo, si hay muchos mensajes encolados, se podría solicitar el iniciar otra instancia del servicio que los consuma y procese en paralelo.

  \item \textbf{Actuar}: Si el bucle ha planificado alguna acción, se intentará \textbf{ejecutar} en esta etapa final. Mediante \textbf{efectores} en el sistema, el bucle es capaz de cambiar la configuración actual del mismo. Dependiendo del éxito de esta etapa, la adaptación se lleva a cabo o no. Finalizada esta, se vuelve a recopilar información y el bucle continúa iterando.
\end{itemize}

Este tipo de proceso está presente en gran variedad de contextos como puede ser operación de plantas industriales, en procesos naturales, etc. \textcolor{red}{Citar TFM planta embalaje}. En la ingeniería de \emph{software}, encontramos diversas aplicaciones de los bucles de control. Pero normalmente están implícitos en la implementación. \cite{brunEngineeringSelfAdaptiveSystems2009}

los bucles de control pueden ser implícitos, dentro del código y las condiciones, o explícitos. \cite{brunEngineeringSelfAdaptiveSystems2009} Lo ideal es contar con bucles externos, esto nos permite separar la funcionalidad de las capacidades de adaptación. \ Esto facilita la implementación.

Garlan et al. also advocate to make self-adaptation external, as opposed to internal or hard-wired, to separate the concerns of system
functionality from the concerns of self-adaptation [9,16].

\textcolor{red}{Hablar de agentes autónomos como aplicación práctica. \cite{savaglioAgentbasedInternetThings2020}}

\subsection{Arquitectura para sistemas autónomos: Bucles MAPE-K}
\label{sub:bucles-mapek}

Un estilo arquitectónico muy representativo de este tipo de sistemas es el basado en bucles MAPE-K \cite{ibmcorporationArchitecturalBlueprintAutonomic2006, fonsServiciosAdaptivereadyPara2021} propuesto por IBM. Se trata de una arquitectura para sistemas distribuidos autónomos que requieran del mínimo de intervención humana para operar. Nace con el objetivo reducir en el coste de operación y hacer más gestionable la complejidad de los sistemas.

Estos sistemas son capaces de auto-gestionarse en base a \textbf{políticas}. Las políticas son un conjunto de objetivos de alto nivel que definen los usuarios encargados del sistema. El sistema debe tratar de alcanzarlos durante su funcionamiento. Además, estos motivan los cambios en el sistema, que trata de adaptarse para alcanzarlos.

Su componentes principales son los \textbf{elementos autónomos}. Cada uno de estos es capaz de autogestionarse, y colaborar en conjunto con el resto de elementos autónomos del sistema  para alcanzar los objetivos. \textcolor{red}{¿Agent based?} A su vez, estos pueden dividirse en dos partes: los recursos manejados y un manejador autónomo (el bucle de control).

Los \textbf{recursos manejados} son las unidades de funcionalidad del sistema. Puede ser cualquier tipo de recurso, \emph{hardware} o \emph{software}. Para dotarlas de capacidad de autoadaptación, las emparejamos con un \textbf{manejador autónomo}, el bucle de control. Como es un componente distinto al que implementa la funcionalidad, es entonces de tipo externo. Gestiona al recurso en base a la información que recoge del entorno de ejecución y las políticas que guían su adaptación.

Para poder ser gestionado externamente, el recurso debe implementar \textbf{\emph{touchpoints}} (\textcolor{red}{¿puntos de contacto?}): interfaces que permiten al bucle de control obtener información del estado del sistema y cambiar su configuración Existen dos tipos de \emph{touchpoints}: sondas y efectores.

En la figura \ref{fig:autonomic-element} tenemos una representación de un elemento autónomo. Distinguimos las dos partes: el manejador y el recurso. El manejador está acoplado al recurso a través de sus sensores y efectores. Podemos apreciar que \textcolor{red}{siete} componentes distintos conforman el bucle: \cite{ibmcorporationArchitecturalBlueprintAutonomic2006}

\begin{figure}[h]
  \centering
  \includegraphics[scale=2]{02_contexto_tecnologico/images/autonomic-element}
  \caption[Representación de un elemento autónomo. Distinguimos el recurso manejado y el manejador autónomo. El manejador es un bucle MAPE-K (\emph{Monitor}, \emph{Analysis}, \emph{Planification}, \emph{Execution} y \emph{Knowledge})]{Representación de un elemento autónomo. Distinguimos el recurso manejado y el manejador autónomo. El manejador es un bucle MAPE-K (\emph{Monitor}, \emph{Analysis}, \emph{Planification}, \emph{Execution} y \emph{Knowledge}). Basada en imagen de \cite{ibmcorporationArchitecturalBlueprintAutonomic2006}.}
  \label{fig:autonomic-element}
\end{figure}

\subsubsection{Sondas}
Para poder monitorizar el recurso y su entorno deberemos \textbf{instrumentarlos}. Consiste en implementar \textbf{sondas} que expongan datos relevantes a los monitores del bucle. Pueden capturar y transmitir cualquier métrica que queramos controlar: \emph{health checks}, rendimiento del servicio u otras propiedades del sistema.

\subsubsection{Monitor}
El monitor recibe las mediciones de las sondas. Se encarga de recogerlas, agregarlas y filtrarlas para determinar si ha ocurrido un evento relevante que deba ser reportado. Si se considera que lo son, estos valores se almacenan como propiedades de adaptación en la base de conocimiento. \cite{fonsEspecificacionSistemasAutoadaptativos2021} Por ejemplo, en un sistema de climatización, si la temperatura de una habitación cambia.


\subsubsection{Base de conocimiento}

La base de conocimiento (\emph{knowledge base}) está compuesta por una o más fuentes de información que el bucle tiene a su disposición. En ellas, se almacenan las \textbf{propiedades de adaptación}. En conjunto, estas propiedades conforman un modelo abstracto del sistema, que describe su estado pasado y actual: componentes, conexiones entre ellos y su configuración. \cite{garlanIncreasingSystemDependability2003}

El bucle del control trabaja con un modelo del sistema de alto nivel \cite{garlanIncreasingSystemDependability2003}. Esto le permite definir las adaptaciones desacoplándose de los elementos. Los sistemas adaptativos se basan principalmente en bucles de control. trabajan sobre system models - and in particular, architectural models - are maintained at run time
and used as a basis for system reconfiguration and repair \cite{garlanIncreasingSystemDependability2003}

\textcolor{red}{Es una arquitectura knowledge-driven \cite{taylorSoftwareArchitectureFoundations2009}.}

El conocimiento informa a todas las etapas del bucle de control. Por tanto, se trata de un componente transversal.

\subsubsection{Analizador}

En base a las propiedades de adaptación, podemos razonar sobre el estado actual del sistema y detectar situaciones que requieran de una acción correctiva. Para ello contamos con el módulo de análisis.

Una aproximación para implementarlo es mediante \textbf{reglas de adaptación}, compuestas por una condición y una acción. Las reglas se suscriben a cambios de las propiedades de adaptación. Cuando ocurra alguno, se evalúa su condición. Si esta se cumple, se ejecuta la acción asociada. En caso contrario, no hará nada.

\textcolor{red}{Siguiendo con el ejemplo del sistema de climatización, un ejemplo de regla sería...}

La acción de la regla describe una \textbf{propuesta de cambio} en la configuración del sistema. Estas se formulan en base a \textbf{operadores arquitectónicos}. \cite{garlanIncreasingSystemDependability2003} Dependiendo del estilo arquitectónico de nuestro sistema, tendremos disponibles una serie de operaciones para alterar su arquitectura.

Por ejemplo, nuestro recurso manejado podría estar implementado como microservicios. En este caso, los operadores podrían consistir en desplegar o eliminar servicios, establecer conexiones entre los servicios, eliminarlas, o cambiar las propiedades de configuración del servicio. \cite{fonsServiciosAdaptivereadyPara2021}

\subsubsection{Planificador}

Si alguna regla se dispara, el planificador recibe los cambios propuestos. Comparando sus acciones con el modelo que tenemos del estado del sistema (las propiedades de adaptación), determina si todavía es necesario ejecutar estas acciones. También comprobará si es seguro aplicarlas, ya que no deben dejar el sistema en un estado inconsistente. En caso de que las propuestas sean válidas, estas se agruparán en un \textbf{plan de adaptación}.

\subsubsection{Ejecutor}

En la etapa final del bucle tenemos al ejecutor. El ejecutor interactúa con los efectores del sistema manejado para llevar a cabo las acciones planificadas. Para ello, traduce las acciones de alto nivel (nivel de arquitectura) a acciones de más bajo nivel (en términos del propio sistema). \cite{garlanIncreasingSystemDependability2003}

\subsubsection{Efectores}

Los \textbf{efectores} sirven para modificar el estado del sistema manejado. Pueden ser ficheros de configuración, comandos, \emph{endpoints}, etc.

----------------------------------------

\textcolor{red}{Además, podemos observar que el propio manejador expone sensores y efectores, lo que permite que sean controlados por \textbf{manejadores autónomos orquestadores}. Estos gestionan a un nivel superior uno o más elementos autonómicos. Son por tanto, elementos componibles.}

\textcolor{red}{¿Hablar del nivel en el que se encuentra el bucle de control? Sistema, infraestructura, mixto, mesh \cite{mendoncaGeneralityVsReusability2018}}

\textcolor{red}{Hablar del \emph{human manager}, la capa superior al sistema. Emite las políticas y monitoriza su funcionamiento a través de las sondas del bucle orquestador.}

\textcolor{red}{Hablar de human in the loop: solicitamos la intervención del humano cuando no contamos con suficiente información para tomar una acción correctiva.}
