%%%%%%%%%%%%%%%%%%%%%%%%%%%%%%%%%%%%%%%%%%%%%%%%%%%%%%%%%%%%%%%%%%%%%%%%%%%%%%%
%                       CARREGA DE LA CLASSE DE DOCUMENT                      %
%                                                                             %
% Les opcions admissibles son:                                                %
%      12pt / 11pt            (cos dels tipus de lletra; no feu servir 10pt)  %
%                                                                             %
% catalan/spanish/english     (llengua principal del treball)                 %
%                                                                             %
% french/italian/german...    (si necessiteu fer servir alguna altra llengua) %
%                                                                             %
% listoffigures               (El document inclou un Index de figures)        %
% listoftables                (El document inclou un Index de taules)         %
% listofquadres               (El document inclou un Index de quadres)        %
% listofalgorithms            (El document inclou un Index d'algorismes)      %
%                                                                             %
%%%%%%%%%%%%%%%%%%%%%%%%%%%%%%%%%%%%%%%%%%%%%%%%%%%%%%%%%%%%%%%%%%%%%%%%%%%%%%%

\newcommand{\relativepath}{"style"}
\documentclass[11pt,spanish,listoffigures,listoftables,listlistings]{\relativepath/tfgetsinf}

%%%%%%%%%%%%%%%%%%%%%%%%%%%%%%%%%%%%%%%%%%%%%%%%%%%%%%%%%%%%%%%%%%%%%%%%%%%%%%%
%                     CODIFICACIO DEL FITXER FONT                             %
%                                                                             %
%    windows fa servir normalment 'ansinew'                                   %
%    amb linux es possible que siga 'latin1' o 'latin9'                       %
%    Pero el mes recomanable es fer servir utf8 (unicode 8)                   %
%                                          (si el vostre editor ho permet)    %
%%%%%%%%%%%%%%%%%%%%%%%%%%%%%%%%%%%%%%%%%%%%%%%%%%%%%%%%%%%%%%%%%%%%%%%%%%%%%%%

\usepackage[utf8]{inputenc}

%%%%%%%%%%%%%%%%%%%%%%%%%%%%%%%%%%%%%%%%%%%%%%%%%%%%%%%%%%%%%%%%%%%%%%%%%%%%%%%
%                        ALTRES PAQUETS I DEFINICIONS                         %
%                                                                             %
% Carregueu aci els paquets que necessiteu i declareu les comandes i entorns  %
%                                          (aquesta seccio pot ser buida)     %
%%%%%%%%%%%%%%%%%%%%%%%%%%%%%%%%%%%%%%%%%%%%%%%%%%%%%%%%%%%%%%%%%%%%%%%%%%%%%%%

\usepackage{graphicx}
\usepackage{wrapfig}

%%%%%%%%%%%%%%%%%%%%%%%%%%%%%%%%%%%%%%%%%%%%%%%%%%%%%%%%%%%%%%%%%%%%%%%%%%%%%%%
%                        DADES DEL TREBALL                                    %
%                                                                             %
% titol, alumne, tutor i curs academic                                        %
%%%%%%%%%%%%%%%%%%%%%%%%%%%%%%%%%%%%%%%%%%%%%%%%%%%%%%%%%%%%%%%%%%%%%%%%%%%%%%%

\title{Refactorización de una infraestructura de bucles MAPE-K como microservicios}
\author{Adriano Vega Llobell}
\tutor{Joan Fons i Cors}
\curs{2021-2022}

%%%%%%%%%%%%%%%%%%%%%%%%%%%%%%%%%%%%%%%%%%%%%%%%%%%%%%%%%%%%%%%%%%%%%%%%%%%%%%%
%                     PARAULES CLAU/PALABRAS CLAVE/KEY WORDS                  %
%                                                                             %
% Independentment de la llengua del treball, s'hi han d'incloure              %
% les paraules clau i el resum en els tres idiomes                            %
%%%%%%%%%%%%%%%%%%%%%%%%%%%%%%%%%%%%%%%%%%%%%%%%%%%%%%%%%%%%%%%%%%%%%%%%%%%%%%%

\keywords{????, ?????????, ????, ?????????????????} % Paraules clau
         {computación autónoma, arquitecturas de microservicios, bucles de control, MAPE-K, computación en la nube}              % Palabras clave
         {?????, ????? ?????, ?????????????}        % Key words

%%%%%%%%%%%%%%%%%%%%%%%%%%%%%%%%%%%%%%%%%%%%%%%%%%%%%%%%%%%%%%%%%%%%%%%%%%%%%%%
%                              INICI DEL DOCUMENT                             %
%%%%%%%%%%%%%%%%%%%%%%%%%%%%%%%%%%%%%%%%%%%%%%%%%%%%%%%%%%%%%%%%%%%%%%%%%%%%%%%
\begin{document}

\bibliographystyle{ieeetr}

%%%%%%%%%%%%%%%%%%%%%%%%%%%%%%%%%%%%%%%%%%%%%%%%%%%%%%%%%%%%%%%%%%%%%%%%%%%%%%%
%              RESUMS DEL TFG EN VALENCIA, CASTELLA I ANGLES                  %
%%%%%%%%%%%%%%%%%%%%%%%%%%%%%%%%%%%%%%%%%%%%%%%%%%%%%%%%%%%%%%%%%%%%%%%%%%%%%%%

\begin{abstract}
????
\end{abstract}
\begin{abstract}[spanish]
  La Computación Autónoma (\foreign{english}{Autonomic Computing}) promueve la ingeniería, diseño y desarrollo de sistemas con capacidades de autoadaptación, a través del uso de bucles de control. Estas capacidades le confieren a estos sistemas la posibilidad de adaptarse a entornos cambiantes, a conflictos operacionales e incluso a la optimización dinámica en su ejecución. Por otra parte, en la última década, la computación en el cloud y las arquitecturas basadas en microservicios se han postulado como una infraestructura muy flexible y dinámica para desplegar soluciones altamente disponibles y eficientes. Hay una tendencia clara a aplicar este tipo de infraestructuras, gracias a los múltiples beneficios que aporta.

  En este trabajo se explorará cómo diseñar soluciones que, aplicando los conceptos de los bucles de control (AC), estén preparadas para desplegarse en la nube. Para ello se tomará como punto de partida la infraestructura FaDA (desarrollada por el grupo PROS/Tatami del instituto VRAIN/UPV) que propone una estrategia para realizar la ingeniería de sistemas autoadaptativos usando bucles de control MAPE-K.

  Como resultado de este TFM se espera obtener la definición arquitectónica de soluciones autoadaptativas (incluyendo tanto al bucle de control MAPE-K como directrices para la implementación de los diferentes componentes adaptativos de la solución) diseñadas para desplegarse nativamente como microservicios en la nube. Por último, se aplicará la propuesta realizada al desarrollo de un caso práctico para demostrar su viabilidad y aplicabilidad.
\end{abstract}
\begin{abstract}[english]
????
\end{abstract}

%%%%%%%%%%%%%%%%%%%%%%%%%%%%%%%%%%%%%%%%%%%%%%%%%%%%%%%%%%%%%%%%%%%%%%%%%%%%%%%
%                              CONTINGUT DEL TREBALL                          %
%%%%%%%%%%%%%%%%%%%%%%%%%%%%%%%%%%%%%%%%%%%%%%%%%%%%%%%%%%%%%%%%%%%%%%%%%%%%%%%

\mainmatter

%%%%%%%%%%%%%%%%%%%%%%%%%%%%%%%%%%%%%%%%%%%%%%%%%%%%%%%%%%%%%%%%%%%%%%%%%%%%%%%
%                         CAPITOLS (tants com calga)                          %
%%%%%%%%%%%%%%%%%%%%%%%%%%%%%%%%%%%%%%%%%%%%%%%%%%%%%%%%%%%%%%%%%%%%%%%%%%%%%%%

%%%%%%%%%%%%%%%%%%%%%%%%%%%%%%%%%%%%%%%%%%%%%%%%%%%%%%%%%%%%%%%%%%%%%%%%%%%%%%%
%                                  INTRODUCCIO                                %
%%%%%%%%%%%%%%%%%%%%%%%%%%%%%%%%%%%%%%%%%%%%%%%%%%%%%%%%%%%%%%%%%%%%%%%%%%%%%%%

\chapter{Introducción}
\label{chap:introduccion}

La Computación Autónoma (\foreign{english}{Autonomic Computing}) promueve la ingeniería, diseño y desarrollo de sistemas con capacidades de auto-adaptación, a través del uso de bucles de control. Estas capacidades le confieren a estos sistemas la posibilidad de adaptarse a entornos cambiantes, a conflictos operacionales e incluso a la optimización dinámica en su ejecución. Por otra parte, en la última década, la computación en el cloud y las arquitecturas basadas en microservicios se han postulado como una infraestructura muy flexible y dinámica para desplegar soluciones altamente disponibles y eficientes. Hay una tendencia clara a aplicar este tipo de infraestructuras, gracias a los múltiples beneficios que aporta.

\textcolor{red}{La revolución digital \emph{citation needed} ha permeado en todos los aspectos de nuestras vidas. En nuestro día a día usamos una gran variedad de aplicaciones: redes sociales, aplicaciones de ofimática, comercios en línea\dots. A su vez, gracias a las posibilidades que ofrecen los avances en las tecnologías, cada vez ofrecen distintas funcionalidades y están en constante evolución. \emph{citation needed} Esto deriva en mayor complejidad de estos programas. \emph{citation needed}}

\textcolor{red}{Las aplicaciones están en constante evolución. Hay muchas partes móviles}

\textcolor{red}{Tambien el internet de las cosas \emph{citation needed}}

\textcolor{red}{Uno de los requisitos claves es la disponibilidad: nuestros servicios deben estar en funcionamiento en todo momento para atender a nuestros usuarios. Tomemos por ejemplo el caso de una tienda \foreign{english}{on-line}. Necesitamos asegurar que esté disponible el mayor tiempo posible. Si surgiera una incidencia y se degrada la capacidad de atender a clientes, o directamente no podemos atender a ninguno, perderemos ingresos.}

\textcolor{red}{Debido a esto querremos que nuestro sistema sea capaz de adaptarse a picos de demanda, aumentando su capacidad de cómputo cuando tengamos mayor afluencia de clientes. Por ejemplo, en temporadas de rebajas como \emph{black friday}. Operar sistemas capaces de escalar, deriva en sistemas complejos. Como no es viable tener a operarios pendientes del estado del sistema para llevar a cabo estas adaptaciones. Deben hacerse automáticamente.}

\textcolor{red}{En el ámbito de la computación autónoma encontramos el concepto de sistemas \textbf{autoadaptativos}: aquellos capaces de ajustar su propio comportamiento en base a cambios en su entorno de operación. Se caracteriza por dotarlos con capacidades para razonar sobre su estado de operación y su entorno. En base a estos parámetros, el sistema puede intuir que debe reconfigurarse para cumplir con los objetivos que tiene marcados. Para ello, en base a una serie de estrategias predefinidas, es capaz de elegir su siguiente configuración. \cite{garlanIncreasingSystemDependability2003}. Esto conlleva mover a tiempo de ejecución las decisiones de arquitectura y funcionalidad. Con ello, buscamos permitir un comportamiento dinámico del sistema. \cite{brunEngineeringSelfAdaptiveSystems2009}.}

\section{Motivación}

En este trabajo se exploró cómo diseñar soluciones que, aplicando los conceptos de los bucles de control (AC), estén preparadas para desplegarse en la nube. Para ello se tomó como punto de partida la infraestructura FaDA\footnote{Página oficial: \url{http://fada.tatami.webs.upv.es/}} (desarrollada por el grupo PROS/Tatami\footnote{Página oficial: \url{http://www.pros.webs.upv.es/}} del instituto VRAIN/UPV\footnote{Página oficial: \url{https://vrain.upv.es/}}). Esta propone una estrategia para realizar la ingeniería de sistemas auto-adaptativos usando bucles de control MAPE-K\cite{ibmcorporationArchitecturalBlueprintAutonomic2006, fonsServiciosAdaptivereadyPara2021}.

Actualmente, el bucle de control está implementado como un servicio monolítico. Todos sus componentes operan dentro del mismo proceso, incluidos los específicos a sistemas manejados (sondas, monitores\dots). Se trata por tanto de una implementación muy rígida. En caso de querer modificar algún componente, hay que redesplegarlo entero.

Por ello, se buscó dividir su funcionalidad en microservicios. Es decir, \textbf{cambiar la topología} de la solución. Con ello, podríamos independizar los compomentes y mejoraríamos su despliegue y su escalabilidad\textcolor{red}{cita sam newman}.

\section{Objetivos}

Como resultado de este TFM se espera obtener la definición arquitectónica de soluciones auto-adaptativas (incluyendo tanto al bucle de control MAPE-K como directrices para la implementación de los diferentes componentes adaptativos de la solución) diseñadas para desplegarse nativamente como microservicios en la nube. Por último, se aplicará la propuesta realizada al desarrollo de un caso práctico para demostrar su viabilidad y aplicabilidad.

Para el desarrollo del trabajo nos planteamos los siguientes objetivos objetivos:

\begin{enumerate}
  \item  Definición de las piezas necesarias para componer una solución completa, y que permita ir extendiendo el bucle (en nuestro caso, las piezas eran M, A, P, E i K, pero también los Monitores, Sondas, Reglas, etc. que se convertirían en estos microservicios);

  \item La definición de las APIs (REST) para permitir la comunicación entre las piezas
\end{enumerate}

\textcolor{red}{La idea es separar cada una de sus etapas en microservicios individuales. De esta forma, podemos desarrollarlas de forma independiente entre ellas, replicarlas para mejorar su escalabilidad, o sustituirlas por }implementaciones distintas, etc.

Para desarrollar el trabajo, propusimos el siguiente plan:
\begin{itemize}
  \item Cada etapa del bucle será un microservicio distinto. Extraeremos cuatro microservicios distintos: Planificador, Analizador,
\end{itemize}

\textcolor{red}{Por tanto, los conectores elegidos para comunicar los microservicios han sido más centrados en comunicar con las APIs públicas que expone cada uno.}



\section{Estructura de la memoria}

????? ????????????? ????????????? ????????????? ????????????? ?????????????

%\section{Notes bibliografiques} %%%%% Opcional

%????? ????????????? ????????????? ????????????? ????????????? ?????????????

\textcolor{red}{Añadir diagrama de gant con los 7 hitos}


\chapter{Contexto Tecnológico}
\label{chap:contexto_tecnologico}

En este capítulo presentamos algunos de los conceptos más relevantes para el trabajo. Entre ellos se incluyen las arquitecturas de \foreign{english}{software}, la computación autónoma y los bucles de control. Estos conceptos nos acompañarán a lo largo de la memoria.

\section{Arquitecturas de \foreign{english}{Software}}

En esta sección haremos una breve introducción a las arquitecturas de \foreign{english}{software}. Describiremos su motivación y los elementos que las componen. Esta sección es interesante por dos motivos:

\begin{itemize}
  \item En el trabajo tratamos la migración de un sistema con arquitectura monolítica a una distribuida basada en microservicios. Trabajamos con componentes, conectores y otros elementos arquitectónicos.

  \item Por otro lado, el bucle MAPE-K es capaz de cambiar la arquitectura del recurso manejado en tiempo de ejecución. Sus adaptaciones se describen en base a operadores arquitectónicos: añadir o eliminar componentes, conectar o desconectarlos\dots
\end{itemize}

\subsection{Definición}

Según \cite{taylorSoftwareArchitectureFoundations2009}, la \textbf{arquitectura de un sistema \emph{software}} es el conjunto de todas las \textbf{decisiones principales de diseño} que se toman durante su ciclo de vida; aquellas que sientan las bases del sistema. Estas afectan a todos sus apartados: la funcionalidad que debe ofrecer, la tecnología para su implementación, cómo se desplegará, etc. En conjunto, definen una pauta que guía (y a la vez refleja) el diseño, la implementación, la operación y la evolución del sistema.

Todos los sistemas \emph{software} cuentan con una. La diferencia radica en si esta ha sido diseñada y descrita explícitamente o ha quedado implícita en su implementación. \cite{taylorSoftwareArchitectureFoundations2009} En el segundo caso es probable que, con el paso del tiempo, se ``erosione`` su arquitectura: se implementan funcionalidades sin respetar la estructura. También se olvida el por qué de ciertas decisiones. En general, se vuelve más difícil de mantener y desarrollar nuevas funcionalidades. Se convierte en una ''gran bola de barro''. \cite{footeBigBallMud1997}

Para evitarlo, es vital dedicar tiempo para plantear y definir una buena arquitectura. Según \cite{martinChapter15What2018}, una buena arquitectura es aquella que es <<\emph{fácil de entender, fácil de desarrollar, fácil de mantener y fácil de desplegar}>>. Esto se traducirá en una reducción de costes de mantenimiento y operación.

\subsection{Componentes de una arquitectura}

Otra posible definición de arquitectura la encontramos en el estándar IEEE 42010-2011 \cite{ieeeStandard420102011Systems2011}: es <<\emph{un conjunto de conceptos o propiedades fundamentales, personificados por sus elementos, sus relaciones, y los principios que guían su diseño y evolución}>>. Podemos describirlas entonces usando estos tres conceptos: \cite{perryFoundationsStudySoftware1992}

    \begin{itemize}
        \item \textbf{Elementos}: Son las piezas fundamentales que conforman el sistema. Representan las unidades de funcionalidad de la aplicación. Se utilizan para describir \textbf{\emph{qué}} partes componen el sistema. Por ejemplo: un módulo, un servicio web, un conector...

        \item \textbf{Forma}: El conjunto de propiedades y relaciones de un elemenento con otros o con el entorno de operación. Describe \textbf{\emph{cómo}} está organizado el sistema. Por ejemplo: un servicio A contacta con otro, B, usando una llamada HTTP.

        \item \textbf{Justificación}: Razonamiento o motivación de las decisiones que se han tomado. Responden al \textbf{\emph{por qué}} algo se hace de una manera determinada. Nos aporta detalles más precisos sobre el sistema que no se pueden representar mediante los elementos o la forma. Un ejemplo podría ser qué alternativas se consideraron para tomar una decisión; y por qué se descartaron en favor de la elegida.

    \end{itemize}

Para este trabajo, nos interesan especialmente los elementos. Concretamente los componentes y los conectores.

\subsubsection{Componentes}

El primer tipo de elemento que debemos tratar son los componentes. Según \cite{taylorSoftwareArchitectureFoundations2009}, los \textbf{componentes} son <<\emph{elementos arquitectónicos que encapsulan un subconjunto de la funcionalidad y/o de los datos del sistema}>>. Dependiendo de las características de nuestro sistema (y del nivel de abstracción que usemos) pueden tomar distintas formas: objetos, módulos dentro un mismo proceso, servicios distribuidos, etc.

\begin{wrapfigure}{r}{0.40\linewidth}
  \centering
  \includegraphics[scale=0.8]{cap_contexto_tecnologico/images/componente-ejemplo}
  \caption{El servicio de monitorización representado como un componente. Ofrece una interfaz (\emph{IMonitoringService}) y depende de otra para funcionar (\emph{IKnowledgeService}).}
  \label{fig:componenteEjemplo}
\end{wrapfigure}

Los componentes exponen una \textbf{interfaz} que permite acceder a la funcionalidad o datos que encapsulan. A su vez, también declaran una serie de \textbf{dependencias} con interfaces de otros. Allí se incluyen todos los elementos que requieren para poder funcionar. En la figura \ref{fig:componenteEjemplo} tenemos un ejemplo. \emph{Monitoring Service} expone la interfaz \emph{IMonitoringService}. Para poder funcionar, depende de un componente que ofrezca \emph{IKnowledgeService}.

Por si solos, estos componentes independientes no aportan mucho valor. Más bien son la unidad básica de composición: podemos combinar varios de ellos para que trabajen conjuntamente y realicen tareas más complejas. Así, podemos \textbf{componer sistemas}. \cite{mehtaTaxonomySoftwareConnectors2000} La integración y la interacción entre ellos son aspectos clave que debemos abordar.

\subsubsection{Conectores}

Para que los componentes puedan interactuar, necesitamos definir uno o más mecanismos de comunicación. Recurriremos entonces a los \textbf{conectores}. Se trata de elementos arquitectónicos que nos ayudan a investigar y especificar la comunicación entre componentes. \cite{perryFoundationsStudySoftware1992} Son elementos independientes a la aplicación. No están acoplados a componentes específicos. Son por tanto \textbf{reutilizables}. \cite{taylorComponentMessagebasedArchitectural1996a}

Internamente, están compuestos por uno o más \textbf{conductos} o canales. A través de estos se realiza la transmisión de información. Según su \textbf{cardinalidad}, estos podrán conectar más o menos componentes. Hay una gran variedad de conductos disponibles: comunicación interproceso, en red, etc. Clasificamos los conectores según la complejidad de los conductos que utilizan \cite{mehtaTaxonomySoftwareConnectors2000}:

\begin{itemize}
    \item \textbf{Conectores simples}: solo cuentan con un conducto, sin lógica asociada. Son conectores sencillos. Suelen estar ya implementados en los lenguajes de programación. Por ejemplo: una llamada a función en un programa o el sistema de entrada / salida de ficheros.

    \item \textbf{Conectores complejos}: cuentan con uno o más conductos. Se definen por composición a partir de múltiples conectores simples. Además, pueden contar con funcionalidad para manejar el flujo de datos y/o control. Suelen encontrarse en librerias o \foreign{english}{middlewares}. Por ejemplo: un balanceador de carga que redirige peticiones a los nodos.
\end{itemize}

Una vez hayamos decidido que dos componentes necesitan comunicarse, es momento de evaluar qué mecanismo de comunicación es más adecuado. Basándonos en nuestros requisitos, la arquitectura ya definida, y los mecanismos de despliegue que queremos usar, elegimos el conector apropiado. Podemos orientarnos con taxonomías como la de \cite{mehtaTaxonomySoftwareConnectors2000}.

Fijémonos por ejemplo en la figura \ref{fig:componentesYConectorEjemplo}. En ella mostramos dos elementos que queremos comunicar. Vemos que no se ha especificado todavía ningún detalle sobre cómo se implementará. Esto nos permitirá estudiar sus necesidades y elegir el mecanismo óptimo para la interacción. \cite{taylorSoftwareArchitectureFoundations2009}.

\begin{figure}[h!]
  \centering
  \includegraphics[scale=0.78]{cap_contexto_tecnologico/images/conector}
  \caption{Ejemplo de comunicación de dos componentes a través de un conector.}
  \label{fig:componentesYConectorEjemplo}
\end{figure}

\section{Computación autónoma y bucles de control}

Según \cite{ibmcorporationArchitecturalBlueprintAutonomic2006}, la \textbf{computación autónoma} tiene como objetivo dotar a los sistemas de \textbf{autonomía} en su operación. Es decir, capacidades para gestionarse a si mismos. Estas capacidades les permitirá adaptarse a los cambios en su entorno de ejecución. Mediante la autonomía, buscamos una reducción en el coste de operación y hacer más manejable la complejidad de los sistemas.

El sistema decide si es necesario ejecutar adaptaciones en base a directivas de alto nivel, los \textbf{objetivos}. Un operario humano define estas metas que el sistema debe alcanzar o mantener durante su ejecución. A partir de las políticas y la información del entorno, puede intuir que es necesario reconfigurarse para cumplirlas.

Para ello, cuenta con una serie de estrategias predefinidas que le permiten elegir su siguiente configuración. \cite{garlanIncreasingSystemDependability2003} Las adaptaciones pueden aplicarse de distintas formas: cambios en los parámetros de configuración, habilitar o deshabilitar funcionalidades, etc. Esto conlleva mover a tiempo de ejecución las decisiones de arquitectura y funcionalidad. Con ello, buscamos permitir un comportamiento dinámico del sistema. \cite{brunEngineeringSelfAdaptiveSystems2009}

Siguiendo con el ejemplo de la tienda \foreign{english}{on-line}, el operario podría definir un umbral máximo de carga por cada instancia. Cuando se supere, el sistema podría decidir que se requiere una acción correctiva. Por ejemplo, esta acción podría consistir en desplegar nuevas instancias del servicio. Cuando la carga de los servicios baje, podrá optar por eliminarlas.

\subsubsection{Bucles de control}

Para implementar estas capacidades de adaptación se recurre a la teoría de control y al \textbf{bucle de control} (o \emph{feedback loop}). \cite{brunEngineeringSelfAdaptiveSystems2009} Se trata de un proceso iterativo para la gestión de sistemas. A partir de información sobre el estado del sistema y su entorno, pauta acciones correctivas. Estas se basan en heurísticas definidas por los administradores del sistema. Puede dividirse en cuatro etapas (figura \ref{fig:bucle-control}):

\begin{figure}[h]
  \centering
  \includegraphics[scale=0.065]{cap_introduccion/images/feedback-loop}
  \caption[Un bucle de control genérico. Consta de cuatro actividades: Recopilar información, analizarla, decidir y actuar si procede.]{Un bucle de control genérico. Consta de cuatro actividades: Recopilar información, analizarla, decidir y actuar si procede. Obtenida de \cite{dobsonSurveyAutonomicCommunications2006}.}
  \label{fig:bucle-control}
\end{figure}

\begin{itemize}
  \item \textbf{Recopilar información}: El bucle \textbf{monitoriza} el estado del sistema a través de \textbf{sondas}. Estas reportan información del sistema y del entorno de ejecución. Pueden ser métricas de rendimiento, estado de los componentes, cambios en en el entorno, etc.

  Estos datos en bruto deben ser limpiados, filtrados y agregados para sintetizarlos en propiedades de nuestro interés. Si se considera que son relevantes, se almacenan para informar las siguientes etapas del bucle.

  \item \textbf{Analizar}: Basándose en la información considerada de interés, la etapa de análisis debe identificar \textbf{síntomas}: indicadores de una situación que requiera de nuestra atención. Puede ser mediante heurísticas predefinidas, análisis estadístico u otros métodos. Un ejemplo de síntoma sería ''uso de CPU elevado'', ''número elevado de mensajes encolados en un sistema de mensajería'', etc.

  \item \textbf{Decidir}: A partir de los síntomas, el bucle debe determinar si es necesario tomar alguna acción correctiva. Podría detectarse que no estamos cumpliendo los objetivos, o que puede optimizarse la configuración actual. Para ello, se \textbf{planifica} qué acciones deben llevarse a cabo para que el sistema se adapte y alcance una configuración deseable. Por ejemplo, si hay muchos mensajes encolados, se solicitaría iniciar otra instancia del servicio que los consuma y procese en paralelo.

  \item \textbf{Actuar}: Si se ha planificado alguna acción se intentará \textbf{ejecutar} en esta etapa final. Mediante \textbf{efectores} en el sistema, el bucle es capaz de modificar su configuración. Dependiendo del éxito de ejecución, la adaptación se lleva a cabo o no. Finalizada esta etapa, se vuelve a recopilar información e inicia de nuevo el proceso.
\end{itemize}

\subsubsection{En la ingenieria de \foreign{english}{software}}

En la ingeniería de \emph{software}, los bucles de control suelen suelen implementarse de dos formas distintas: \textbf{implícitos} o \textbf{explícitos}. La más habitual es la primera: se encuentran implícitos en la implementación de los procesos del sistema. \cite{brunEngineeringSelfAdaptiveSystems2009} No son componentes externos dedicados. Esto dificulta su implementación y mantenimiento ya que están entrelazados con la funcionalidad.

Por otro lado, aproximaciones como las de \cite{ibmcorporationArchitecturalBlueprintAutonomic2006} o \cite{garlanIncreasingSystemDependability2003} optan por la segunda: bucles como componentes externos. Esto permite separar la funcionalidad de las capacidades de adaptación.  Al dividirse estas responsabilidades, se puede reducir la complejidad de la implementación. En este trabajo nos centraremos en esta segunda variante.

En el caso de los bucles externos, pueden categorizarse además en base al \textbf{nivel} en el que operan. \cite{mendoncaGeneralityVsReusability2018} Esto determinará el nivel de abstracción que tienen sobre el sistema que controlan, afectando a su reusabilidad en otras arquitecturas. De menor a mayor nivel de abstracción (y de reusabilidad) tenemos: nivel del sistema, mixto e infraestructura.

En el \textbf{nivel de sistema}, el bucle de control es un componente que se despliega al mismo nivel que el sistema manejado. Así, tendrá mucho más conocimiento de la solución y podrá ofrecer adaptaciones específicas para ella. Esto implica que acaba acoplado a ella y es menos reusable.

Por otro lado, en el \textbf{nivel de infraestructura}, el bucle se encuentra en un nivel de abstracción superior al sistema manejado. No tiene conocimiento sobre su implementación específica. Solo expone una serie de adaptaciones genéricas aplicables según la infraestructura en la que corre. Finalmente, el \textbf{nivel mixto} es una mezcla de ambas aproximaciones. El bucle tendrá componentes en ambas capas, capaces de comunicarse entre ellas para ofrecer una mejor capacidad de adaptación.

En cuanto a aplicaciones prácticas, podemos encontrarlos en gran variedad de contextos: balanceadores de carga \cite{mishraLoadBalancingCloud2020}, operación de plantas industriales \cite{climentpenadesDissenyPrototipatSolucions2020a}, etc. Uno de los campos en lo que está teniendo más impacto es en el Internet de las Cosas (IoT). \cite{savaglioAgentbasedInternetThings2020} En él, cada uno de los elementos debe operar de forma autónoma y ser capaz de colaborar con el resto de elementos de la red para cumplir con un objetivo común.

\section{Arquitecturas para sistemas autónomos: Bucles MAPE-K}
\label{sec:bucles-mapek}

Un estilo arquitectónico muy representativo es el basado en bucles MAPE-K \cite{ibmcorporationArchitecturalBlueprintAutonomic2006, fonsServiciosAdaptivereadyPara2021} propuesto por IBM. Se trata de una referencia arquitectónica para desarrollar sistemas distribuidos autónomos. Nace con el objetivo hacer más manejable la complejidad de estos sistemas; y reducir sus costes de operación, requiriendo de la minima intervención humana.

Sus componentes principales son los \textbf{elementos autónomos}. Cada uno de ellos es capaz de autogestionarse y colaborar con el resto de elementos del sistema para alcanzar los objetivos. Podría considerarse como una arquitectura basada en agentes. \cite{savaglioAgentbasedInternetThings2020} A su vez, los elementos autónomos pueden dividirse en dos partes: un recurso manejado y un manejador autónomo (el bucle de control).

Los \textbf{recursos manejados} son las unidades de funcionalidad. Puede ser cualquier tipo de recurso, \emph{hardware} o \emph{software}. Para dotarlos de capacidad de autoadaptación, los emparejamos con un \textbf{manejador autónomo}: el bucle de control. Gestiona al recurso en base a la información que recoge del entorno de ejecución y las políticas que guían su adaptación.

El bucle es de tipo externo, ya que es un componente distinto al que implementa la funcionalidad. Por tanto, el recurso debe implementar puntos de contacto (\textbf{\emph{touchpoints}}): interfaces que permitan obtener información de su estado (sondas) y cambiar su configuración (efectores).

Estos elementos autónomos se auto-gestionan en base a \textbf{políticas}: un conjunto de objetivos de alto nivel definidos por sus administradores. El sistema tratará de mantener su cumplimiento durante su ejecución. Para alcanzarlos, el manejador autónomo planifica cambios en la configuración del recurso manejado.

\subsection{Estructura del bucle MAPE-K}

En la figura \ref{fig:autonomic-element} mostramos una representación de un elemento autónomo. Distinguimos las dos partes principales: el manejador y el recurso. El manejador contacta con el recurso a través de sus sensores y efectores. Podemos apreciar los componentes que conforman el bucle, y que describimos a continuación: \cite{ibmcorporationArchitecturalBlueprintAutonomic2006}

\begin{figure}[h]
  \centering
  \includegraphics[scale=2]{cap_contexto_tecnologico/images/autonomic-element}
  \caption[Representación de un elemento autónomo. Distinguimos el recurso manejado y el manejador autónomo. El manejador es un bucle MAPE-K (\emph{Monitor}, \emph{Analysis}, \emph{Planification}, \emph{Execution} y \emph{Knowledge})]{Representación de un elemento autónomo. Distinguimos el recurso manejado y el manejador autónomo. El manejador es un bucle MAPE-K. Basada en imagen de \cite{ibmcorporationArchitecturalBlueprintAutonomic2006}.}
  \label{fig:autonomic-element}
\end{figure}

Para presentar estos componentes, describiremos un ejemplo de cómo se manejaría un servicio web. Nos centraremos en escalar este servicio en base a la carga del sistema. Deseamos que, en caso de carga elevada, se desplieguen nuevas instancias. Si la carga bajara, el sistema debería eliminar las instancias redundantes.

\subsubsection{Sondas}

Para monitorizar el recurso y su entorno debemos \textbf{instrumentarlos}. Consiste en implementar \textbf{sondas} que expongan datos relevantes a los monitores del bucle. Pueden capturar y transmitir cualquier aspecto que queramos controlar: \emph{health checks}, rendimiento del servicio u otras propiedades del sistema.

Para nuestro servicio web, una métrica relevante sería el número de peticiones por segundo que está atendiendo. La sonda reportaría el número de peticiones que se han atendido hasta un determinado momento.

\subsubsection{Monitor}

El monitor recibe las mediciones de las sondas. Se encarga de recogerlas, agregarlas y filtrarlas para extraer información relevante. La información se almacenará como propiedades de adaptación en la base de conocimiento.\cite{fonsEspecificacionSistemasAutoadaptativos2021} El monitor y las sondas componen la etapa de recopilar información de los bucles de control.

Siguiendo con nuestro ejemplo, el monitor recibiría el número de peticiones atendidas, y las agregaría en una métrica de serie temporal de peticiones por segundo. Esta sería una de nuestras propiedades de adaptación. En base a ella, las siguientes etapas tomarán las decisiones convenientes para escalar nuestro servicio.

\subsubsection{Base de conocimiento}

La base de conocimiento (\emph{knowledge base}) es el componente base de toda la arquitectura. Informa a todas las etapas del bucle de control. Por lo que se trata de un componente transversal.

Está compuesta por una o más fuentes de información que el bucle tiene a su disposición. A partir de ellas, se almacenan las \textbf{propiedades de adaptación}. Estas describen el estado pasado y presente del sistema y su entorno: métricas, componentes, conexiones entre ellos, parámetros de configuración\dots

En conjunto, estas propiedades conforman un modelo abstracto del estado del recurso manejado que se mantiene en tiempo de ejecución. \cite{garlanIncreasingSystemDependability2003}. Las demás etapas del bucle operan en base a él. Como veremos más adelante, los efectores se encargan de traducir las acciones correctivas del modelo de alto nivel a términos del recurso manejado.

\subsubsection{Analizador}

En base al modelo abstracto del sistema, podemos razonar sobre el estado actual sin acoplarnos al recurso manejado. Podemos definir heurísticas que nos permitan detectar situaciones que requieran de una acción correctiva. Esta es la función del analizador.

Para implementarlo, una posible aproximación es mediante \textbf{reglas de adaptación}. Estas pueden dividirse en dos partes: la condición y la acción. La condición se define a partir de las propiedades de adaptación y evalúa si es necesario ejecutar la acción correctiva.

La acción de la regla describe una \textbf{propuesta de cambio} en la configuración del sistema. Estas se formulan en base a \textbf{operadores arquitectónicos}. \cite{garlanIncreasingSystemDependability2003} Dependiendo del estilo arquitectónico de nuestro sistema, tendremos disponibles una serie de operaciones para alterar su arquitectura.

Por ejemplo, nuestro recurso manejado podría estar implementado como microservicios. En este caso, los operadores podrían consistir en desplegar o eliminar servicios, establecer conexiones entre los servicios, eliminarlas, o cambiar las propiedades de configuración del servicio. \cite{fonsServiciosAdaptivereadyPara2021}

Las reglas se suscriben a cambios de las propiedades de las que dependen. Cuando ocurra alguno, se evalúa su condición. Si esta se cumple, se ejecuta la acción asociada. En caso contrario, no hará nada.

Respecto al servicio web, definiremos reglas tomando el valor del número de peticiones por segundo. Podemos definirlas con umbrales para este valor: si es muy alto, la regla solicita el despliegue de una nueva instancia. Cuando la carga baje, y si el servicio está replicado, podremos eliminarlas.

\subsubsection{Planificador}

Si alguna regla se dispara, el planificador recibe su propuesta de cambio. Este módulo se encarga de validar las acciones propuestas y agruparlas en un \textbf{plan de adaptación}. Para ello, recurre al conocimiento y compara el estado actual del sistema con las acciones solicitadas.

Deberá verificar si estas acciones siguen siendo necesarias. Podría ocurrir que desde que se solicitaron hasta que se genera el plan de adaptación, haya cambiado el estado del sistema. También comprobará si es seguro aplicarlas, ya que no deben dejar el sistema en un estado inconsistente.

\subsubsection{Ejecutor}

En la etapa final del bucle tenemos al ejecutor. Recibe el plan de adaptación del planificador y, como su nombre indica, es el encargado de ejecutarlo. Para ello, manipula los efectores del recurso manejado. Deberá identificar a cuáles debe transmitir el comando para realizar la adaptación.

Si una adaptación se lleva a cabo correctamente, deberá reflejarse en el conocimiento el nuevo estado, una vez se confirme. En caso de error, deberemos tener mecanismos de compensación que reviertan las acciones ejecutadas. Así, evitamos que el sistema quede en un estado inconsistente.

\subsubsection{Efectores}

Los \textbf{efectores} son el segundo tipo de \foreign{english}{touchpoint} que debe ofrecer el recurso manejado. Ofrecen una interfaz común que permite al bucle modificar la configuración o estado del sistema. Deberán interpretar estas acciones, descritas en conceptos de alto nivel (nivel de arquitectura) y traducirlas a acciones de más bajo nivel (en términos del propio sistema). \cite{garlanIncreasingSystemDependability2003} Es decir, deberán determinar cómo ejecutarlas en el recurso manejado.

La comunicación entre este servicio y el sistema es un tanto especial: dependerá del sistema manejado; de si tenemos control sobre su implementación. Si no es así, tendremos que adaptarnos a la implementación que ofrezca este (HTTP, mensajería...).

En el caso del servicio web, la acción correspondiente sería desplegar o eliminar instancias. El efector conocerá el sistema de despliegue (p.e. Docker) y cómo solicitar la activación o desactivación de un servicio.

\subsection{Sistemas distribuidos basados en elementos autónomos}

Si nos fijamos en la figura \ref{fig:autonomic-element}, veremos que en la parte superior del elemento autónomo figuran sondas y efectores. Esto nos indica que pueden actuar también como recursos manejados, reportando mediciones y ofreciendo efectores para manipularlo. Nos permite colocar un manejador autónomo que actúe como \textbf{orquestador}. \cite{ibmcorporationArchitecturalBlueprintAutonomic2006}

Los orquestadores gestionan uno o más elementos autónomos, responsabilizándose de tareas de más alto nivel. Facilitan también la cooperación entre sus elementos manejados. Por ejemplo, si nuestro elemento autonómico fuera un servidor web, el orquestador podría encargarse de gestionar varios servidores web distintos. Podría actuar como balanceador de carga u otros aspectos.

Por encima de los orquestadores tendríamos al administrador u \textbf{operario humano}. Como ya comentamos, este monitoriza el funcionamiento del sistema autónomo y lo gestionará mediante las políticas. Incluso puede participar en el proceso de toma de decisiones del bucle cuando este no cuenta con suficiente información para tomar una acción correctiva. Esto se conoce como \textbf{\foreign{english}{human in the loop}} (humano en el bucle). \cite{gilDesigningHumanLoop2016a}.

La arquitectura final tendría el siguiente aspecto, mostrado en la figura \ref{fig:autonomic-system}.

\begin{figure}[htb]
  \centering
  \includegraphics[scale=0.6]{cap_contexto_tecnologico/images/mape-k-architecture}
  \caption[Arquitectura de un sistema autoadaptativo basado en MAPE-K.]{Arquitectura de un sistema autoadaptativo basado en MAPE-K. Imagen obtenida de \cite{ibmcorporationArchitecturalBlueprintAutonomic2006}.}
  \label{fig:autonomic-system}
\end{figure}


\chapter{Sistema actual: plataforma FAdA}
\label{chap:sistema_original}

\textcolor{red}{En este capitulo describiremos el funcionamiento del sistema actual, aquel que queremos dividir en microservicios. Repasaremos su implementación, su estructura actual y qué queremos lograr. Exploraremos sus componentes y describiremos nuestros objetivos para dividirlo.}

\begin{wrapfigure}{r}{0.3\linewidth}
  \centering
  \includegraphics[scale=0.45]{cap_sistema_original/images/fada-logo}
\end{wrapfigure}

La plataforma FAdA\footnote{Página oficial: \url{http://fada.tatami.webs.upv.es/}} está enfocada en el desarrollo de sistemas autoadaptativos. Es un desarrollo del grupo PROS/Tatami\footnote{Página oficial: \url{http://www.pros.webs.upv.es/}} del instituto VRAIN/UPV\footnote{Página oficial: \url{https://vrain.upv.es/}}. Está compuesta por una serie de herramientas y guías metodológicas. Entre ellas encontramos extensiones de modelado para el IDE Eclipse, generadores de código y diversas implementaciones de bucles de control genéricos.

\section{Desarrollo de servicios \foreign{english}{Adaptive Ready}}

Comenzaremos describiendo la extensión para el IDE Eclipse. Tomando una aproximación de \textbf{desarrollo dirigido por modelos} (o \foreign{english}{model driven development}, MDD), esta herramienta permite diseñar soluciones autoadaptativas. Las soluciones están compuestas por especificaciones de servicios \textbf{\foreign{english}{adaptive ready}} (ARS). \cite{fonsServiciosAdaptivereadyPara2021} Se trata de servicios que ofrecen una serie de interfaces o \foreign{english}{touchpoints} que permiten manipularlos desde un bucle de control externo. En la figura \ref{fig:adaptive-ready-services} mostramos un esquema del servicio y de cómo el bucle orquesta la solución.

A partir de los modelos, podemos generar código para los servicios. La plataforma ofrece una serie de generadores para el lenguaje Java. Basándose en la especificación, generarán todo el código de infraestructura para dar soporte a estas interfaces de adaptación. Para integrarlas con nuestro sistema, deberemos implementar la lógica asociada a los comandos e importar el módulo en nuestra solución. Dependiendo del tipo de despliegue por el que optemos, pueden generar microservicios, módulos OSGi\footnote{Página oficial: \url{https://www.osgi.org}.}, entre otros.

Para operar estas soluciones, la plataforma ofrece distintas implementaciones de bucles de control. Dependen también del modelo de despliegue que elijamos. Disponemos de soluciones para gestionar microservicios sobre la plataforma Kubernetes\cite{fonsServiciosAdaptivereadyPara2021}, otros que permiten operar componentes OSGi\footnote{Página oficial: \url{https://www.osgi.org}.}, etc. Para este trabajo nos centraremos en una versión reducida del bucle de control para microservicios: MAPE-K \foreign{english}{Lite}.

\textcolor{red}{Los servicios adaptive ready son agnósticos a la la lógica de las adaptaciones. Simplemente implementan su funcionalidad y los contratos de adaptación. Desde el bucle de control reciben los comandos de adaptación a través de sus efectores.}

\begin{figure}[htb]
  \centering
  \includegraphics[scale=0.4]{cap_sistema_original/images/adaptive-ready-services}
  \caption[Servicio adaptive-ready y Bucle MAPE-K sobre arquitectura de mi-
  croservicios adaptive-ready.]{Servicio adaptive-ready (izquierda) y Bucle MAPE-K sobre arquitectura de microservicios adaptive-ready (derecha). Obtenida de \cite{fonsServiciosAdaptivereadyPara2021}.}
  \label{fig:adaptive-ready-services}
\end{figure}


\section{Bucle de control MAPE-K \foreign{english}{Lite}}

El bucle de control MAPE-K \foreign{english}{Lite} de FAdA nos permite gestionar soluciones autoadaptativas basadas en microservicios. Es una versión reducida del bucle presentado en \cite{fonsEspecificacionSistemasAutoadaptativos2021}. Sigue la arquitectura MAPE-K descrita en la sección \ref{sec:bucles-mapek}. En su implementación actual, se despliega como un servicio monolítico. Todos sus componentes, tanto del bucle como los del recurso manejado (monitores, reglas..), se ejecutan dentro del mismo proceso.

El bucle de control es genérico. No está acoplado a un dominio o solución concreta. Además pueden hacer uso de él varios sistemas. El proceso acepta como entradas las mediciones de las sondas y emite comandos dirigidos a los efectores del recurso manejado. En nuestro caso, los \foreign{english}{touchpoints} de los servicios adaptive ready. En la figura \ref{fig:bucle-mapek3} mostramos un modelo que detalla el flujo de control e información dentro de sus etapas.

\begin{figure}[htb]
  \centering
  \includegraphics[scale=1]{cap_introduccion/images/bucle-mape-k}
  \caption[Arquitectura de un Bucle MAPE-K. El flujo de información y de control entre las etapas del bucle están representados con flechas.]{Arquitectura de un Bucle MAPE-K. El flujo de información y de control entre las etapas del bucle están representados con flechas. Obtenida de \cite{fonsEspecificacionSistemasAutoadaptativos2021}}
  \label{fig:bucle-mapek3}
\end{figure}

\subsection{Estructura del bucle}

En la sección \ref{sec:estructura-mape-k} ya describimos la estructura de un bucle MAPE-K típico. Como vemos ver en la figura \ref{fig:bucle-mapek3}, este no difiere mucho en su implementación. Sigue el mismo proceso y utiliza los mismos componentes: sondas, monitores, módulo de análisis, planificador, ejecutor y efectores. Pero sí que hay una faceta en la que queremos hacer hincapié: el uso de reglas de adaptación en la etapa de análisis.

\subsubsection{Módulo de análisis y reglas de adaptación}

El módulo de análisis del bucle de control se ha implementado mediante un conjunto de \textbf{reglas de adaptación}. Se trata de la codificación de heurísticas que nos permiten corregir u optimizar el funcionamiento del sistema. En base a los reportes de los monitores, se buscan oportunidades para mejorar la configuración del sistema. Estas pueden ser generales o especificas a un recurso manejado concreto.

Las reglas de adaptación se dividen en dos componentes: la condición y la acción correctiva. La \textbf{condición} es una función que determina si es necesario ejecutar la acción. Se define a partir de las propiedades de adaptación del conocimiento. Esta se evaluará en tiempo de ejecución cuando se produzca un cambio en la configuración o se obtenga nueva información del entorno. cambie alguna de las propiedades de las que depende.

Por otro lado, la \textbf{acción correctiva} es una \textbf{propuesta de cambio} en la configuración del sistema. En ella se describe cuál debería ser la siguiente configuración. Las acciones se formulan usando operadores arquitectónicos. Los describiremos en la siguiente sección. El planificador recibirá sus propuestas y decidirá si es viable aplicarlas o no

\subsubsection{Operadores arquitectónicos}

Las acciones correctivas que realiza un sistema autoadaptativo se declaran usando \textbf{operadores arquitectónicos}. \cite{garlanIncreasingSystemDependability2003} Estas acciones modifican la configuración o la arquitectura del sistema en tiempo de ejecución. Dependiendo del estilo arquitectónico, tendremos disponibles unos operadores determinados. Por lo general suelen ser muy similares: añadir o eliminar componentes, modificar las conexiones entre ellos o cambiar parámetros de configuración.

El bucle de control que nos ocupa es específico para gestionar soluciones basadas en microservicios. Por tanto, ofrecerá los operadores arquitectónicos correspondientes a este estilo. En \cite{fonsEspecificacionSistemasAutoadaptativos2021} podemos encontrar los cinco tipos que ofrece:

\begin{itemize}
  \item \textbf{Desplegar servicios} (\textbf{\foreign{english}{deploy}}): Añadimos una nueva instancia de un servicio.

  \item \textbf{Eliminar servicios} (\textbf{\foreign{english}{undeploy}}): Eliminamos una instancia de un servicio.

  \item \textbf{Enlazar servicios} (\textbf{\foreign{english}{bind}}): Añadimos una conexión entre dos servicios. A partir de entonces podrán comunicarse.

  \item \textbf{Desenlazar servicios} (\textbf{\foreign{english}{unbind}}): Eliminamos una conexión existente entre dos servicios. Ya no podrán comunicarse.

  \item \textbf{Cambiar configuración} (\textbf{\foreign{english}{set parameter}}): Modificamos un parámetro de configuración de un servicio.
\end{itemize}

En la figura \ref{fig:adaptaciones-microservicios} mostramos un ejemplo de los efectos de estas acciones en la arquitectura del sistema.

\begin{figure}[htb]
  \centering
  \includegraphics[scale=1.8]{cap_sistema_original/images/adaptaciones}
  \caption[Ejemplo de adaptaciones en un sistema basado en microservicios.]{Ejemplo de adaptaciones en un sistema basado en microservicios. Imagen original obtenida de \cite{fonsServiciosAdaptivereadyPara2021}}
  \label{fig:adaptaciones-microservicios}
\end{figure}

\subsubsection{Ejemplo de regla de adaptación}


Respecto al servicio web, definiremos reglas tomando el valor del número de peticiones por segundo. Podemos definirlas con umbrales para este valor: si es muy alto, la regla solicita el despliegue de una nueva instancia. Cuando la carga baje, y si el servicio está replicado, podremos eliminarlas.

\textcolor{red}{Añadir ejemplo de regla de adaptación del climatizador}.


\chapter{Diseño de la solución}
\label{chap:diseño}

Como comentamos en el capítulo \ref{chap:introduccion}, el objetivo del trabajo es transformar un servicio monolítico en un sistema distribuido basado en microservicios. Se trata de un cambio arquitectónico importante. Queremos por tanto diseñar una estrategia ingenieril para llevar a a cabo la migración; teniendo en cuenta las particularidades del sistema.

El servicio en cuestión implementa un \textbf{bucle de control MAPE-K}\cite{ibmcorporationArchitecturalBlueprintAutonomic2006,fonsServiciosAdaptivereadyPara2021}, que ya describimos en la sección \ref{sec:bucles-mapek}. Por suerte, partimos de un sistema cuyos componentes presentan una división funcional clara (cada etapa del bucle). Nos facilitará definir las fronteras de nuestros servicios.

Debido a esto, el foco de este capítulo pasará a los \textbf{conectores de \emph{software}}. Necesitamos establecer qué estrategias de comunicación utilizaremos para comunicar los servicios.

\textcolor{red}{Buscar libros de descomposición de monolitos en microservicios.}

\section{Distribución de los componentes}

El primer paso fue identificar los componentes que compondrían nuestra arquitectura. Por suerte, partíamos de un sistema existente, con una arquitectura bien definida y documentada. Conocíamos el rol de cada uno de sus componentes y sus requisitos. Asi que el primer problema al que nos enfrentamos estaba relacionado con la distribución de los servicios. ¿Cómo definimos las fronteras entre cada uno de ellos? ¿Qué componentes debe abarcar cada microservicio?

En la figura \ref{fig:bucle-mapek2} presentamos otra vista de la arquitectura actual del bucle. Una de las decisiones que tomamos muy temprano en el diseño fue separar cada etapa del bucle en su propio servicio. Esto nos aportaba varios beneficios:

\begin{figure}[htb]
  \centering
  \includegraphics[scale=1.15]{cap_introduccion/images/bucle-mape-k}
  \caption[Arquitectura de un Bucle MAPE-K. El flujo de información y de control entre las etapas del bucle están representados con flechas.]{Arquitectura de un Bucle MAPE-K. El flujo de información y de control entre las etapas del bucle están representados con flechas. Obtenida de \cite{fonsEspecificacionSistemasAutoadaptativos2021}}
  \label{fig:bucle-mapek2}
\end{figure}

El más evidente es que nos permitía independizar la implementación de cada etapa. Si fuera necesario, podríamos emplear distintas tecnologías para cada una. Incluso podríamos ofrecer implementaciones alternativas de algunos componentes. Estos podrían ofrecer distintas estrategias de la misma funcionalidad, como podrían ser distintos planificadores. \textcolor{red}{TODO: cita sam newman}

Por otro lado, también nos permitirá escalar cada etapa de forma independiente. Si por ejemplo el servicio de análisis estuviera recibiendo más peticiones que las demás, no sería necesario instanciar el bucle completo. En su lugar, podemos limitarnos a desplegar solo una nueva instancia del componente afectado.

\textcolor{red}{Detectamos otra posible división de funcionalidad:} actualmente, el bucle está muy acoplado al dominio de sus recursos manejados. Todo corre bajo el mismo proceso: el bucle, los monitores, sus reglas de adaptación y demás elementos específicos de la solución. Ese proceso solo podrá manejar aquellos sistemas cuyos módulos tenga cargados.

Decidimos entonces desacoplarlos. Separar cada etapa del bucle MAPE-K de los elementos específicos de cada solución. Así, tendremos cada etapa como un microservicio agnóstico a una solución concreta; y por otro lado, estarán los servicios específicos para cada solución, con conocimiento del dominio del recurso manejado: monitores, reglas de adaptación, efectores\dots

Podremos entonces aprovechar la misma infraestructura para manejar varios sistemas simultáneamente (\emph{multi-tennancy}). \textcolor{red}{ampliar + cita}

En cuanto al despliegue, mantendremos el bucle a nivel de sistema, \cite{mendoncaGeneralityVsReusability2018} como funcionaba hasta ahora. Esto significa que se desplegará conjuntamente con los microservicios del recurso manejado. En la figura \ref{fig:mape-k-microservices} mostramos los microservicios que componen nuestra arquitectura.

\begin{figure}[htb]
  \centering
  \includegraphics[scale=0.25]{cap_diseño/images/mape-k-microservices}
  \caption{Diagrama con los componentes que forman nuestra arquitectura distribuida}
  \label{fig:mape-k-microservices}
\end{figure}

\textcolor{red}{Figura \ref{fig:mape-k-microservices}: Borrar los servicios específicos de planificador y ejecutor. Agrupar los servicios para poder aumentar zoom y hacerlo más legible. Añadir línea de divisón entre la capa del bucle y el dominio del recurso manejado.}

\section{Conectando los servicios}

El siguiente problema al que nos enfrentamos está relacionado con la comunicación: si dividimos estos componentes en microservicios, ¿cómo hacemos para que se comuniquen? Hay que tener en cuenta que estos pueden estar desplegados y replicados en distintas máquinas. No podemos asumir que están en el mismo \emph{host}.

Aprovechando la separación entre bucle de control y el dominio del recurso, investigamos posibles arquitecturas. Nos decantamos por \textbf{arquitecturas de servicios jerarquizados}. Queríamos explotar esta separación para mantener al bucle aislado del dominio de la solución. Dimos con el estilo arquitectónico C2 (\emph {components and connectors})\cite{taylorComponentMessagebasedArchitectural1996a, UCISoftwareArchitecture}, en el que nos hemos inspirado.

\subsection{Jerarquías de microservicios: Arquitectura C2 y arquitectura limpia}

Este estilo organiza sus componentes en jerarquías o capas: cada servicio se encuentra en un nivel determinado, según su nivel de abstracción. En las capas inferiores, se encuentran los servicios más externos, más ''acoplados'' al entorno. Por ejemplo, aquellos servicios que requieran de acceder al sistema de ficheros, estarían en esta capa. Por otro lado, en las capas superiores se encuentran los servicios en niveles de abstracción superior, que dependen lo mínimo del entorno.

En cuanto a la comunicación, un componente solo debe contactar con sus vecinos inmediatos (en una capa superior o inferior). Esto evita que el servicio pueda contactar con otras capas, limitando su alcance y su conocimiento del despliegue del sistema. Además, dentro del mismo nivel no pueden contactar entre ellos. Según la dirección de la comunicación, se emplean mecanismos distintos (figura \ref{fig:C2-arch-example}):

\begin{itemize}
  \item \textbf{Peticiones} (\emph{requests}): Se trata de solicitudes a un servicio para que ejecute una acción. Un componente se comunica directamente con un vecino en una capa superior. La petición viaja de ''fuera hacia dentro'' en cuanto al nivel de abstracción. Por ejemplo, una petición de un cliente a un servicio web podría estar bajo esta categoría.

  \item \textbf{Notificaciones}: Representan eventos ocurridos en el sistema. Un componente de más arriba en la jerarquía (más interno) envía un mensaje hacia abajo, sin especificar receptor. Todos los servicios por debajo lo recibirán, y decidirán si tratarlo o no. Esto evita que nuestro servicio se acople a aquellos que están por debajo (son más concretos). Un ejemplo sería notificar al resto de servicios sobre la creación de un nuevo usuario.
\end{itemize}

\begin{figure}[htb]
  \centering
  \includegraphics[scale=0.45]{cap_diseño/images/c2SampleArch}
  \caption[Ejemplo del estilo arquitectónico C2 (\emph{Components and Connectors})]{Ejemplo del estilo arquitectónico C2 (\emph{Components and Connectors}). \cite{UCISoftwareArchitecture}}
  \label{fig:C2-arch-example}
\end{figure}

Basándonos en este estilo, definimos las capas de nuestro sistema. Esto nos permitió dividir los microservicios en niveles y elegir los conectores más adecuados para cada tipo de comunicación.

Distinguimos cuatro niveles distintos, de menor nivel de abstracción a mayor:

\begin{itemize}
  \item \textbf{Nivel del recurso manejado}: En este nivel se encuentran las sondas y efectores. Son los elementos que tienen más contacto con el recurso manejado. Hacen de intermediarios entre este y el resto del bucle, para reducir su acoplamiento.

  \item \textbf{\textcolor{red}{Nivel de específico solución}}: En esta capa se encuentran componentes del bucle específicos para el dominio del recurso manejado. Monitores específicos, reglas de adaptación... No los incluimos en el mismo nivel que las sondas y efectores porque necesitamos comunicar con ellos. Además que guardan más relación con el bucle que con el recurso manejado.

  \item \textbf{Nivel del bucle}: Aquí se encuentran los servicios de las etapas del bucle: servicio de monitorización, análisis, planificación y ejecución. Esta capa debe ser agnóstica al dominio de los recursos manejados. Además, actúa como intermediario entre los servicios de la solución y el conocimiento. Limitan cómo acceder a él.

  \item \textbf{Conocimiento}: Es la capa más interna y la base de la arquitectura. No depende de ningún otro componente, por lo que tiene el nivel de abstracción más alto. Todos los componentes del nivel del bucle dependen de ella para funcionar.

\end{itemize}

Habiendo definido esta jerarquía, vimos ciertas similitudes con arquitecturas \emph{domain driven}, como \emph{Clean Architecture}. \cite{martinChapter22Clean2018} En ella, el sistema se organiza en base a una \textbf{regla de dependencia}: \emph{''la dependencia entre los componentes solo puede apuntar hacia dentro, hacia políticas de alto nivel''}. Es decir, la arquitectura se organiza en capas concéntricas. En el centro se encuentra el dominio, con el mayor nivel de abstracción. Este no tiene dependencias con ninguna capa exterior. Por otro lado, cada capa más externa tiene dependencias sólo con la capa a la que envuelve. Sólo puede comunicarse con componentes dentro de esta.

Basándonos en la descripción anterior, nuestra capa central será la del conocimiento. A partir de ahí, cada nivel superior dependería de aquel al que ''envuelve'': el bucle al conocimiento, la solución al bucle...Por tanto, para que nuestra arquitectura sea más comprensible, optamos por representarla los diagramas de \emph{Clean Architecture} para representarlo. En la figura \ref{fig:clean-mapek-architecture} mostramos el resultado:

\begin{figure}[htb]
  \centering
  \includegraphics[scale=0.45]{cap_diseño/images/clean-arch-2-MAPEK-style-small}
  \caption[Representación de nuestra propuesta arquitectónica. Inspirado en Arquitectura Limpia (\emph{Clean Architecture}). Las flechas negras representan las peticiones, y las moradas, las notificaciones.]{Representación de nuestra propuesta arquitectónica. Inspirado en Arquitectura Limpia (\emph{Clean Architecture}). Las flechas negras representan las peticiones, y las moradas, las notificaciones. \footnotemark }
  \label{fig:clean-mapek-architecture}
\end{figure}

\footnotetext{Imagen original de arquitectura limpia obtenida de: \url{https://threedots.tech/post/ddd-cqrs-clean-architecture-combined/}}

\subsection{Definiendo los mecanismos de comunicación}

Como comentamos antes, vamos a inspirarnos en los mecanismos de comunicación descritos por C2: las peticiones y notificaciones. Pero, durante nuestra etapa de prototipado, nos dimos cuenta que estos no cubren todas nuestras necesidades. Hay dos casos que no están contemplados: la comunicación del módulo de análisis con el planificador, y la del planificador con el ejecutor. Ambos módulos se encuentran en la misma capa. Y, como dependen del conocimiento, no podemos moverlos a una superior para utilizar notificaciones.

\textcolor{red}{Las notificaciones no nos sirven, ya que la comunicación es entre dos módulos específicos. Aunque nos interesa el desacoplamiento entre módulos que ofrecen. Las peticiones tampoco casan del todo, ya que requerimos desacoplar los módulos. Deberían mantener su independencia en el mayor grado posible.} Por ello, requerimos de un tercer patrón de comunicación. Una combinación de ambos: las peticiones asíncronas.

Los tres patrones de comunicaciones que usaremos entonces son:

\begin{itemize}
  \item \textbf{Peticiones síncronas}: Comunicaciones síncronas dirigidas a un servicio determinado. Solo permitidas entre servicios de una capa más externa a un servicio en la capa interior adyacente.

  \item \textbf{Peticiones asíncronas}: Comunicaciones asíncronas dirigidas a un tipo de servicio determinado. Se trata de peticiones de trabajo asíncronas: se envían y el destinatario lo procesará cuando pueda. El cliente continuará su ejecución, sin esperar respuesta. \emph{fire and forget}.

  Para evitar el acoplamiento entre los componentes, deberemos buscar un conector que permita enviar el mensaje sin conocer específicamente al destinatario.

  Este mecanismo de comunicación solo está permitido entre elementos del mismo nivel.

  \item \textbf{Notificaciones}: Comunicaciones asíncronas no dirigidas. El servicio publica un evento que potencialmente recibirán todos los servicios en la capa exterior adyacente. El cliente lo envía y continua su ejecución, sin esperar respuesta.
\end{itemize}

\subsection{Conectores}

Una vez determinadas las necesidades de comunicación de nuestro sistema, debemos buscar los conectores adecuados. Seguimos la estrategia descrita en \cite{taylorSoftwareArchitectureFoundations2009} para elegir conectores; y nos basamos en los patrones de comunicación en sistemas distribuidos descritos en \cite{newmanBuildingMicroservicesDesigning2021}.

Comenzamos investigando las peticiones síncronas. Tomemos por ejemplo la comunicación entre el servicio de monitorización (\emph{monitoring service}) y el servicio de conocimiento (\emph{knowledge service}). Recordemos que el servicio de conocimiento almacena todas las propiedades de adaptación. El resto de servicios necesitan consultarlas y actualizarlas durante su funcionamiento. En la figura \ref{fig:monitor-knowledge-initial} representamos inicialmente ambos componentes y un conector, sin especificar de qué tipo será.

% TODO: Cambiar por imagen de componentes, que ofrezcan y requieran interfaces.
\begin{figure}[htb]
  \centering
  \includegraphics{cap_diseño/images/Monitor-Knowledge-Initial-Connector}
  \caption{Boceto inicial: queremos conectar el servicio de monitorización con la base de conocimiento para poder leer propiedades de adaptación.}
  \label{fig:monitor-knowledge-initial}
\end{figure}

El siguiente paso es identificar qué interacciones debe existir entre ambos componentes. En este caso, el servicio de monitorización debe contactar con el servicio de conocimiento para leer y actualizar el valor de las propiedades.Por tanto, existen operaciones de lectura y escritura de los datos. Además, como es una comunicación ''descendente'' (\emph{monitoring service} está en la capa superior), el patrón a utilizar serán las peticiones síncronas.

Ahora, debemos identificar qué \textbf{tipos de conector} serían adecuados para este patrón. Sabiendo que hemos optado por una arquitectura distribuida, la elección se simplifica: los servicios pueden estar desplegados en máquinas distintas, por tanto el paso de mensajes será a través de la red.

Conociendo esto, en lugar de recurrir a la taxonomía que lista \cite{mehtaTaxonomySoftwareConnectors2000}, optamos por consultar las estrategias de comunicación habituales para sistemas distribuidos descritas en \cite{newmanBuildingMicroservicesDesigning2021}. Se trata de cuatro mecanismos distintos: Invocación a métodos remotos (\emph{Remote Procedure Call}), APIs REST, consultas con GraphQL o \emph{brokers} de mensajería. Tuvimos que evaluarlos mediante un análisis de \emph{trade-offs} para determinar las ventajas y desventajas de cada uno.

\textcolor{red}{Smart endpoints, dumb pipes: https://simplicable.com/new/smart-endpoints-and-dumb-pipes}

\textbf{Invocación de métodos remotos} o (\emph{\textbf{Remote Procedure Call}}): Esta patrón se basa en el estilo cliente-servidor. Un servidor expone una serie de funciones que el cliente puede invocar mediante peticiones a través de la red. Estas peticiones incluyen el nombre de la función a ejecutar y sus parámetros. Al finalizar la ejecución, el servidor puede devolver su resultado, si lo hubiera. Existen varios protocolos que implementan este mecanismo como gRPC o SOAP.

Una evolución de RPC suele emplearse en la programación orientada a objetos: el paradigma de \textbf{objetos distribuidos}. \cite{tanenbaumChapter10Distributed2007} En este caso, el programa cliente puede interactuar con objetos en servidores remotos como si fueran locales. Esta interacción se realiza a través de objetos que actúan como \emph{proxies}, abstrayendo de la llamada al servidor.

Los \emph{proxies} ofrecen una interfaz para que el cliente invoque sus métodos localmente. Internamente, estos métodos realizan una llamada al servicio remoto donde se encuentra el objeto realmente. El servidor remoto procesa la petición y nos devolverá un resultado. Así, abstraen al cliente de todo este proceso de comunicación. En la figura \ref{fig:rpc-distributedobjects} tenemos un esquema de este mecanismo.

\begin{figure}[htb]
  \centering
  \includegraphics[scale=1.5]{cap_diseño/images/rpc-distributedobjects}
  \caption[Funcionamiento del sistema de objetos distribuidos]{Funcionamiento del sistema de objetos distribuidos. \cite{tanenbaumChapter10Distributed2007}}
  \label{fig:rpc-distributedobjects}
\end{figure}

\begin{itemize}
  \item \textbf{Ventajas}:
  \begin{itemize}
    \item Permite distribuir la carga de procesamiento del sistema. Esto puede ayudar para escalar la aplicación.

    \item Abstrae al cliente de la interacción con un servidor remoto. Le resulta prácticamente indistinguible de un objeto local.

    \item \textcolor{red}{Los \emph{proxies} o (\emph{stubs} en la terminología de RPC) suelen generarse a partir de un contrato que define qué operaciones ofrecen estos objetos. Por ejemplo: SOAP con WDSL, gRPC; o en el caso de objetos distribuidos, Java RMI.} ¿Y la ventaja?
  \end{itemize}

  \item \textbf{Desventajas}:
  \begin{itemize}
    \item No se puede abstraer completamente al cliente de las llamadas a través de la red. Pueden darse errores que no ocurrirían durante una invocación de un método sobre un objeto local. Por ejemplo, que el servidor no esté disponible. \cite{jausovecFallaciesDistributedSystems2020}

    \item Dificulta la integración con otras aplicaciones. Cada servicio ofrece sus propias funciones distintas. No están estandarizadas.

    \item Si adoptamos sistemas como Java RMI, nuestro sistema se acopla a esa tecnología concreta. \cite{newmanBuildingMicroservicesDesigning2021}. Nos quita flexibilidad en cuanto a qué otras tecnologías podemos utilizar en nuestra arquitectura.

    \item El cliente debe actualizarse y recompilarse con cada cambio en el esquema del servidor. Esto puede ser problemático para casos donde tenemos que desplegar una actualización para que nuestros clientes puedan continuar utilizando la aplicación.
  \end{itemize}
\end{itemize}

\textbf{\emph{Representational State Transfer} (REST)}: Se basa también en el estilo arquitectónico cliente-servidor, pero con ciertas restricciones adicionales. \cite{taylorSoftwareArchitectureFoundations2009} Su concepto principal son los \textbf{recursos}: cualquier elemento sobre el que la API pueda ofrecernos información; y que pueda tener asociado un identificador único (una URI). \cite{richardsonRESTfulWebServices2007} Por ejemplo, podrían ser las entidades del dominio que gestiona nuestro servicio: usuarios, mediciones de temperaturas\dots

Las acciones que podemos ejecutar sobre los recursos (leer, crear, actualizar, \dots) las define el protocolo de comunicación sobre el que se implemente. Gracias a esto, la API que pueden ofrecer los servicios REST es común. Solo cambia el ``esquema de los datos``, los tipos de recursos que ofrecen. Esto facilita enormemente la integración con otros servicios. \cite{nallyRESTVsRPC2018} La implementación más habitual es sobre el protocolo HTTP. Define métodos estandarizados como \emph{GET} para las lecturas, \emph{PUT} para las actualizaciones, etc.

\begin{itemize}
  \item \textbf{Ventajas}:

  \begin{itemize}
    \item \textbf{\emph{Stateless}}: El servidor no mantiene el estado de la sesión del cliente. Esto permite que cada petición sea independiente de las demás.

    \item \textbf{Escalable}: Como las sesiones deben ser \emph{stateless}, podremos replicar nuestro servicio y que distintas instancias puedan atender las peticiones que surjan durante una misma sesión.

    \item \textbf{API Sencilla}: Solo hay que implementar unos pocos métodos estándar para interactuar con la API.

    \item \textbf{Comunicación síncrona}: Es el mecanismo ideal para comunicaciones síncronas, donde el cliente requiere la respuesta del servicio para poder continuar con su procesamiento. También podemos dar soporte a para comunicaciones \emph{fire and forget}, donde el cliente envía un mensaje y no espera ninguna respuesta a su petición.

    \item \textbf{Interoperabilidad}: Ampliamente utilizado en servicios de Internet. Es ideal para que clientes externos contacten con nuestro sistema mediante peticiones síncronas. \cite{newmanBuildingMicroservicesDesigning2021}

    \item \textbf{Generación de clientes}: Para facilitar la comunicación con APIs REST, podemos generar librerias cliente utilizando el estándar OpenAPI. Lo explicaremos con maś detalle en la sección \ref{chap:OpenAPI}.
  \end{itemize}

  \item \textbf{Desventajas}:

  \begin{itemize}
    \item \textbf{Dirigida}: Necesitamos conocer de antemano la ubicación del servidor al que queremos hacer una petición.

    \item \textbf{Rendimiento}: El rendimiento es peor comparado con mecanismos RPC. El tamaño de un mensaje HTTP serializado en XML o JSON es mayor que si estuviera en un formato binario.

    \item \textbf{API Sencilla}: También es una desventaja. Hay operaciones complejas que pueden ser difíciles de representar con los métodos ofrecidos por el protocolo de comunicación. Pueden requerir más tiempo de diseño, o incluso, ser implementados siguiendo el patrón RPC.
  \end{itemize}
\end{itemize}

\textbf{GraphQL}\footnote{Página oficial: \url{https://graphql.org/}} \textcolor{red}{AMPLIAR}: Se trata de un protocolo de consultas. Permite a los clientes ejecutar consultas personalizadas sobre los datos de un servidor. No necesitan de lógica específica para ejecutarla. De esta forma, el cliente puede obtener toda la información que necesita. Reduce el número de peticiones ejecutadas. También evita traerse datos innecesarios.

\begin{itemize}
  \item \textbf{Ventajas}:

  \begin{itemize}
    \item \textbf{Ideal para móviles}: Gracias a que reduce la cantidad de llamadas, es ideal para entornos donde queremos optimizar el uso de red.

    \item \textbf{Rendimiento}: Ofrece un mayor rendimiento comparado con otras alternativas que no ofrezcan un endpoint ya implementado. Y debamos obtener la misma información por composición, haciendo varias llamadas.
  \end{itemize}

  \item \textbf{Desventajas}:

  \begin{itemize}
    \item \textbf{Solo permite lecturas}: Es un lenguaje de consultas. No tiene comandos que permita escrituras.

    \item \textbf{Solo permite lecturas síncronas}:

    \item \textbf{Exponemos datos a la red}:

    \item \textbf{Problemas de rendimiento}: El cliente puede hacer consultas muy pesadas que penalicen el rendimiento de la base de datos sobre la que opera nuestro servicio.
  \end{itemize}
\end{itemize}

\textbf{\foreign{english}{Brokers} de mensajería}: Es un mecanismo de \textbf{comunicación asíncrona} muy popular. Sobre todo en arquitecturas basadas en eventos. Contamos con un servicio que actúa como intermediario, el \emph{broker}. Este gestiona la comunicación entre los servicios del sistema. \cite{newmanBuildingMicroservicesDesigning2021} Hay varias estrategias de comunicación posibles: colas de trabajo, \emph{publish-suscribe}, híbrida\dots

Tomemos por ejemplo las \textbf{colas de trabajo}. \cite{royChapterMessagePatterns2017} Es una estrategia para implementar comunicaciones asíncronas dirigidas. Nos permiten desacoplar la comunicación entre componentes usando colas de mensajería como intermediarias. Para ello, un servicio, el productor, publica mensajes en la cola. Estos mensajes representan peticiones de trabajo que pueden ser costosas de procesar. Un servicio, el trabajador, estará la escucha de los mensajes que llegan y los irá consumiendo. Estos mensajes se procesan siguiendo un orden FIFO (\emph{first in, first out}). En la figura \ref{fig:work-queues} mostramos un ejemplo con dos consumidores, a la escucha de la misma cola.

\begin{figure}[htb]
  \centering
  \includegraphics[scale=0.65]{cap_diseño/images/work-queues}
  \caption[Representación de las colas de trabajo. Ejemplo de comunicación asíncrona dirigida.]{Representación de las colas de trabajo. Ejemplo de comunicación comunicación asíncrona dirigida. \footnotemark }
  \label{fig:work-queues}
\end{figure}

\footnotetext{Imagen obtenida de: \url{https://www.rabbitmq.com/tutorials/tutorial-two-dotnet.html}}

Otra estrategia posible es \foreign{english}{publish-suscribe}: sirve para implementar comunicación \emph{multicast}. Se basa en el uso de \textbf{temas} o \textbf{\foreign{english}{topics}}: categorías de mensajes que pueden resultar de interés. Un servicio (el productor) envía un mensaje al \foreign{english}{broker}, indicando que pertenece a un tema determinado. El \emph{broker} recibe el mensaje y se encarga de reenviarlo a todos los servicios subscritos a este tema en concreto. \cite{rabbitmqPublishSubscribeDocumentation} En la figura \ref{fig:publish-subscribe} tenemos un ejemplo de esta estrategia.

\textcolor{red}{Describir fanout}
\textcolor{red}{Describir exchanges}

\begin{figure}[htb]
  \centering
  \includegraphics[scale=0.5]{cap_diseño/images/publish_subscribe}
  \caption[Estrategia \emph{publish/suscribe}: el \emph{broker} actúa como intermediario en la comunicación \emph{multicast}.]{Estrategia \emph{publish/suscribe}: el \emph{broker} actúa como intermediario en la comunicación \emph{multicast}. Imagen obtenida de \footnotemark}
  \label{fig:publish-subscribe}
\end{figure}

\footnotetext{Java Messaging Service: \url{https://docs.oracle.com/cd/E19509-01/820-5892/ref_jms/index.html}}

La mayor ventaja de este estilo de comunicación es el \textbf{desacoplamiento} entre los servicios. \cite{korabUnderstandingMessageBrokers2017}
Ninguno de ellos necesita conocer detalles sobre cómo están desplegado los otros: su dirección, el número de instancias, si están activos en este momento, etc. Solo necesitan conocer el formato de los mensajes y la dirección del \emph{broker} para enviarlos o recibirlos.

\begin{itemize}
  \item \textbf{Ventajas}:

  \begin{itemize}
    \item \textbf{Comunicación asíncrona}: El servicio no necesita quedarse a la espera de una respuesta del servidor. Puede procesar otras operaciones hasta que se le notifique del resultado, si lo hubiera.

    \item \textbf{Desacoplamiento de los servicios}: Ni los productores ni los consumidores necesitan conocer el origen o destino de sus mensajes. Solo su formato y la dirección del \emph{broker}.

    \item \textbf{Envío garantizado de mensajes}: El \emph{broker} garantiza que el mensaje será entregado \emph{al menos} una vez al consumidor. Reintentará el reenvío hasta que se confirme su recepción.

  \end{itemize}

  \item \textbf{Desventajas}:

  \begin{itemize}
    \item \textbf{Requisitos de infraestructura}: Utilizar un \emph{broker} de mensajería puede incrementar la dificultad de nuestros despliegues. El \emph{broker} puede convertirse en un punto de fallo. Para operar de forma fiable, estos sistemas requieren de replicación. \cite{newmanBuildingMicroservicesDesigning2021}

    \item \textbf{Envío garantizado de mensajes}: Para poder garantizar el envío de un mensaje, el \emph{broker} puede recurrir a reenviarlo. Debemos diseñar nuestros sistemas de forma que estos mensajes duplicados sean descartados si ya han sido procesados.
  \end{itemize}
\end{itemize}

En la tabla \ref{tab:comparativa-mecanismos-comunicacion} presentamos un resumen de esta comparativa:

\begin{longtable}{|p{4.4cm} | c | c | c | c|}
  \hline
  & \textbf{RPC} & \textbf{REST} & \textbf{GraphQL} & \textbf{Broker mensajería} \\
  \hline
  \textbf{Tipo de comunicación entre componentes} & Directa & Directa & Directa & Directa y \emph{Multicast} \\
  \hline
  \textbf{Acoplamiento entre componentes} & Alto & Medio & Alto & Bajo \\
  \hline
  \textbf{Interoperabilidad} & Baja & Alta & Alta & Alta\footnotemark \\
  \hline
  \textbf{Comandos de lectura} & Sí & Sí & Sí & Sí \\
  \hline
  \textbf{Comandos de escritura} & Si & Sí & No & Sí \\
  \hline
  \textbf{Comunicación síncrona} & Sí & Sí & Sí & No \\
  \hline
  \textbf{Comunicación asíncrona} & No & Sí & No & Si \\
  \hline
  \caption{Comparativa de los mecanismos de comunicación.}
  \label{tab:comparativa-mecanismos-comunicacion}
\end{longtable}

\footnotetext{Depende de si tenemos control sobre los componentes que queremos integrar.}


--------------------------------------------------------------

Ahora analizaremos qué protocolo elegimos para cada mecanismo de comunicación.

\subsection{Peticiones síncronas}

De estas cuatro opciones, podemos descartar inmediatamente la opción de GraphQL. Se trata de un conector más orientado a las consultas de datos. En nuestro caso, necesitamos ejecutar también escrituras de los valores de las propiedades. Aunque podría ser interesante para consultas más avanzadas, utilizar dos protocolos de comunicación en paralelo aumentaría la complejidad de la arquitectura.

También optamos por descartar el \emph{broker} de mensajería. Como requerimos de comunicación directa, nos convenía que esta fuera síncrona. Para obtener propiedades del conocimiento, resultaba más sencillo de implementar mediante comunicación síncrona.

Finalmente, hay que tener en cuenta que una de nuestras prioridades es la \textbf{interoperabilidad}: es una API expuesta ''hacia fuera'', hacia una capa más externa; prima por tanto la compatibilidad con cualquier tipo de cliente. Descartamos entonces RPC, dado que nos acoplaría a una tecnología concreta y a APIs no estándares.

Terminamos por tanto decantándonos por implementar la comunicación utilizando un conector REST sobre HTTP. Implementaremos ambas funciones mediante \emph{endpoints} HTTP. Su especificación se detalla a continuación en las tablas \ref{tab:especificacion-get-property} y \ref{tab:especificacion-put-property}.

\newsavebox\getpropertyrequestbox
\begin{lrbox}{\getpropertyrequestbox}
  \begin{minipage}[t]{1in}
    \begin{verbatim}
Request:
HTTP GET property/currentTemperature

Response: 200 Ok
{
  value: {
    "Value":16.79,
    "Unit": 1, // Celsius
    "ProbeId":"c02234d3-329c-4b4d-aee0-d220dc25276b",
    "DateTime":"2022-01-15T18:19:38.5231231Z"
  },
  lastModification: "2022-01-15T18:19:39.123213Z"
}
    \end{verbatim}
  \end{minipage}
\end{lrbox}

\begin{table}[htb]
  \centering

  \begin{tabular}{|m{3.4cm}|p{2.5cm}|p{1cm}|p{3cm}|}
      \hline

      \textbf{Operación HTTP} & GET & \textbf{Ruta} & property/\{\emph{propertyName}\} \\
      \hline

      \textbf{Descripción} & \multicolumn{3}{|l|}{Devuelve el valor de la propiedad, si existe.} \\
      \hline

      \textbf{Parámetros} & \emph{propertyName} & \multicolumn{2}{|m{0.55\linewidth}|}{El nombre de la propiedad que deseamos obtener. Se lee a partir de la ruta de la petición.}\\
      \hline

      \multirow{3}*{\textbf{Respuestas posibles}}
            & \textbf{Código 200 (Ok)} & \multicolumn{2}{|m{0.55\linewidth}|}{La propiedad se ha encontrado. Incluye un \emph{payload} con el siguiente esquema:

            \begin{itemize}
              \item \emph{Value}: Valor de la propiedad serializado en JSON.
              \item \emph{LastModification}: Fecha y hora de la última modificación de esta propiedad.
            \end{itemize}}\\

            \cline{2-4}

            & \textbf{Código 400 (Bad request)} & \multicolumn{2}{|m{0.55\linewidth}|}{La petición está mal formada, no es acuerdo al contrato.}\\

            \cline{2-4}

            & \textbf{Código 404 (Not found)} & \multicolumn{2}{|m{0.55\linewidth}|}{No se ha encontrado ninguna propiedad con el nombre proporcionado.}\\
      \hline

      \textbf{Ejemplo} & \multicolumn{3}{|b{0.7\linewidth}|}{Petición para obtener la propiedad \emph{currentTemperature}:
      \usebox\getpropertyrequestbox} \\

      \hline
  \end{tabular}

  \caption{Especificación de la operación para obtener una propiedad del servicio de conocimiento.}
  \label{tab:especificacion-get-property}
\end{table}

\newsavebox\putpropertyrequestbox
\begin{lrbox}{\putpropertyrequestbox}
  \begin{minipage}[t]{2in}
    \begin{verbatim}
Request:
HTTP PUT property/currentTemperature

{
  value: {
    "Value":16.79,
    "Unit": 1, // Celsius
    "ProbeId":"c02234d3-329c-4b4d-aee0-d220dc25276b",
    "DateTime":"2022-01-15T18:19:38.5231231Z"
  }
}

Response: 204 (No content)
        \end{verbatim}
  \end{minipage}
\end{lrbox}

\begin{table}[htb]
  \centering

  \begin{tabular}{|m{3.4cm}|m{2.5cm}|b{1cm}|b{3cm}|}
      \hline

      \textbf{Operación HTTP} & PUT & \textbf{Ruta} & property/\{\emph{propertyName}\} \\
      \hline

      \textbf{Descripción} & \multicolumn{3}{|b{0.7\linewidth}|}{ Actualiza (o crea, si no existe) el valor de la propiedad con el nombre dado.} \\
      \hline

      \multirow{2}*{\textbf{Parámetros}}
            & \emph{propertyName} & \multicolumn{2}{|b{0.55\linewidth}|}{El nombre de la propiedad que deseamos crear o actualizar. Se lee a partir de la ruta de la petición.}\\

            \cline{2-4}

            & \emph{SetPropertyDTO} & \multicolumn{2}{|b{0.55\linewidth}|}{ Un DTO que contiene el valor a asignar en la propiedad serializado en JSON. El DTO se encuentra en el cuerpo de la petición.} \\
      \hline

      \multirow{2}*{\textbf{Respuestas posibles}}
            & \textbf{Código 204 (No content)} & \multicolumn{2}{|b{0.55\linewidth}|}{La propiedad se ha creado o actualizado correctamente. No incluye \emph{payload} en el cuerpo de la respuesta.}\\

            \cline{2-4}

            & \textbf{Código 400 (Bad request)} & \multicolumn{2}{|b{0.55\linewidth}|}{La petición está mal formada, no es acuerdo al contrato.}\\
      \hline

      \textbf{Ejemplo} & \multicolumn{3}{|b{0.7\linewidth}|}{Petición para actualizar la propiedad \emph{currentTemperature} con una medición de un termómetro:
      \usebox\putpropertyrequestbox} \\

      \hline
  \end{tabular}

  \caption{Especificación de la operación para actualizar o crear una propiedad del servicio de conocimiento.}
  \label{tab:especificacion-put-property}
\end{table}

Una vez definida la interfaz que expondrá el servicio de conocimiento, nos queda definir cómo se contactará desde el servicio de monitorización. ¿Implementamos las llamadas manualmente con un cliente HTTP? Aunque no sería muy complicado, tendríamos que mantenerlo manualmente cuando evolucione el sistema. Optamos entonces por una alternativa: el estándar OpenAPI.

\subsubsection{Open API}
\label{chap:OpenAPI}

\begin{wrapfigure}{r}{0.3\linewidth}
  \vspace{5pt}
  \includegraphics[scale=0.32]{cap_diseño/images/openapi-logo}
  \centering
  \vspace{5pt}
\end{wrapfigure}

OpenAPI es un lenguaje estándar para describir APIs RESTful. Nos permite describir de forma estructurada las operaciones que ofrece un servicio HTTP, manteniéndose agnóstico a su implementación. Esta descripción ayuda tanto a humanos como a computadoras a descubrir y utilizar las funcionalidades de la API. La OpenAPI Initiative (OAI) dirige el proyecto bajo el manto de la \emph{Linux Foundation}.

Un documento OpenAPI habitual documenta el funcionamiento de la API y el conjunto de recursos que la componen. Describe las operaciones HTTP que podemos ejecutar sobre estos recursos, incluyendo las estructuras de datos que recibe o envía y los códigos de respuesta. Estos códigos indican al cliente el resultado de la ejecución de la operación. \cite{openapi_initiativeOpenAPISpecificationV3} Más adelante mostraremos un ejemplo, con el \textcolor{red}{fragmento} \ref{ls:openapi-get}.

La especificación puede escribirse manualmente o puede generarse a partir de una implementación existente. Así, podemos desarrollar nuestro servicio en un determinado lenguaje y obtener su descripción en OpenAPI. Podemos aprovecharla en varios ámbitos del desarrollo, gracias a la gran variedad de herramientas existentes: generación de documentación, generación de casos de prueba, identificar cambios incompatibles, etc. \cite{westerveldChapterOpenAPIAPI2021}

Uno de los casos de uso más interesantes es la generación de código a partir de la definición. Existen una serie de generadores\footnote{\url{https://github.com/OpenAPITools/openapi-generator}} capaces de generar clientes o servidores conformes a la especificación. Ofrecen soporte a una gran variedad de lenguajes: Java, C\#, JavaScript\dots En el caso de cliente, actúa como un proxy que nos abstrae de la lógica de comunicación con el servidor, similar a lo descrito en el apartado de RPC.

Para el desarrollo de este trabajo, nos interesaba especialmente debido a las diferencias tecnológicas existentes: el bucle MAPE-K original estaba desarrollado en Java, pero el prototipo se desarrolló con el lenguaje C\# junto con el framework ASP.NET Core. Se tomó esta decisión para reducir el tiempo de aprendizaje y centrar los esfuerzos en la definición de la arquitectura del sistema.

\textcolor{red}{Gracias a la generación de código, pudimos obtener la especificación de los servicios desarrollados en ASP.NET Core, y generar clientes o servidores en cualquier lenguaje soportado, Java incluido. El bucle MAPE-K original después podría ser refactorizado usando este código autogenerado.}

A continuación explicaremos brevemente cómo utilizamos OpenAPI para documentar nuestras APIs y generar la especificación estas. Para ello, continuaremos con el ejemplo del servicio de conocimiento que hemos descrito a lo largo de este capítulo. Vamos a centrarnos en la implementación de la operación para obtener una propiedad del conocimiento, que describimos en la tabla \ref{tab:especificacion-get-property}.

En el \textcolor{red}{fragmento} \ref{ls:csharp-get}, podemos observar que se trata de un método C\# llamado \emph{GetProperty}. Su implementación es sencilla: busca en un diccionario la propiedad cuyo nombre se le pasa por parámetro. En caso de encontrarla, devuelve su valor con un código 200 OK. En caso contrario, devuelve un código de error que describe qué ha ocurrido exactamente (llamada incorrecta o no se ha encontrado la propiedad).

Aparte de la implementación, podemos comprobar que el método se ha decorado con una serie de comentarios (líneas 1-8) y atributos (10-12). Esta documentación describe qué hace el método, sus entradas y posibles respuestas. OpenAPI es capaz de utilizar estos elementos opcionales para generar una especificación más completa. Por tanto, resulta muy recomendable utilizarlos.

\begin{lstlisting}[language={[Sharp]C},caption={Implementación del método GetProperty decorado para generar la especificación OpenAPI.},captionpos=b, label=ls:csharp-get]
/// <summary>
///    Gets a property given its name.
/// </summary>
/// <param name="propertyName"> The name of the property to find. </param>
/// <returns> An IActionResult with result of the query. </returns>
/// <response code="200"> The property was found. Returns the value of the property. </response>
/// <response code="404"> The property was not found. </response>
/// <response code="400"> There was an error with the provided arguments. </response>
[HttpGet("{propertyName}")]
[ProducesResponseType(typeof(PropertyDTO), StatusCodes.Status200OK)]
[ProducesResponseType(StatusCodes.Status404NotFound)]
[ProducesResponseType(StatusCodes.Status400BadRequest)]
public IActionResult GetProperty([FromRoute]string propertyName)
{
    if (string.IsNullOrEmpty(propertyName))
    {
        return BadRequest();
    }

    bool foundProperty = properties.TryGetValue(propertyName, out PropertyDTO property);

    if (!foundProperty)
    {
        return NotFound();
    }

    return Ok(property);
}
\end{lstlisting}

Haciendo uso de las librerías de OpenAPI, generamos la especificación a partir del servicio de conocimiento. En el \textcolor{red}{fragmento} \ref{ls:openapi-get}, podemos ver cómo se describe la operación en este estándar:

\begin{lstlisting}[language=python,caption={Especificación OpenAPI del método para obtener una propiedad del conocimiento (\lstinline{GetProperty}).},captionpos=b, label=ls:openapi-get]
"paths": {
  "/Property/{propertyName}": {
    "get": {
      "tags": [
        "Property"
      ],
      "summary": "Gets a property given its name.",
      "parameters": [
        {
          "name": "propertyName",
          "in": "path",
          "description": "The name of the property to find.",
          "required": true,
          "schema": {
            "type": "string"
          }
        }
      ],
      "responses": {
        "200": {
          "description": "The property was found. Returns the value of the property.",
          "content": {
            "application/json": {
              "schema": {
                "$ref": "#/components/schemas/PropertyDTO"
              }
            }
          }
        },
        "404": {
          "description": "The property was not found.",
        },
        "400": {
          "description": "There was an error with the provided arguments.",
        }
      }
    }
  }
\end{lstlisting}

Podemos apreciar que en la ruta (\emph{/Property/\{propertyName\}}) está disponible una operación de tipo \emph{get} y que acepta determinados parámetros y ofrece unas posibles respuestas. Aparece una referencia a otro esquema (línea 25), que representa la estructura de la respuesta en ese caso concreto. También aparecen los comentarios opcionales que indicamos en el \textcolor{red}{fragmento} \ref{ls:csharp-get}. Encontramos grandes similitudes con la especificación presentada en la tabla \ref{tab:especificacion-get-property}.

Los convenios de los generadores de código de OpenAPI pueden no ser de nuestro agrado. Por ejemplo, pueden resultar muy verbosos o puede resultar muy pesado trabajar con DTOs directamente. Por suerte, tenemos dos opciones para solventar esto: Modificar las plantillas de generación de código. Al ser de código abierto, podríamos modificar las existentes o crear nuestras propias plantillas con nuestros propios convenios.

Otra opción, más fácil de implementar, es desarrollar código por encima del API Client Generado. Es el caso del servicio de Análisis. Como trabajar con DTOs directamente se hacía muy pesado, optamos por implementar un `system configuration request` builder. Esto nos permitia configurar la petición de una forma más descriptiva para el usuario:

\begin{lstlisting}[language={[Sharp]C},caption={Implementación de la misma petición siguiendo el patrón \emph{builder}.},captionpos=b, label=ls:api-cliente-request-builder]
var changeRequests = new List<ServiceConfigurationDTO>
{
  new()
  {
    ServiceName = ClimatisationAirConditionerConstants.AppName,
    IsDeployed = true,
    ConfigurationProperties = new List<ConfigurationProperty>()
    {
      new()
      {
          Name = ClimatisationAirConditionerConstants.Configuration.Mode,
          Value = AirConditioningMode.Cooling.ToString(),
      },
    },
  },
};

var symptoms = new List<SymptomDTO> { new(SymptomName, "true") };

var systemConfigurationChangeRequest = new SystemConfigurationChangeRequestDTO()
{
  ServiceConfiguration = changeRequests,
  Symptoms = symptoms,
  Timestamp = DateTime.UtcNow,
};

await _systemApi.SystemRequestChangePostAsync(
  systemConfigurationChangeRequest,
  CancellationToken.None);
\end{lstlisting}


\begin{lstlisting}[language={[Sharp]C},caption={Implementación de la misma petición siguiendo el patrón \emph{builder}.},captionpos=b, label=ls:api-cliente-request-builder]
await _systemService.RequestChangeAsync(changeRequest =>
{
  changeRequest
    .ForSymptom(TemperatureGreaterThanHotThreshold)
    .WithService(ClimatisationAirConditionerConstants.AppName, service =>
    {
      service.MustBePresent()
        .WithParameter(
          ClimatisationAirConditionerConstants.Configuration.Mode,
          AirConditioningMode.Cooling.ToString());
    });
});
\end{lstlisting}

Finalmente, la arquitectura del conector que emplearemos para implementar las peticiones aparece en la figura \ref{fig:monitor-knowledge-connector-architecture}. La figura muestra como el servicio de monitorización contacta al de conocimiento para asignarle un valor a la propiedad \emph{Temp}.

El conector, delimitado por una línea discontinua roja, está compuesto por dos elementos: una API REST y un cliente. Los otros dos grupos de elementos representan los procesos de los servicios de monitorización y conocimiento. El servicio de monitorización se comunica a con la API través del API Client, que está en su proceso actuando como \emph{proxy}.

%%% TODO: Actualizar la imagen para que aparezca PUT en vez de POST.
%%% TODO: ¿Darle la vuelta a los servicios? Monitor arriba y Knowledge abajo.
\begin{figure}[h!]
  \centering
  \includegraphics[scale=0.64]{cap_diseño/images/Monitor-Knowledge-Connector}
  \caption{Diseño del conector usando implementación Cliente - Servidor}
  \label{fig:monitor-knowledge-connector-architecture}
\end{figure}

\pagebreak

\subsection{Notificaciones}

El siguiente mecanismo de comunicación a definir son las notificaciones. Recordemos que esta comunicación es desde un servicio a todos los que se encuentren en la capa superior (\emph{multicast}). No debe estar acoplada a ningún servicio concreto. Potencialmente, todos deberían recibir el mensaje y decidir si procesarlo o no.

Como ejemplo, tomaremos la comunicación entre el servicio de conocimiento y los servicios en la capa superior (el nivel del bucle). Cada vez que se modifique una propiedad o una configuración de un servicio, el servicio emitirá un evento notificando del cambio a la capa superior. Así, por ejemplo, el servicio de análisis sabrá que debe reevaluar las reglas de adaptación.

Sabiendo esto, podemos descartar de entrada GraphQL. Es un protocolo basado en lecturas. Como el objetivo es enviar información a otros servicios, no nos sirve. Respecto a RPC y REST, tampoco nos sirven, no tienen un buen soporte de multicast. Además de que tenemos el requisito de bajo acoplamiento.

Por tanto, optamos por implementarlo usando un \emph{broker} de mensajería. Concretamente, siguiendo el patrón \emph{publish}-\emph{subscribe}. El servicio de conocimiento publicará el evento a través del \emph{broker} de mensajería. Este evento tendrá un \emph{topic} asociado. Todos los servicios interesados deberán suscribirse a este \emph{topic}. El \emph{broker} reenviará el mensaje a una cola específica para cada uno de los servicios suscritos al tema. Así podrán procesarlo cuando puedan, de forma asíncrona.

De esta manera logramos el desacoplamiento de los componentes y permitimos el procesamiento asíncrono de estos eventos.

Aunque potencialmente otros servicios podrían suscribirse a estos cambios, vamos a centrarnos en la suscripción del módulo de análisis.

Los eventos incluirán la información mínima indispensable. En este caso, el nombre de la propiedad que ha cambiado. Esta decisión la tomamos así debido a que es una comunicación asíncrona. Si el evento incluyera el valor de la propiedad y se procesa mucho más tarde, podría derivar en adaptaciones Así evitamos una adaptación incorrecta del sistema. Para evitarlo, incluyendo solo el nombre de la propiedad, obligamos a las reglas a que soliciten el valor de la propiedad en el momento en que se evaluen. Así siempre se ejecutaran con la información actualizada.

El evento que publicaría el módulo de conocimiento cuando cambia una propiedad podría ser como el siguiente (tabla \ref{tab:especificacion-property-changed-integrationevent}):

\newsavebox\propertychangedeventbox
\begin{lrbox}{\propertychangedeventbox}
  \begin{minipage}[t]{2in}
    \begin{verbatim}
{
  "PropertyName":"Temperature"
}
        \end{verbatim}
  \end{minipage}
\end{lrbox}

\begin{table}[htb]
  \centering

  \begin{tabular}{|m{2.3cm}|p{2.5cm}|p{2.6cm}|b{1.5cm}|b{1.5cm}|}
      \hline

      \textbf{Evento} & \multicolumn{2}{|b{0.35\linewidth}|}{\emph{PropertyChangedIntegrationEvent }} & \textbf{\emph{Exchange}} & \emph{AdaptionLoop.Knowledge}  \\
      \hline

      \textbf{Descripción} & \multicolumn{4}{|b{0.6\linewidth}|}{Evento de integración que notifica sobre el cambio de una propiedad adaptación.} \\
      \hline

      \textbf{Propiedades}
            & \emph{propertyName} & \multicolumn{3}{|b{0.6\linewidth}|}{Nombre de la propiedad que ha cambiado.} \\
      \hline

      \textbf{Ejemplo} & \multicolumn{4}{|b{0.7\linewidth}|}{Evento que notifica del cambio de la propiedad \emph{Temperature}:\linebreak
      \usebox\propertychangedeventbox} \\

      \hline
  \end{tabular}

  \caption{Especificación del evento que notifica sobre el cambio de una propiedad del conocimiento.}
  \label{tab:especificacion-property-changed-integrationevent}
\end{table}


Para definir esta comunicación investigamos si había algún estándar equivalente a OpenAPI. Y así es, se llama AsyncAPI\footnote{Página oficial: \url{https://www.asyncapi.com/}}. Es un estándar para especificar la comunicación a través de eventos.Por desgracia, todavía no ha alcanzado el grado de madurez de su homónimo. No tiene el mismo número de herramientas disponible. Por ejemplo, no tiene un catálogo tan amplio de generadores de código que tiene el primero. Tampoco podemos tomar la aproximación de extraer la especificación a partir de una implementación existente.

Aun así, lo utilizaremos para describir nuestros eventos en un formato estándar. En el listing \ref{ls:asyncapi-propertychanged-integrationevent} hemos descrito el mensaje de la tabla \ref{tab:especificacion-property-changed-integrationevent}. Podemos apreciar que es muy parecido a la especificación del método get del listing \ref{ls:openapi-get}.

Aparece la descripción de la estructura del mensaje \texttt{PropertyChangedIntegrationEvent} y su documentación. La mayor diferencia es la mención del canal (el exchange en nuestro caso) y el método (subscribe). Esto indica que los consumidores podrán suscribirse a este evento a partir de este canal.

% TODO: Pintar bien los YAML.
\begin{lstlisting}[language={C++},caption={Ejemplo del evento de integración \emph{builder}.},captionpos=b, label=ls:asyncapi-propertychanged-integrationevent]
asyncapi: 2.4.0
info:
  title: Knowledge Service
  version: 1.0.0
  description: This service contains all the knowledge properties to inform the rest of the loop.
channels:
  AdaptionLoop.Knowledge:
    subscribe:
      message:
        $ref: '#/components/messages/PropertyChangedIntegrationEvent'
components:
  messages:
    PropertyChangedIntegrationEvent:
      description: >-
        Integration event about a change in an adaption property.
      payload:
        type: object
        properties:
          propertyName:
            type: string
            description: The name of the property that changed
\end{lstlisting}

Para implementar este patrón, nuestro conector estará compuesto por tres elementos: un publicador, el \foreign{english}{broker} y un consumidor. El funcionamiento será el siguiente: el servicio de conocimiento recibe una petición para actualizar una propiedad de adaptación. Si esta actualización se lleva a cabo, deberá crear el evento e invocar al publicador, que es un componente que se despliega con este servicio. El publicador recibirá el evento y lo publicará al \foreign{english}{broker} de mensajería en el \foreign{english}{exchange}.

El topic que usaremos es el nombre de la propiedad que ha cambiado.

El broker, que conoce todos los servicios que están subscritos a tema, lo añadirá en la cola de mensajería de cada uno de ellos. Los consumidores, desplegados en cada servicio cliente que se subscribe a estos mensajes, serán notificados de esto. Los procesarán en cuanto puedan. En la \textcolor{red}{figura X} mostramos cómo sería este nuevo conector.

\subsection{Peticiones asíncronas}

El último mecanismo de comunicación a diseñar son las \textbf{peticiones asíncronas}. Se trata de aquellas peticiones de trabajo que un microservicio del mismo nivel de la jerarquía le lanza a otro, sin esperar la respuesta. Como comentamos, tenemos dos casos en nuestra arquitectura que requieren de este patrón: la comunicación entre el módulo de análisis y el planificador, y entre el planificador y el ejecutor. Nos centraremos en el primero.

Recordemos que una vez se evalúan las reglas de adaptación, si alguna de ellas se ejecuta, propone un cambio en la configuración del sistema. El módulo de análisis recibirá esta propuesta y se la pasará al planificador. Este generará el plan de adaptación. Esta comunicación se realiza entre servicios en el mismo nivel de la jerarquía.

A la hora de escoger el mecanismo de comunicación, el razonamiento fue muy similar al de las notificaciones. Optamos por implementarlas usando un broker de mensajería, siguiendo el patrón de colas de trabajo. La arquitectura del comunicador es muy similar a la de las notificaciones: un publicador, un broker y un consumidor.

En lugar de ser una comunicación \foreign{english}{fanout} a través de un exchange, como en las notifcaciones, los mensajes van dirigidos a una cola concreta. El servicio consumidor cuenta con una cola de mensajería específica para las peticiones de trabajo. El publicador la conoce y, a través del broker, envía los mensajes allí. EL consumidor los irá recuperando y procesando en cuando esté disponile.

En la tabla \ref{tab:especificacion-system-configuration-change-request-box} presentamos la especificación de la petición asíncrona para solicitar un cambio de configuración de sistema. Vemos que es muy similar a \ref{tab:especificacion-property-changed-integrationevent}.

\newsavebox\systemconfigurationchangerequestbox
\begin{lrbox}{\systemconfigurationchangerequestbox}
  \begin{minipage}[t]{2in}
    \begin{verbatim}
{
  "Timestamp": "2022-06-19T16:38:30.6092751Z",
  "Symptoms":[
    {
      "Name": "temperature-lesser-than-cold-threshold",
      "Value": "true"
    }
  ],
  "ConfigurationRequests":  [
    {
      "ServiceName": "Climatisation.AirConditioner.Service",
      "IsDeployed": true,
      "ConfigurationProperties": [
        {
          "Name": "Mode",
          "Value": "Heating"
        }
      ],
      "Bindings": []
    }
  ]
}
        \end{verbatim}
  \end{minipage}
\end{lrbox}

\begin{longtable}{|m{2cm}|m{2.3cm}|m{10cm}|b{0.85cm}|b{2.75cm}|}
  \hline

  \textbf{Nombre} & \multicolumn{2}{|b{0.37\linewidth}|}{\emph{SystemConfigurationChangeRequest}} & \textbf{Cola} & \emph{AdaptionLoop.Planification.Requests}  \\
  \hline

  \textbf{Descripción} & \multicolumn{4}{|b{0.82\linewidth}|}{Petición que representa una propuesta de cambio de la configuración del sistema.} \\
  \hline

  \textbf{Propiedades}
    & \emph{Timestamp} & \multicolumn{3}{|m{0.67\linewidth}|}{Fecha y hora de la petición de cambio.} \\
    \cline{2-5}
    & \emph{Symptoms} & \multicolumn{3}{|m{0.67\linewidth}|}{Colección de síntomas que han desencadenado la petición de cambio.} \\
    \cline{2-5}
    & \emph{Configuration Requests} & \multicolumn{3}{|m{0.67\linewidth}|}{Colección peticiones de configuración de la propuesta de cambio.

    A su vez, está compuesto por:
    \begin{itemize}
      \item \textbf{\emph{ServiceName}}: Identificador del servicio cuya configuración queremos cambiar.
      \item \textbf{\emph{IsDeployed}}: Indica si el servicio debe estar desplegado o no en la siguiente configuración.
      \item \textbf{\emph{Bindings}}: Colección de conexiones que indican a qué servicios debe estar conectado (o no) este servicio en la siguiente configuración.
      \item \textbf{\emph{ConfigurationProperties}}: Colección de pares clave-valor que representan valores de configuración que queremos actualizar.
    \end{itemize}} \\
  \hline

  \textbf{Ejemplo} & \multicolumn{4}{|b{0.82\linewidth}|}{Solicitud de cambio de configuración para cambiar el modo de un aire acondicionado a modo calefacción (\emph{heating}). Fue desencadenada porque la temperatura es menor que un umbral determinado:\linebreak
  \usebox\systemconfigurationchangerequestbox} \\

  \hline

  \caption{Especificación de las peticiones de cambio de configuración del sistema.}
  \label{tab:especificacion-system-configuration-change-request-box}
\end{longtable}

Respecto a la especificación con AsyncAPI, las peticiones asíncronas no están soportadas todavía. El grupo está todavía estudiando cómo implementarlas. \footnotetext{Discusión disponible en: \url{https://github.com/asyncapi/spec/pull/594}}. Está propuesta para incluirla en la versión 3.0.0 de la especificación. Como mencionamos anteriormente, el estándar todavía es muy joven y tiene trabajo por delante.

En cuanto a sus componentes, el conector seguiría la siguiente arquitectura: \textcolor{red}{dibujo de arquitectura}.

\section{Diseño final}

Añadir diagrama con el diseño final, mostrando el diseño de los componentes con todos los conectores.



\chapter{Implementación}
\label{chap:implementación}

Una vez descrito el diseño del sistema, llegamos a la etapa de implementación. Uno de los objetivos del trabajo era verificar que la arquitectura elegida era viable. Para ello, optamos por implementar un sistema autoadaptativo muy básico. Tanto el diseño como la implementación se desarrollaron de forma incremental. Gracias a esto, el diseño fue evolucionando según detectábamos nuevas necesidades o problemas que no resolvía nuestra arquitectura.

Por otro lado, el bucle MAPE-K original está desarrollado en Java. Pero, el prototipo que se desarrolló en este trabajo, fue implementado lenguaje C\# usando el \foreign{english}{framework} ASP.NET Core. Se tomó esta decisión para reducir el tiempo de aprendizaje y centrar los esfuerzos en la definición de la arquitectura del sistema.

Gracias a la generación de código, se podría obtener la especificación del prototipo y generar los clientes o servidores en cualquier lenguaje soportado. Java incluido. El bucle MAPE-K original después podría ser refactorizado usando este código autogenerado.

\textcolor{red}{Implementación: Al ser FAdA un sistema tan grande y en continuo desarrollo, vamos a optar por implementar una versión reducida de su funcionalidad en lugar de aplicarlo directamente sobre la plataforma. Nos basaremos en las interfaces de la funcionalidad clave del bucle y dxrlos manejadores adaptive ready. Así podemos verificar sobre un prototipo si los patrones de comunicación elegidos son los adecuados. Esto nos permite reducir el alcance del proyecto.}

En este capitulo describiremos la implementación de los microservicios del nivel de conocimiento de conocimiento y del bucle. Dejaremos para más adelante, en el capitulo \ref{chap:caso_estudio}, la descripción de la implementación del nivel de solución y sistema manejado. En este capitulo ofrecemos una vista más concreta de la implementación, y las tecnologías empleadas. En el otro describiremos cómo encaja todo a nivel general y veremos cómo opera.

La implementación se llevo a cabo en 4 hitos distintos, cada uno correspondiente a una etapa distinta del bucle:

\begin{itemize}
  \item \textbf{Hito 1 - Servicio de monitorización y conocimiento}
  \item \textbf{Hito 2 - Servicio de análisis y reglas}
  \item \textbf{Hito 3 - Planificador}
  \item \textbf{Hito 4 - Ejecutor y efectores}
\end{itemize}

\section{Servicio de monitorización y conocimiento}

En esta primera etapa desarrollamos el proceso de monitorización. Este abarca desde que la sonda realiza sus mediciones hasta que se graban en el conocimiento. Esto implicó implementar varios componentes: las sondas y monitores del caso de estudio (capitulo \ref{chap:caso_estudio}), el componente de monitorización del bucle MAPE-K y la base de conocimiento.

%% TODO: dotnet logo?
Para su desarrollo se optó por el lenguaje C\# y el \foreign{english}{framework} ASP.NET\footnote{Página oficial: \url{https://docs.microsoft.com/en-us/aspnet/core/introduction-to-aspnet-core}}. Este \emph{framework} es específico para implementar servidores web de la plataforma .NET de Microsoft. Lo elegimos porque ya contábamos con experiencia de desarrollo en esta plataforma. Además de que soporta los principales sistemas operativos (Windows, Linux y Mac).

\subsection{Peticiones síncronas}

En este hito también se prototiparon los conectores para peticiones síncronas. Aquellos servicios que las soporten expondrán \foreign{english}{endpoints} HTTP. Por ejemplo, el servicio de conocimiento expone aquellos que permiten recuperar o modificar propiedades de adaptación.  En el fragmento \ref{ls:csharp-get} ya mostramos un ejemplo de su implementación.

Gracias al uso de la librería \emph{Swashbuckle.AspNetCore}\footnote{Página oficial: \url{https://github.com/domaindrivendev/Swashbuckle.AspNetCore}}, pudimos generar su especificación en el estándar OpenAPI. Esto nos aportó dos cosas: una interfaz de usuario para interactuar con la API y la posibilidad de generar el API client.

Hablemos primero sobre la interfaz de usuario. La librería añade a nuestro servicio el \foreign{english}{endpoint} \texttt{/swagger}. Accediendo a esta ruta, se nos servirá una interfaz con un listado de todas las operaciones que ofrece la API (figura \ref{fig:swagger-knowledge-ui}). De cada petición nos muestra su documentación (la que añadimos en el código) y sus parámetros. Incluso nos permite ejecutarlas. De esta forma, los usuarios pueden investigar qué ofrece la API y hacer pruebas.

\begin{figure}[htb]
  \centering
  \includegraphics[scale=1.5]{cap_implementacion/images/swagger-knowledge-ui}
  \caption{Interfaz de usuario ofrecida por Swagger para el servicio de conocimiento. Se genera a partir de las especificación OpenAPI.}
  \label{fig:swagger-knowledge-ui}
\end{figure}

Por otro lado, también nos permite generar el API Client. Como comentamos en la sección \ref{chap:OpenAPI} de OpenAPI, tenemos gran variedad de generadores de código a nuestra disposición. Nosotros optamos por la librería \texttt{OpenAPI.Generator}\footnote{Página del proyecto: \url{https://github.com/OpenAPITools/openapi-generator}}. En concreto, el generador de código de \verb|C#|. Usándolo pudimos generar una librería que permite contactar con nuestro servicio, sin necesidad de implementar mucho código. Por ejemplo, el componente de monitorización del bucle contacta a través de un API Client generado.

\subsection{Componentes: Módulos de monitorización y conocimiento}

El módulo del conocimiento es un servicio muy sencillo. Ofrece operaciones de lectura y escritura sobre las propiedades de adaptación y las claves de configuración de servicios manejados. Para el prototipo estas se almacenan en diccionarios en memoria. Cuando el servicio se reinicie, los valores se perderán.

Por encima de este, tenemos el servicio de monitorización. En nuestra implementación, actúa como intermediario entre los monitores de la solución y el conocimiento. También se trata de un servicio sencillo. Ofrece operaciones útiles para los monitores de la solución: lectura de propiedades del conocimiento y otras para que los monitores reporten sus mediciones. De esta forma, los monitores de solución podrán informarse a partir del conocimiento para determinar si una medición es válida o no.

\textcolor{red}{¿Añadir ejemplos de código? ¿Describir los componentes implementados? ¿Debería unificarse con el capítulo de implementación del caso de estudio?}


A continuación explicaremos cómo utilizamos OpenAPI. Para ello, continuaremos con el ejemplo del servicio de conocimiento que hemos descrito a lo largo de esta sección. Nos centraremos en la implementación de la operación para obtener una propiedad del conocimiento, descrita en la tabla \ref{tab:especificacion-get-property}.

En el \textcolor{red}{fragmento} \ref{ls:csharp-get} podemos observar que se trata de un método C\# llamado \emph{GetProperty}. Su implementación es sencilla: busca en un diccionario la propiedad cuyo nombre se le pasa por parámetro. Si la encuentra, devuelve su valor con un código 200 OK. En caso contrario, devuelve un código de error que describe el motivo (formato de la petición incorrecto o no se ha encontrado la propiedad).

Aparte de la implementación, podemos comprobar que el método cuenta con una serie de comentarios (líneas 1-8) y atributos (10-12). Esta documentación describe qué hace el método, sus entradas y posibles respuestas. OpenAPI es capaz de utilizarlos para generar una especificación más completa. Por tanto, resulta muy recomendable incluirlos.

\begin{lstlisting}[language={[Sharp]C},caption={Implementación del método GetProperty decorado para generar la especificación OpenAPI.},captionpos=b, label=ls:csharp-get]
/// <summary>
///    Gets a property given its name.
/// </summary>
/// <param name="propertyName"> The name of the property to find. </param>
/// <returns> An IActionResult with result of the query. </returns>
/// <response code="200"> The property was found. Returns the value of the property. </response>
/// <response code="404"> The property was not found. </response>
/// <response code="400"> There was an error with the provided arguments. </response>
[HttpGet("{propertyName}")]
[ProducesResponseType(typeof(PropertyDTO), StatusCodes.Status200OK)]
[ProducesResponseType(StatusCodes.Status404NotFound)]
[ProducesResponseType(StatusCodes.Status400BadRequest)]
public IActionResult GetProperty([FromRoute]string propertyName)
{
    if (string.IsNullOrEmpty(propertyName))
    {
        return BadRequest();
    }

    bool foundProperty = properties.TryGetValue(propertyName, out PropertyDTO property);

    if (!foundProperty)
    {
        return NotFound();
    }

    return Ok(property);
}
\end{lstlisting}

Haciendo uso de las librerías de OpenAPI, generamos la especificación a partir del servicio de conocimiento. En el \textcolor{red}{fragmento} \ref{ls:openapi-get}, podemos ver cómo se describe la operación en este estándar:

\begin{lstlisting}[language=python,caption={Especificación OpenAPI del método para obtener una propiedad del conocimiento (\lstinline{GetProperty}).},captionpos=b, label=ls:openapi-get]
"paths": {
  "/Property/{propertyName}": {
    "get": {
      "tags": [
        "Property"
      ],
      "summary": "Gets a property given its name.",
      "parameters": [
        {
          "name": "propertyName",
          "in": "path",
          "description": "The name of the property to find.",
          "required": true,
          "schema": {
            "type": "string"
          }
        }
      ],
      "responses": {
        "200": {
          "description": "The property was found. Returns the value of the property.",
          "content": {
            "application/json": {
              "schema": {
                "$ref": "#/components/schemas/PropertyDTO"
              }
            }
          }
        },
        "404": {
          "description": "The property was not found.",
        },
        "400": {
          "description": "There was an error with the provided arguments.",
        }
      }
    }
  }
\end{lstlisting}

Podemos apreciar que en la ruta \emph{/Property/\{propertyName\}} está disponible una operación de tipo \emph{GET}. Esta acepta determinados parámetros y describe unas posibles respuestas. También aparece una referencia a otro esquema (línea 25), que representa la estructura de la respuesta en ese caso concreto. También aparecen los comentarios opcionales que indicamos en el \textcolor{red}{fragmento} \ref{ls:csharp-get}. Encontramos grandes similitudes con la especificación presentada en la tabla \ref{tab:especificacion-get-property}.

%% TODO: ¿Eliminar?
\textcolor{red}{Los convenios de los generadores de código de OpenAPI pueden no ser de nuestro agrado. Por ejemplo, pueden resultar muy verbosos o puede resultar muy pesado trabajar con DTOs directamente. Por suerte, tenemos varias alternativas para solucionarlo: Modificar las plantillas de generación de código. Al ser de código abierto, podríamos modificar las existentes o crear nuestras propias plantillas con nuestros propios convenios.}

\textcolor{red}{Otra opción, más fácil de implementar, es desarrollar código por encima del API Client generado. Es el caso del servicio de Análisis. Como trabajar con DTOs directamente se hacía muy pesado (fragmento \ref{ls:analysis-api-client-original}), optamos por implementar un \foreign{english}{builder} de peticiones. Esto nos permitia configurar la petición de una forma más descriptiva para el usuario (fragmento \ref{ls:analysis-api-cliente-request-builder}):}

\begin{lstlisting}[language={[Sharp]C},caption={Implementación de petición original. Trabajar con DTOs era muy verboso.},captionpos=b, label=ls:analysis-api-client-original]
var changeRequests = new List<ServiceConfigurationDTO>
{
  new()
  {
    ServiceName = ClimatisationAirConditionerConstants.AppName,
    IsDeployed = true,
    ConfigurationProperties = new List<ConfigurationProperty>()
    {
      new()
      {
          Name = ClimatisationAirConditionerConstants.Configuration.Mode,
          Value = AirConditioningMode.Cooling.ToString(),
      },
    },
  },
};

var symptoms = new List<SymptomDTO> { new(SymptomName, "true") };

var systemConfigurationChangeRequest = new SystemConfigurationChangeRequestDTO()
{
  ServiceConfiguration = changeRequests,
  Symptoms = symptoms,
  Timestamp = DateTime.UtcNow,
};

await _systemApi.SystemRequestChangePostAsync(
  systemConfigurationChangeRequest,
  CancellationToken.None);
\end{lstlisting}


\begin{lstlisting}[language={[Sharp]C},caption={Implementación de la misma petición siguiendo el patrón \emph{builder}.},captionpos=b, label=ls:analysis-api-cliente-request-builder]
await _systemService.RequestChangeAsync(changeRequest =>
{
  changeRequest
    .ForSymptom(TemperatureGreaterThanHotThreshold)
    .WithService(ClimatisationAirConditionerConstants.AppName, service =>
    {
      service.MustBePresent()
        .WithParameter(
          ClimatisationAirConditionerConstants.Configuration.Mode,
          AirConditioningMode.Cooling.ToString());
    });
});
\end{lstlisting}


\section{Servicio de análisis y reglas}
\label{sec:implementacion-modulo-reglas}

En el segundo hito, acordamos implementar la evaluación de reglas de adaptación. Esto requería de implementar el servicio de análisis del bucle MAPE-K y los servicios de reglas de la solución. En este hito empezamos a plantearnos el diseño de las comunicaciones ascendentes: las notificaciones. Con ellas, evitaríamos que se los componentes se acoplaran a la capa superior.

\subsection{Notificaciones}

%% TODO: Logo RabbitMQ?
Comenzaremos describiendo el desarrollo de las notificaciones. Como ya se describió en la sección \ref{sec:notificaciones}, este componente se implementó mediante un \foreign{english}{broker} de mensajería. Elegimos \texttt{RabbitMQ}\footnote{Página oficial: \url{https://www.rabbitmq.com/}}, uno ''sencillo'' y ampliamente utilizado. \cite{newmanBuildingMicroservicesDesigning2021} Con él pudimos  implementar los dos patrones de comunicación que necesitamos: las notificaciones y las peticiones asíncronas.

Para implementar nuestro conector, utilizamos una librería llamada \texttt{Rebus}\footnote{Página oficial: \url{https://github.com/rebus-org/Rebus}}. Esta nos permite interactuar con un bus, abstrayéndonos de la tecnología concreta utilizada para la comunicación. Así, podríamos cambiar de tecnología de transporte en cualquier momento por otra que se ajuste más a nuestros requisitos.

Finalmente, para desacoplar la funcionalidad de de la publicación y consumición de mensajes del bus, empleamos \texttt{MediatR}\footnote{Página oficial: \url{https://github.com/jbogard/MediatR}}. Esta librería implementa el patrón mediador. Nos permite propagar mensajes dentro de un proceso. En nuestro caso, los eventos. Ni el emisor ni el receptor requieren tener referencias del otro. Es similar a un \foreign{english}{broker} de mensajería, pero funcionando interproceso.

Ahora nos centraremos en la comunicación entre el módulo de conocimiento y el servicio de análisis. Una vez se confirma la escritura de una propiedad o configuración en el conocimiento, este debe notificar a los servicios en la capa superior. Para ello, comienza propagando internamente un \textbf{evento de integración} (línea 11 del fragmento \ref{ls:knowledge-set-property}) mediante el mediador.


\begin{lstlisting}[language={[Sharp]C},caption={Implementación del método que asigna valor a una propiedad. Muestra un ejemplo de propagación interna de eventos de integración.},captionpos=b, label=ls:knowledge-set-property]
private async Task SetProperty(SetPropertyDTO propertyDto)
{
    var newValue = new()
    {
        Value = propertyDto.Value,
        LastModification = DateTime.UtcNow,
    };

    properties.AddOrUpdate(propertyDto.Name, newValue, (_, _) => newValue);

    await _mediator.Send(
      new PropertyChangedIntegrationEvent(propertyDto.Name));
}

\end{lstlisting}

El mediador determinará que nuestro componente publicador es el destinatario de este evento. Lo hace en base a las interfaces que implementa (linea 2 del fragmento \ref{ls:knowledge-property-changed-publisher}). Este componente publica el evento en el bus de mensajería (línea 15). Rebus, en base a la configuración del servicio, lo enviará a nuestra instancia de \texttt{RabbitMQ} y lo publicará en el \foreign{english}{exchange} correspondiente. Todos los suscriptores ubicados en la capa superior lo recibirán en su cola.

\begin{lstlisting}[language={[Sharp]C},caption={El publicador de eventos captura el evento de integración y lo publica en el bus.},captionpos=b, label=ls:knowledge-property-changed-publisher]
public class PropertyChangedIntegrationEventPublisher
  : IIntegrationEventPublisher<PropertyChangedIntegrationEvent>
{
  private readonly IBus _bus;

  public PropertyChangedIntegrationEventPublisher(IBus bus)
  {
      _bus = bus;
  }

  public async Task<Unit> Handle(
      PropertyChangedIntegrationEvent notification,
      CancellationToken cancellationToken)
  {
      await _bus.Publish(notification);

      return Unit.Value;
  }
}
\end{lstlisting}

Finalmente, en el servicio de análisis, tenemos el componente consumidor (fragmento \ref{ls:analysis-property-changed-consumer}). \texttt{Rebus} obtendrá el evento de la cola de mensajería y se lo pasará a nuestro consumidor. Este lo recibirá y lo propagará internamente en el servicio (línea 20). Todos los manejadores (\foreign{english}{handlers}) del evento lo recibirán y podrán tratarlo.

\begin{lstlisting}[language={[Sharp]C},caption={El consumidor recibe el evento de integración del bus y lo propaga internamente. Todos los manejadores de este evento lo recibirán.},captionpos=b, label=ls:analysis-property-changed-consumer]
public class PropertyChangedIntegrationEventConsumer
  : IIntegrationEventConsumer<PropertyChangedIntegrationEvent>
{
  private readonly IMediator _mediator;

  public PropertyChangedIntegrationEventConsumer(IMediator mediator)
  {
      _mediator = mediator;
  }

  public async Task Handle(PropertyChangedIntegrationEvent message)
  {
      await _mediator.Publish(message);
  }
}
\end{lstlisting}

\subsection{Componentes: Servicio de análisis y módulos de reglas}

En cuanto a la implementación del módulo de análisis, este realmente no cuenta tampoco con mucha lógica. Participa como intermediario entre el conocimiento y los servicios de reglas. Como hemos visto, recibe los eventos de cambios en las propiedades y los propaga a la capa superior. Ofrece operaciones de solo lecturas de propiedades y configuración del conocimiento. Así, limitamos las escrituras por parte de las reglas.

Se podría considerar que este servicio ha quedado anémico \textcolor{red}{referencia}. En trabajos posteriores, este servicio podría ampliarse añadiendo autenticación y autorización. Así, se podría evitar que servicios no autorizados accedan a sus operaciones o soliciten adaptaciones maliciosas.

Respecto a los módulos de reglas, ofrecemos una implementación de referencia. Aunque, como estos se encuentran en nivel de la solución, el desarrollador es libre de elegir si seguirla u optar por otra distinta.

En el fragmento \ref{ls:adaption-rule-base} mostramos la clase base de las reglas de adaptación. Vemos que está suscrita a los eventos de integración de cambio de propiedad de adaptación y cambio en la configuración del sistema (lineas 2-3). Cuando el consumidor capture uno de ellos, lo propagará internamente. Todas las reglas afectadas lo capturarán.

Esta clase base se desarrolló siguiendo el patrón plantilla (o \foreign{english}{template})\footnote{\url{https://refactoring.guru/design-patterns/template-method}}. Define un método que evalúa la condición de la regla (\texttt{EvaluateCondition}) y, si esta se cumple, la ejecuta (\texttt{Execute}). Aquellas reglas que hereden de esta deberán de implementar ambos métodos.

\begin{lstlisting}[language={[Sharp]C},caption={Clase base para implementar reglas de adaptación. Se evalúa la condición, y si esta se cumple, se ejecuta.},captionpos=b, label=ls:adaption-rule-base]
public abstract class AdaptionRuleBase
    : IIntegrationEventHandler<PropertyChangedIntegrationEvent>,
      IIntegrationEventHandler<ConfigurationChangedIntegrationEvent>
{
    // ..

    private async Task Handle()
    {
        try
        {
            if (await EvaluateCondition())
            {
                await Execute();
            }
        }
        catch (Exception e)
        {
            _diagnostics.RuleEvaluationError(_ruleName, e);

            throw;
        }
    }

    protected abstract Task<bool> EvaluateCondition();

    protected abstract Task Execute();

    // ..
}
\end{lstlisting}

En cuanto a las suscripciones, las herederas deberán indicar de qué propiedades o claves de configuración dependen. En base a ellas, deberemos suscribirnos a los temas de las notificaciones que emite el módulo de análisis. Para ello, hemos implementado una serie de atributos que permiten declarar estas dependencias. Con ellos, decoraremos las clases

En el fragmento \ref{ls:adaption-rule-dependencies} mostramos un ejemplo. En la línea 1 tenemos el atributo que describe las dependencias con la propiedad de adaptación \texttt{Temperature}. Por otro lado, en las líneas 2-5 tenemos la declaración de dependencias con dos claves de configuración del servicio \texttt{Climatisation.AirConditioner}: \texttt{TargetTemperature} y \texttt{Mode}.

\begin{lstlisting}[language={[Sharp]C},caption={Las reglas declaran sus dependencias sobre propiedades de adaptación usando atributos. Estos se utilizarán para las suscripciones a los temas de los eventos.},captionpos=b, label=ls:adaption-rule-dependencies]
[RuleKnowledgePropertyDependency(ClimatisationConstants.Property.Temperature)]
[RuleServiceConfigurationDependency(
    ClimatisationAirConditionerConstants.AppName,
    ClimatisationAirConditionerConstants.Configuration.TargetTemperature,
    ClimatisationAirConditionerConstants.Configuration.Mode)]
public class DisableAirConditionerWhenCoolingAndTargetTemperatureAchievedAdaptionRule
  : AdaptionRuleBase
{
    // ...
}
\end{lstlisting}

Para que el servicio se suscriba a las notificaciones emplearemos la \textbf{reflexión}: analizaremos el ensamblado buscando todas las reglas y obtendremos los valores de sus atributos. En base a ellos, nos suscribiremos a los \foreign{english}{topics} en el \foreign{english}{broker} de mensajería.

\begin{lstlisting}[language={[Sharp]C},caption={Para suscribirnos a los \foreign{english}{topics} de las notificaciones obtenemos las dependencias de las reglas mediante reflexión.},captionpos=b, label=ls:rules-registration]
public static IServiceCollection AddAdaptionLoopAnalysisServices(
  this IServiceCollection services,
  IConfiguration configuration,
  Assembly rulesAssembly)
{
    // ...

    services.AddBus(
        configuration,
        rulesAssembly,
        registerSubscriptions: async bus =>
        {
            var subscriptions = GetRulesBusTopicNames(rulesAssembly);

            foreach (var subscription in subscriptions)
            {
                await bus.Advanced.Topics.Subscribe(subscription);
            }
        });

    return services;
}
\end{lstlisting}

\section{Planificador}

En el tercer hito implementamos las peticiones de cambio de las reglas y el servicio de planificación. Aquí surge también la necesidad del tercer patrón de comunicación: las peticiones asíncronas.

\subsection{Peticiones asíncronas}

Continuando con las reglas de adaptación, llega el turno de describir su ejecución. Esta tiene lugar una vez se evalúa la condición y se estima que es necesaria una acción correctiva. En el cuerpo de la regla se describe la acción como una serie de cambios a la configuración del sistema. Para implementar el método, requerimos de un mecanismo para solicitar los cambios.

Las reglas comunicarán la solicitud al módulo de análisis mediante una petición síncrona. En el fragmento \ref{ls:change-request-builder} mostramos un ejemplo. Usando su API Client, implementamos un \foreign{english}{builder}\footnote{Patrón \foreign{english}{builder}: \url{https://refactoring.guru/design-patterns/builder}} para simplificar la creación de solicitudes de cambio. En este ejemplo, indicamos cuál debería ser la siguiente configuración para el servicio \texttt{Climatisation AirConditioner Service} (líneas 7-14). Deberá estar activo (lineas 7-10) y su propiedad \texttt{Mode} con el valor \texttt{Cooling} (lineas 11-13). En esta petición se incluye también el síntoma que desencadena el cambio (línea 6).

\begin{lstlisting}[language={[Sharp]C},caption={Implementación de la misma petición siguiendo el patrón \emph{builder}.},captionpos=b, label=ls:change-request-builder]
protected override async Task Execute()
{
  await _systemService.RequestConfigurationChange(changeRequest =>
  {
    changeRequest
      .ForSymptom(TemperatureGreaterThanHotThreshold)
      .WithService(ClimatisationAirConditionerConstants.AppName, service =>
      {
        service
          .MustBePresent()
          .WithParameter(
            ClimatisationAirConditionerConstants.Configuration.Mode,
            AirConditioningMode.Cooling.ToString());
      });
  });
}
\end{lstlisting}

El servicio de análisis recibirá la petición y la redirigirá al planificador. Lo hará mediante una petición asíncrona. Su implementación es muy similar a la de las notificaciones, explicada en detalle en el apartado anterior. La mayor diferencia radica en la cardinalidad de la comunicación: el mensaje se publicará directamente en la cola de trabajo, en lugar de publicarlo en un exchange. Es decir, como mucho lo procesará un solo servicio.

Esto implica que sólo cambiará la implementación del publicador. En la línea 14 del fragmento \ref{ls:request-publisher} vemos que el mensaje se enruta directamente a la cola \texttt{PlanningServiceQueue}. El resto de la implementación será muy similar. Solo cambiará las interfaces que debamos implementar para cada tipo de componente (\texttt{IRequestConsumer} en lugar de \texttt{IIntegrationEventConsumer}, etc.).

\begin{lstlisting}[language={[Sharp]C},caption={Las peticiones asíncronas se publican a una cola determinada.},captionpos=b, label=ls:request-publisher]
public class SystemConfigurationChangeRequestPublisher
  : IRequestPublisher<SystemConfigurationChangeRequest>
  where TRequest : Request
{
  public SystemConfigurationChangeRequestPublisher(IBus bus)
  {
      _bus = bus;
  }

  public async Task<Unit> Handle(
      SystemConfigurationChangeRequest request,
      CancellationToken cancellationToken)
  {
      await _bus.Advanced.Routing.Send(
          AdaptionLoopPlanningConstants.Queues.PlanningServiceQueue,
          request);

      return Unit.Value;
  }
}
\end{lstlisting}

\subsection{Componentes: Servicio de planificación}

El planificador recibirá esta petición de cambio y deberá elaborar un \textbf{plan de adaptación}. Para ello, verificará que la solicitud es viable. En nuestro prototipo, para reducir el alcance del proyecto, nos limitamos a comprobar que la configuración sea distinta a la actual. Añadirá al plan las \textbf{acciones de adaptación} requeridas para alcanzar el estado deseado. Si el sistema ya estuviera en ese estado, el plan de cambio se queda vacío y no se propaga.

Ya comentamos que los operadores arquitectónicos con los que operaremos serán del ámbito de los microservicios: activar servicio, eliminar servicio, vincular servicios, desvincularlos y cambiar su configuración.

Por ejemplo, en el fragmento \ref{ls:adaption-change-plan} encontramos un plan de adaptación para la regla descrita en la sección anterior. Solo contiene una acción de adaptación: cambiar el valor de la propiedad \texttt{Mode} a \texttt{Cooling}. Como el servicio de aire acondicionado ya estaba en funcionamiento, no se ha incluido una acción para desplegarlo.

\begin{lstlisting}[language=python,caption={Plan de adaptación generado para la regla anterior. Solo contiene una acción de adaptación: cambiar la configuración \texttt{Mode} del servicio \texttt{AirConditioner}.},captionpos=b, label=ls:adaption-change-plan]
{
  "ChangePlan": {
    "Timestamp": "2022-07-09T09:53:01.1868834Z",
    "Actions":
    [
      {
        "Type": "SetParameter",
        "ServiceName": "Climatisation.AirConditioner.Service",
        "PropertyName": "Mode",
        "PropertyValue": "Cooling"
      }
    ]
  },
  "Symptoms":
  [
    {
      "Name": "temperature-lesser-than-cold-threshold",
      "Value": "true"
    }
  ]
}
\end{lstlisting}

\section{Ejecutor y efectores}

En el hito final implementamos el módulo ejecutor y los efectores. Cerramos así el ciclo del bucle de adaptación. El ejecutor recibe el plan de adaptación del planificador. A partir de este, deberá distribuir las acciones de adaptación que contiene entre los servicios de efectores.

Para transmitirle el plan de adaptación del planificador al ejecutor, volvemos a recurrir a las peticiones asíncronas. El plan de adaptación que se le pasa como evento de integración es el que ya mostramos en el fragmento \ref{ls:adaption-change-plan}.

Una vez captura la petición, el ejecutor deberá determinar qué servicios debe cambiar y manipular sus efectores. En nuestra implementación de referencia, simplemente agrupamos las acciones por el nombre del servicio afectado. Publicaremos cada grupo como notificaciones individuales, usando el nombre del servicio como tema.

Los servicios de efectores del recurso manejado las capturarán. Estos servicios procesan las acciones asignadas. Si contienen el efector correspondiente, la ejecutarán. Dependiendo del tipo de acción, el efector hará una acción u otra: desplegar un servicio, o eliminarlo, cambiar la configuración, etc.

En cuanto a la comunicación con el sistema, este caso es un tanto especial. El mecanismo dependerá del sistema manejado; de si tenemos control sobre su implementación. Si no es así, tendremos que adaptarnos a aquellos que ofrezca el recurso (HTTP, mensajería...).

Continuando con nuestro ejemplo del modo del aire acondicionado, el sistema modificaría la configuración del mismo. Una vez se confime el cambio, si se ha llevado a cabo, se ejecutará actualizará su valor en la base de conocimiento. Dependerá de si el sistema es capaz de hacerlo o debe recaer la responsabilidad en el efector.


\chapter{Caso de estudio: Sistema de climatización}
\label{chap:caso_estudio}

Para verificar la arquitectura definida, decidimos implementar un pequeño sistema autoadaptativo. Se trata de un sistema de climatización, que gestiona la temperatura de una habitación. Para ello, dispondremos de un aire acondicionado, que calentará o enfriará la habitación según corresponda.

\section{Análisis}

El primer paso es capturar los requisitos del sistema a implementar. Cómo hemos comentado, queremos desarrollar un sistema de climatización. Este sistema regulará la temperatura de una habitación mediante el uso de un aparato de aire acondicionado.

El aparato de aire acondicionado ofrece tres modos de funcionamiento: un modo para calentar la estancia, otro para enfriarla, y un estado neutral (apagado). Además, lo hemos dotado con un termómetro interno que nos reporta la temperatura periódicamente.

Para poder climatizar la habitación, necesitamos que el usuario defina su temperatura objetivo: la temperatura de confort. Cambios en la temperatura deberán activar o desactivar el aparato para mantenerla.

Además, nos interesa evitar que el aire acondicionado se encienda y se apague constantemente cuando se alcance o sobrepase esta temperatura. Por ello, definimos unas temperaturas umbrales, tanto de frío como de calor, a partir de las cuales se encenderá el aparato.

\section{Diseño}

Del análisis anterior ya podemos deducir la existencia de dos componentes: un aparato de aire acondicionado (el sistema gestionado) y un termómetro (la sonda). Aparte de ellos, deberemos implementar la infraestructura necesaria para comunicarse con nuestro bucle MAPE-K: monitores, módulos de reglas y efectores que nos permitan interactuar con el sistema manejado.

Para describir el diseño usaremos la notación de sistemas autoadaptativos descrita en \cite{fonsEspecificacionSistemasAutoadaptativos2021}.

\subsection{Sondas:}

Para implementar el sistema, requerimos de las siguientes sondas:

\begin{longtable}{|r p{11.5cm}|}
    \hline
    \textbf{Sonda:} & \emph{thermometer}  \\
    \textbf{Descripción:} & Reporta la temperatura actual de la habitación (en ºc). \\
    \textbf{Monitor:} & \emph{Climatisation.Monitor} \\
    \textbf{Datos:} & \emph{temperature} \\
    \hline
    \textbf{Sonda:} & \emph{airconditioner-mode-changed-probe}  \\
    \textbf{Descripción:} & Reporta el modo de funcionamiento del aire acondicionado cuando este cambia. \\
    \textbf{Monitor:} & \emph{Climatisation.Monitor} \\
    \textbf{Datos:} & \emph{airconditioner-mode} \\
    \hline
    \textbf{Sonda:} & \emph{airconditioner-adaption-loop-registration}  \\
    \textbf{Descripción:} & Cuando arranca el servicio de aire acondicionado, registra la configuración inicial del sistema. \\
    \textbf{Monitor:} & \emph{Climatisation.Monitor} \\
    \textbf{Datos:} & \emph{airconditioner.is-deployed}, \emph{airconditioner-mode}, \emph{target-temperature}, \emph{cold-temperature-threshold}, \emph{hot-temperature-threshold} \\
    \hline

    \caption{Sondas del sistema de climatización.}
    \label{tab:climatisation-probes}
\end{longtable}

\subsection{Propiedades de adaptación:}

También podemos deducir cuáles son nuestras propiedades de adaptación:

\begin{longtable}{|r p{11.5cm}|}
    \hline
    \textbf{Propiedad:} & \emph{temperature}  \\
    \textbf{Descripción:} & Representa la temperatura actual de la habitación (en ºC).  \\
    \textbf{Tipo de dato:} & \emph{float} \\
    \hline
    \textbf{Propiedad:} & \emph{target-temperature}  \\
    \textbf{Descripción:} & La temperatura de confort definida por el usuario. El sistema deberá adaptarse para alcanzarla.  \\
    \textbf{Tipo de dato:} & \emph{float} \\
    \hline
    \textbf{Propiedad:} & \emph{cold-temperature-threshold}  \\
    \textbf{Descripción:} & La temperatura umbral de frío (en ºc). Si la temperatura baja por debajo de este umbral, deberá calentarse la habitación. \\
    \textbf{Tipo de dato:} & \emph{float} \\
    \hline
    \textbf{Propiedad:} & \emph{hot-temperature-threshold}  \\
    \textbf{Descripción:} & La temperatura umbral de calor (en ºc). Si la temperatura sube por encima de este umbral, deberá enfriarse la habitación. \\
    \textbf{Tipo de dato:} & \emph{float} \\
    \hline
    \textbf{Propiedad:} & \emph{airconditioner.is-deployed}  \\
    \textbf{Descripción:} & Indica si el servicio de aire acondicionado está desplegado y en funcionamiento.  \\
    \textbf{Tipo de dato:} & \emph{bool} \\
    \hline
    \textbf{Propiedad:} & \emph{airconditioner-mode}  \\
    \textbf{Descripción:} & Representa el modo de operación actual del aire acondicionado: \emph{Off} = 0, \emph{Cooling} = 1, \emph{Heating} = 2  \\
    \textbf{Tipo de dato:} & Enumerado \\
    \hline

  \caption{Propiedades de adaptación del sistema de climatización.}
  \label{tab:climatisation-adaption-properties}
\end{longtable}

\subsection{Monitores:}

Necesitaremos definir varios monitores para capturar los datos de las sondas. En algunos casos, para evitar falsos positivos, y que se lleve a cabo adaptaciones provocadas por errores de medición, deberemos filtrar estos datos.

Por ejemplo, en el monitor de las temperaturas, \emph{climatisation.monitor.temperature}. Como en el ejemplo trabajamos con un aire acondicionado ficticio, le hemos establecido un margen de error grande: Si la nueva medida de temperatura está a 5ºc de diferencia o más, y hay menos de un minuto de diferencia entre ellas; la descartaremos. De esta forma, evitamos que el aire acondicionado se active o desactive por un error de medición.

\begin{longtable}{|p{3.7cm} p{10.7cm}|}
    \hline

    \textbf{Monitor:} & \emph{climatisation.monitor.temperature}  \\
    \textbf{Descripción:} & Recibe los reportes de temperatura de los termómetros. También filtra estos datos para detectar casos donde se sospecha un error de lectura. \\
    \textbf{Afecta a propiedades de adaptación:} & \emph{temperature} \\
    \multirow{3}*{\textbf{Acciones:}}
        & \textbf{SI} |\emph{new-temperature} - \emph{temperature}| <= 5.0 \\
        & \textbf{O} request.DateTime - previousMeasurement.DateTime > 60s \\
        & \textbf{ACTUALIZA-KNOWLEDGE} \emph{temperature} = \emph{new-temperature} \\
    \hline

    \textbf{Monitor:} & \emph{climatisation.monitor.configuration}  \\
    \textbf{Descripción:} & Recibe la configuración del aire acondicionado y la registra en el \emph{knowledge}. \\
    \textbf{Afecta a propiedades de adaptación:} & \emph{airconditioner.is-deployed}, \emph{airconditioner-mode}, \emph{target-temperature}, \emph{cold-temperature-threshold}, \emph{hot-temperature-threshold} \\
    \multirow{2}*{\textbf{Acciones:}}
        & \textbf{SI} \emph{property} != \emph{new-value} \\
        & \textbf{ACTUALIZA-KNOWLEDGE} \emph{property} = \emph{new-value} \\
    \hline

  \caption{Monitores del bucle MAPE-K del sistema de climatización.}
  \label{tab:climatisation-monitors}
\end{longtable}

\subsection{Reglas de adaptación}
\label{sec:caso-estudio-diseño-reglas}

En base a cambios de la temperatura local, deberemos decidir si es necesario llevar a cabo una acción correctiva. Por ejemplo, que si la temperatura es inferior al umbral de temperatura fría, el aparato se enciende en modo calentador. Para ello, deberemos implementar un servicio de reglas (\emph{Climatisation.Rules.Service}). En él, incluiremos una serie de reglas que se disparen cuando cambie una de nuestras propiedades de adaptación. En este caso, la temperatura.

Como comentamos en el capítulo anterior, en nuestro ejemplo de bucle MAPE-K, nos limitamos a implementar las adaptaciones de tipo set-parameter. Por tanto, no tendremos reglas de despliegue o de binding.

En la tabla \ref{tab:adaption-rules-climatisation} definimos las cuatro reglas necesarias:

\begin{longtable}{|r p{12.8cm}|}
    \hline
    \textbf{Regla:} & \emph{EnableAirConditionerHeatingModeWhenColdTemperatureThresholdExceeded}  \\
    \textbf{Descripción:} & Activa el aire acondicionado en modo calefacción cuando la temperatura sea inferior al umbral de frío.  \\
    \textbf{Condición:} & \emph{airconditioner-mode} != \emph{Heating} \textbf{AND} \emph{temperature} <= \emph{cold-temperature-threshold}  \\
    \textbf{Cuerpo:}   &  \\
    & \includegraphics[scale=0.75]{cap_caso-estudio/images/adaption-loop-rule-heat} \\
    \hline

    \textbf{Regla:} & \emph{DisableAirConditionerWhenHeatingModeEnabledAndTargetTemperatureAchieved}  \\
    \textbf{Descripción:} & Apaga el aire acondicionado cuando el modo calefacción está activo y se ha alcanzado la temperatura de confort.  \\
    \textbf{Condición:} & \emph{airconditioner-mode} == \emph{Heating} \textbf{AND} \emph{temperature} >= \emph{target-temperature}  \\
    \textbf{Cuerpo:} &  \\
    & \includegraphics[scale=0.75]{cap_caso-estudio/images/adaption-loop-rule-off} \\
    \hline

    \textbf{Regla:} & \emph{EnableAirConditionerCoolingModeWhenTemperatureThresholdExceeded}  \\
    \textbf{Descripción:} & Activa el aire acondicionado en modo enfriar cuando la temperatura sea superior al umbral de calor.  \\
    \textbf{Condición:} & \emph{airconditioner-mode} != \emph{Cooling} \textbf{AND} \emph{temperature} >= \emph{hot-temperature-threshold}  \\
    \textbf{Cuerpo:} &  \\
    & \includegraphics[scale=0.75]{cap_caso-estudio/images/adaption-loop-rule-cooling} \\
    \hline

    \textbf{Regla:} & \emph{DisableAirConditionerWhenCoolingAndTargetTemperatureAchievedAdaptionRule}  \\
    \textbf{Descripción:} & Apaga el aire acondicionado cuando el modo enfiar está activo y se ha alcanzado la temperatura de confort.  \\
    \textbf{Condición:} & \emph{airconditioner-mode} == \emph{Cooling} \textbf{AND} \emph{temperature} <= \emph{target-temperature}  \\
    \textbf{Cuerpo:} &  \\
    & \includegraphics[scale=0.75]{cap_caso-estudio/images/adaption-loop-rule-off} \\
    \hline

  \caption{Reglas de adaptación del sistema de climatización.}
  \label{tab:adaption-rules-climatisation}
\end{longtable}

\subsection{Efectores:}

Una vez se evaluén estas reglas, solicitamos un cambio en la configuración del sistema. El módulo de planificación comprobará contra el conocimiento y el estado actual del sistema cuáles de los cambios solicitados es necesario aplicar. Si por ejemplo la propiedad ya tiene el valor solicitado, no hará falta ejecutarla.

El modulo de ejecución recibirá la petición y se la redirigirá a los efectores del sistema de climatización. En este caso, requerimos de efectores que cambien el modo del aire acondicionado según corresponda.

\begin{table}[htb]
  \centering

  \begin{tabular}{|r p{11.5cm}|}
    \hline
    \textbf{Efector:} & \emph{airconditioner.heat}  \\
    \textbf{Descripción:} & Activa el modo calentar del aire acondicionado. \\
    \hline
    \textbf{Efector:} & \emph{airconditioner.cool}  \\
    \textbf{Descripción:} & Activa el modo enfriar del aire acondicionado. \\
    \hline
    \textbf{Efector:} & \emph{airconditioner.turn-off}  \\
    \textbf{Descripción:} & Apaga el aire acondicionado. \\
    \hline
  \end{tabular}

  \caption{Efectores del sistema de climatización.}
    \label{tab:climatisation-effectors}
\end{table}

Hecho esto, el sistema se adapta a a la nueva situación, y reportará una nueva temperatura en cuanto corresponda. La temperatura variará dependiendo de si está apagado o no.

\subsection{Configuración del sistema}

Requerimos entonces 4 servicios para implementar la solución: Servicio de aire acondicionado, monitor de climatización, el servicio de reglas y el servicio de efectores. Con ellos, podemos adaptarnos al bucle MAPE-K descrito en el capítulo %TODO: Capítulo.

\section{Implementación}

Para la implementación, hemos utilizado las mismas tecnologías descritas en el capítulo \ref{chap:implementación}: microservicios \texttt{ASP.NET}, comunicación mediante APIs REST y \foreign{english}{brokers} de mensajería \texttt{RabbitMQ}. Generamos los API Clients con OpenAPI y demás.

A continuación describiremos todos los servicios implementados:

\subsection{Servicio de aire acondicionado}

El servicio de aire acondicionado será nuestro sistema manejado. Como no disponemos de un aire acondicionado real, hemos optado por implementar uno ficticio. Este cuenta con tres estados de operación: apagado (\texttt{OFF}), enfriando (\texttt{COOLING}) o calentando (\texttt{HEATING}). Según el modo, aumentará o disminuirá la temperatura que reporta un termómetro interno. Además, cuando está apagado, la temperatura ficticia aumenta o disminuye gradualmente según una configuración. De esta forma, podemos simular los cambios de temperatura más rápido y ver si se aplican correctamente las adaptaciones correspondientes.

Además, este servicio expone tres endpoints HTTP, que nos permiten cambiar el modo del aire acondicionado. Serán estos los que invocará el servicio de efectores para manipular su estado.

Por otro lado, para simplificar la configuración del sistema, optamos por registrar las propiedades del aire acondicionado durante su arranque. Tenemos para ello un servicio en segundo plano (un \foreign{english}{background service} de \texttt{ASP.NET}) que registra la configuración inicial durante el arranque. Así le daremos valor a las propiedades como los umbrales de temperatura o las variables como is-deployed. \textcolor{red}{Es necesario un fragmento de código?}

\subsection{Monitor}

El monitor de la solución expone endpoints para recabar las mediciones de las sondas. En este caso, las mediciones de temperatura del aire acondicionado: \texttt{POST Measurement/Temperature}. Este endpoint recibe la temperatura y ejecuta las validaciones descritas en el apartado anterior: descarta valores con diferencias de 5ºC tomados en menos de un minuto, entre otras. Si el valor es válido, lo manda al servicio de monitorización. A su vez este lo almacenará en el conocimiento como propiedad de adaptación.

El servicio también ofrece endpoints para actualizar la configuración del aire acondicionado en el conocimiento. Así podrá registrar su configuración inicial o los modos de operación cuando cambien. Son un subconjunto de los endpoints que ofrece el conocimiento, expuestos a través del servicio de monitorización del bucle.

\subsection{Reglas}

En cuanto al servicio de reglas, este contiene las cuatro reglas definidas en el apartado \ref{sec:caso-estudio-diseño-reglas}. Estas activan o desactivan el aire acondicionado en base a la temperatura de la estancia. Se han implementado siguiendo la estructura descrita en el apartado \ref{sec:implementacion-modulo-reglas}. Todas ellas heredan de las clase abstracta \texttt{AdaptionRuleBase}. Deben implementar los métodos para evaluar la condición y ejecutar su acción; y deben declarar las propiedades de adaptación de las que dependen.

Para describirlas, nos centraremos en la regla \texttt{Disable Air Conditioner When Cooling And Target Temperature Achieved Adaption Rule}. Esta desactiva el aire acondicionado cuando está en modo enfriamiento y se ha alcanzado la temperatura de enfriamiento de la estancia. A lo largo de la memoria ya habíamos mostrado algunos fragmentos de código de esta.

Las reglas describen las propiedades o configuraciones de las que dependen mediante atributos. Se trata de aquellas que requiere para evaluar su condición. Tomemos por ejemplo el fragmento \ref{ls:adaption-rule-dependencies}. Observamos que declara de tres dependencias: \texttt{airconditioner-mode}, \texttt{temperature} y \texttt{hot-temperature-threshold}.

En cuanto a la implementación, ya habíamos mostrado su método \texttt{Execute} en el fragmento \ref{ls:change-request-builder}. Solo nos queda exponer la implementación de referencia del método \texttt{Evaluate Condition}. En el fragmento \ref{ls:adaption-rule-evaluate-condition} ofrecemos la implementación de la condición \emph{airconditioner-mode} == \emph{Cooling} \textbf{AND} \emph{temperature} <= \emph{target-temperature}. Observamos que en las líneas 3-5, 12-15 y 17-20 se obtienen las propiedades de adaptación o claves de configuración desde el servicio de análisis. En base a ellas, en las líneas 22-23 se evalúa la condición descrita.

\begin{lstlisting}[language={[Sharp]C},caption={Implementación de referencia del método \texttt{EvaluateCondition}. La regla obtiene del conocimiento el estado actual del sistema y determina si debe ejecutarse.},captionpos=b, label=ls:adaption-rule-evaluate-condition]
protected override async Task<bool> EvaluateCondition()
{
  var currentTemperature =
    await _propertyService.GetProperty<TemperatureMeasurementDTO>(
      ClimatisationConstants.Property.Temperature);

  if (currentTemperature is null)
  {
      return false;
  }

  var airConditionerMode = await _configurationService
    .GetConfigurationKey<AirConditioningMode?>(
        ClimatisationAirConditionerConstants.AppName,
        ClimatisationAirConditionerConstants.Configuration.Mode);

  var targetTemperature = await _configurationService
      .GetConfigurationKey<float?>(
        ClimatisationAirConditionerConstants.AppName,
        ClimatisationConstants.Configuration.TargetTemperature);

  return airConditionerMode == AirConditioningMode.Cooling
      && currentTemperature.Value <= targetTemperature;
}
\end{lstlisting}

\subsection{Efectores}

Finalmente, tenemos el servicio de efectores. En la última etapa del bucle de adaptación, el módulo de ejecución emite una notificación con las acciones de adaptación que debe ejecutarse para determinado componente del sistema manejado. El servicio de efectores se suscribirá a estas notificaciones de aquellos componentes que gestiona. En este caso, las referentes al aire acondicionado.

Una vez capturada, determinará el tipo de acción que debe ejecutar e invocará a su implementación. En nuestro caso, hacemos un dispatch interno del evento

Para el caso de estudio, por restricciones de tiempo, nos limitamos a implementar las adaptaciones que implicaban cambios en la configuración del sistema manejado (adaptaciones \foreign{english}{set parameter}). La implementación del componente manejado determinará cómo se ejecutarán estas acciones. Dependemos de los efectores que expongan. Por ejemplo, para el servicio de aire acondicionado, cambiar el modo de operación supone invocar a unos endpoints que expone. Por ejemplo: (\texttt{POST /airconditioner/turn-off}).

\begin{lstlisting}[language={[Sharp]C},caption={Implementación de los efectores del aire acondicionado. Invocan a los endpoints HTTP en base a las acciones de adaptación.},captionpos=b, label=ls:effector-airconditioner-set-parameter]
public async Task<Unit> Handle(
  SetAirConditionerModeRequest notification,
  CancellationToken cancellationToken)
{
    var succeeded = Enum.TryParse(
      notification.Value,
      out AirConditioningMode mode);

    if (!succeeded)
    {
        return Unit.Value;
    }

    switch (mode)
    {
        case AirConditioningMode.Off:
            await _airConditionerApi.AirConditionerTurnOffPostAsync(cancellationToken);
            break;

        case AirConditioningMode.Cooling:
            await _airConditionerApi.AirConditionerCoolPostAsync(cancellationToken);
            break;

        case AirConditioningMode.Heating:
            await _airConditionerApi.AirConditionerHeatPostAsync(cancellationToken);
            break;
    }

    return Unit.Value;
}
\end{lstlisting}

\section{Despliegue y Pruebas}

Debido a la gran cantidad de microservicios que componen la solución, optamos por empaquetarlos en contenedores \texttt{Docker}. Para ello, definimos un plan de despliegue con Docker Compose, que nos permitía definir el número de instancias y las dependencias entre ellas. Por ejemplo, todos los servicios que requieran de un bus de mensajería, dependen de la instancia de RabbitMQ desplegada. Por simplicidad, optamos por desplegar una única instancia para toda la solución. Lo ideal sería que cada que requiriera de emitir sus propias notificaciones dispongan de la suya propia, que se adapte más a sus necesidades.

\subsection{Telemetría}

Un punto en el que queremos hacer hincapié es en la telemetría. Debido a que estamos tratando con un sistema distribuido es complicado conocer el estado del sistema en determinado momento. Especialmente en este caso, que participan más de diez servicios distintos.

Por defecto, solo contábamos con los \emph{logs} de consola, que mostramos en la figura \ref{fig:console-logs}. Aparecen en una única ventana intercalados los registros de todos los servicios. Aunque nos pueden resultar útil, es una aproximación ineficiente y según aumente la escala de peticiones simultáneamente se volverá más difícil de interpretar.

\begin{figure}[h]
  \centering
  \includegraphics[scale=1.45]{cap_caso-estudio/images/console-logs}
  \caption{Extracto de \emph{logs} de una ejecución habitual.}
  \label{fig:console-logs}
\end{figure}

Por ello, para que resultara más sencillo trabajar en la implementación de los servicios y diagnosticar qué ocurre con el sistema, decidimos implementar una solución de observabilidad. La observabilidad es \cite{parkerProblemDistributedTracing2020} y consta de tres partes distintas: %TODO CITA
\begin{itemize}
  \item \textbf{Logs}: \textcolor{red}{A recording of an Event. Typically the record includes a timestamp indicating when the Event happened as well as other data that describes what happened, where it happened, etc. \cite{opentelemetryOpenTelemetryDocumentation2022} Provide extremely fine-grained detail on a given service, but have no built-in way to provide that detail in the context of a request. \cite{parkerProblemDistributedTracing2020}}
  \item \textbf{Métricas}: \textcolor{red}{Son agregados que nos permiten conocer el estado de las estancias de nuestros servicios. Records a data point, either raw measurements or predefined aggregation, as timeseries with Metadata. \cite{opentelemetryOpenTelemetryDocumentation2022}}
  \item \textbf{Trazas distribuidas}: \textcolor{red}{Tracks the progression of a single Request, called a Trace, as it is handled by Services that make up an Application. A Distributed Trace transverses process, network and security boundaries. \cite{opentelemetryOpenTelemetryDocumentation2022}  providing visibility into the operation of your microservice architecture. It allows you to gain critical insights into the performance and status of individual services as part of a chain of requests in a way that would be difficult or time-consuming to do otherwise. Distributed tracing gives you the ability to understand exactly what a particular, individual service is doing as part of the whole, enabling you to ask and answer questions about the performance of your services and your distributed system. \cite{parkerProblemDistributedTracing2020}}
\end{itemize}

Para poder capturar todos estos elementos, optamos por usar el estándar OpenTelemetry. Se trata de una librería estándar utilizada para instrumentar el código de las aplicaciones. Distintas compañías del ámbito de la telemetría software ofrecen APIs que capturan el output de esta librería.

Gracias a él pudimos capturar la telemetría de la siguiente forma implementar usando tres servicios distintos:

\begin{figure}[h]
  \centering
  \includegraphics[scale=0.75]{cap_caso-estudio/images/observability-telemetry-collection}
  \caption{Extracto de \emph{logs} de una ejecución habitual.}
  \label{fig:observability-telemetry-collection}
\end{figure}

\subsubsection{Loki: Logs}
Lo primero que queremos ver es cómo mejorar nuestra estrategia de logging. Lo ideal es añadir identificadores de correlación (el traceID), que nos permita rastrear a través de los distintos servicios una misma traza. Por ejemplo, podemos filtrar a partir de ella para ver todos los detalles de los servicios que intervinieron.

\subsubsection{Jaeger: Trazas distribuidas}

Gracias a las trazas distribuidas, podemos ver todas las actividades por las que pasó una petición. En nuestro caso, podemos ver por todos los estados por los que paso.

\subsubsection{Prometheus: Métricas}


\subsubsection{Grafana: Visualización}

Desarrollamos un panel de monitorización con Grafana. Esto nos permitía consultar en un solo lugar las métricas, los logs y las trazas.



\chapter{Despliegue y pruebas}
\label{chap:despliegue}

\textcolor{red}{En este capítulo describiremos cómo se llevó a acabo el despliegue de la solución y las pruebas que se realizaron.}

\section{Despliegue}

Debido a la gran cantidad de microservicios que componen la solución, desde muy temprano fue necesario definir un plan de despliegue. Su número iba en aumento y era muy complicado gestionarlos a mano. Para ello, optamos entonces por empaquetarlos en contenedores de \texttt{Docker}\footnote{Página oficial: \url{https://www.docker.com/}}. Gracias a esto podíamos iniciarlos y pararlos fácilmente. Además que nos aporta una serie de ventajas interesantes: ejecución aislada de los procesos, gestión más fácil de las dependencias, entre otras. \cite{newmanBuildingMicroservicesDesigning2021,delatorreNETMicroservicesArchitecture2021}

\begin{wrapfigure}{r}{0.23\linewidth}
  \vspace{-25pt}
  \centering
  \includegraphics[scale=0.95]{cap_despliegue/images/docker-compose-logo}
  \vspace{-15pt}
\end{wrapfigure}

Para orquestar el despliegue de la solución optamos por \texttt{Docker Compose}\footnote{Página oficial: \url{https://docs.docker.com/compose/}}. Nos permite declarar la configuración del despliegue de nuestros servicios. Esto incluye los parámetros de ejecución, número de instancias, políticas de reinicio, etc. Aunque el bucle MAPE-K \foreign{english}{Lite} original corre sobre Kubernetes, no necesitábamos un orquestador tan ''pesado''. Nuestro plan era ejecutar la solución en un único \foreign{english}{host}.

\texttt{Docker Compose} también nos permite declarar las dependencias entre servicios. Esto fue clave para el despliegue del contenedor de \texttt{RabbitMQ}. Debido a que el protocolo requiere de una conexión permanente al bus\cite{johanssonPartRabbitMQBest2019}, todos los servicios que dependen de él deben desplegarse después. Por ello, declaramos una dependencia con este servicio y definimos una política de reintentos.

\subsection{Observabilidad y telemetría}

Un punto en el que queremos hacer hincapié es en la telemetría. Debido a que estamos tratando con un sistema distribuido, es complicado conocer su estado global en un momento determinado. Especialmente en este caso, en el que participan más de diez microservicios distintos. Si necesitáramos depurar y diagnosticar el comportamiento del sistema, es muy difícil trazar el impacto de una petición.

Por defecto, solo contábamos con los \emph{logs} (registros), que mostramos en la figura \ref{fig:console-logs}. Aparecen intercalados en una única ventana los de todos los servicios y peticiones concurrentes. Aunque nos pueden resultar utilidad, es una aproximación ineficiente. Incluyen demasiada información y es difícil de procesar para una persona. Además que, según aumente la carga de peticiones, aumentará el número de registros y se volverá más difícil de interpretar.

\begin{figure}[h]
  \centering
  \includegraphics[scale=1.45]{cap_despliegue/images/console-logs}
  \caption{Extracto de \emph{logs} de una ejecución habitual.}
  \label{fig:console-logs}
\end{figure}

En el ámbito de los sistemas distribuidos requerimos de soluciones de monitorización y \foreign{english}{logging} más avanzadas. \cite{newmanBuildingMicroservicesDesigning2021} Nuestros servicios tendrán que recopilar y reportar datos de su funcionamiento, lo que se conoce como \textbf{telemetría}. Esto requerirá de \textbf{instrumentar} nuestros sistemas con distintas herramientas o \textbf{sondas}. Es exactamente lo mismo que hacemos en la etapa de monitorización del bucle MAPE-K.

Para explotar estos datos recurrimos a técnicas de \textbf{observabilidad}. Según \cite{parkerProblemDistributedTracing2020}, la observabilidad <<\emph{no es sólo un método para monitorizar sistemas en producción, si no también para ser capaces de entender su comportamiento usando un número relativamente bajo de señales}>>. Con \textbf{señales} se refiere a las distintas fuentes de información de telemetría de las que dispongamos.

La observabilidad nos ayuda a detectar \textbf{fluctuaciones en el funcionamiento} de nuestro sistema. Estas puedes ser errores, realentizaciones, caídas de servicios, etc. También nos permite \textbf{explicar sus causas} a partir de las señales. De nuestros servicios podemos capturar tres tipos de señales distintos. Todas ellas son complementarias, ya que reflejan el funcionamiento desde distintas perspectivas. Son conocidas como \textbf{los tres pilares de la observabilidad}:

\begin{itemize}
  \item \textbf{\foreign{english}{Logs}}: Se trata de eventos de la aplicación que se registran durante su funcionamiento. Pueden ser simples cadenas de texto o estructuras de datos más complejas. En el segundo caso, se trata de registros enriquecidos con propiedades que les dotan de más contexto. Es el mecanismo de telemetría que ofrece más detalle del funcionamiento de un servicio concreto. También es el más usado.

  \item \textbf{Métricas}: Son datos agregados que nos permiten conocer el estado global de nuestros servicios. \cite{opentelemetryOpenTelemetryDocumentation2022} Se calculan a partir de mediciones de parámetros del servicio en un momento determinado. Por ejemplo del número de peticiones recibidas, su duración, etc. Normalmente se representan cómo series temporales: peticiones por segundo, duración media de las peticiones, etc.

  \item \textbf{Trazas distribuidas}: Se trata del mecanismo más reciente. Es una forma de registrar el recorrido que hace una petición a través de los distintos microservicios que componen nuestro sistema. Nos permite ver cómo participa cada uno de ellos en la operación y qué impacto tiene en el rendimiento. \cite{parkerProblemDistributedTracing2020}

  Para registrar una traza, le asignaremos a la petición un identificador único que se propagará con cada sub-petición. Están compuestas por \foreign{english}{spans}, operaciones que se realizan dentro de la petición. Cada uno puede tener otras sub-operaciones anidadas. \cite{opentelemetryOpenTelemetryDocumentation2022}
\end{itemize}

Para explotar estos datos, necesitaremos entonces poder hacer consultas sobre ellos. Pongamos por ejemplo que hemos detectado que ha aumentado considerablemente la métrica de la duración media de las peticiones. A partir de la fecha y hora de este suceso, deberíamos poder recuperar la información necesaria para responder a la pregunta de qué ha pasado. Ya sean \foreign{english}{logs}, trazas u otras métricas relacionadas.

Con este fin se surgen las \textbf{plataformas de observabilidad}. Se trata de conjuntos de servicios que capturan los datos de telemetría. Los procesan y almacenan para su posterior consulta. También contamos con servicios que nos permiten visualizar y consultar de forma conjunta toda esta información. Por ejemplo, mediante \foreign{english}{dashboards} o paneles.

\subsubsection{Plataforma de observabilidad}

Para poder capturar y explotar estas señales, necesitaremos construir nuestra propia plataforma de observabilidad. Para este trabajo hemos optado por una combinación de cuatro servicios. \texttt{Grafana Loki} para capturar los \foreign{english}{logs}. \texttt{Prometheus} para capturar las métricas. \texttt{Jaeger} para capturar las trazas distribuidas. Y \texttt{Grafana} para visualizar y consultar todos los datos.

\begin{wrapfigure}{r}{0.15\linewidth}
  \vspace{-10pt}
  \hspace{10pt}
  \centering
  \includegraphics[scale=0.5]{cap_despliegue/images/opentelemetry-logo}
\end{wrapfigure}

Para capturar la telemetría, empleamos el estándar \textbf{OpenTelemetry}\footnote{Página oficial: \url{https://opentelemetry.io/}}. Se trata de un proyecto desarrollado por la Cloud Native Computing Foundation (CNCF). Tiene el objetivo de definir un mecanismo estándar para recopilar y transmitir datos de telemetría. Para ello, ofrece un conjunto de librerias que nos permite instrumentar nuestras aplicaciones. Podremos enviar estos datos a cualquier plataforma que ofrezca extensiones compatibles.

Con estas herramientas hemos instrumentado todos nuestros servicios. Nuestra plataforma de observabilidad tiene la siguiente estructura (figura \ref{fig:observability-telemetry-collection}):

\begin{figure}[h!]
  \centering
  \includegraphics[scale=0.28]{cap_despliegue/images/observability-telemetry-collection}
  \caption{Estructura de nuestra plataforma de observabilidad}
  \label{fig:observability-telemetry-collection}
\end{figure}

\subsubsection{Grafana Loki: \foreign{english}{Logs}}

\begin{wrapfigure}{r}{0.10\linewidth}
  \vspace{-7pt}
  \hspace{-10pt}
  \centering
  \includegraphics[scale=0.85]{cap_despliegue/images/Loki}
\end{wrapfigure}

\texttt{Loki}\footnote{Página oficial: \url{https://grafana.com/oss/loki/}} es un agregador de \foreign{english}{logs} estructurados desarrollado por Grafana Labs. Todos los servicios instrumentados se los enviarán y este los almacenará de forma centralizada. Para facilitar las consultas, Loki indexa todos los registros en base a etiquetas (\foreign{english}{labels}), metadatos especificados por el usuario. Por ejemplo, el nivel (información, \foreign{english}{warning}...) o el nombre del servicio que los emite.

Nuestros servicios emiten los \foreign{english}{logs} siguiendo el mismo convenio. Estos deben incluir la fecha del evento y su nivel de severidad. También incluirán propiedades que indiquen el nombre del emisor y el del entorno en el que se encuentra. Además, queremos correlacionar \foreign{english}{logs} de distintos servicios que se originen de una misma petición. Para ello los etiquetaremos con un identificador único: el identificador de la traza (\texttt{traceId}). En la figura \ref{fig:loki-ejemplo-logs} mostramos un ejemplo de toda la información registrada.

\begin{figure}[h]
  \centering
  \includegraphics[scale=0.5]{cap_despliegue/images/Ejemplo-log}
  \caption{Ejemplo de la estructura de un registro.}
  \label{fig:loki-ejemplo-logs}
\end{figure}

\subsubsection{Prometheus: Métricas}

\begin{wrapfigure}{r}{0.10\linewidth}
  \vspace{-12pt}
  \centering
  \includegraphics[scale=0.025]{cap_despliegue/images/prometheus-logo}
\end{wrapfigure}

\texttt{Prometheus}\footnote{Página oficial: \url{https://prometheus.io/}} es una herramienta de monitorización y alertas desarrollada originalmente por SoundCloud. Nos permite capturar mediciones de parámetros de nuestros servicios. Estas serán procesadas y almacenadas como series temporales. Sobre ellas, podremos hacer distintos tipos de análisis, consultas y visualizaciones.

Por defecto, ASP.NET captura distintas métricas que podemos exponer con Prometheus. También nos permite definir las nuestras propias. Estas pueden ser de distintos tipos. Los más habituales son los \textbf{contadores} e \textbf{indicadores}. \cite{parkerProblemDistributedTracing2020} Los primeros aumentan su valor cada vez que ocurre un evento determinado. Por ejemplo, el número de peticiones recibidas. Por otro lado, los indicadores representan un valor en un momento determinado. Por ejemplo, el número de usuarios activos actualmente.

Para importar los datos, el servidor de Prometheus ejecutará periódicamente consultas HTTP sobre un \foreign{english}{endpoint} estándar: \texttt{GET /metrics}. Nuestros servicios instrumentados expondrán sus métricas y mediciones allí. En la figura \ref{fig:prometheus-ejemplo-metricas} tenemos un ejemplo. Muestra las métricas por defecto y algunas definidas por nosotros, como un contador de peticiones de configuraciones.

\begin{figure}[htb]
  \centering
  \includegraphics[scale=1.65]{cap_despliegue/images/Prometheus-Metricas-Ejemplo}
  \caption{Ejemplo de las métricas que expone el \foreign{english}{endpoint} de Prometheus en el servicio de conocimiento.}
  \label{fig:prometheus-ejemplo-metricas}
\end{figure}

\subsubsection{Jaeger: Trazas distribuidas}

\begin{wrapfigure}{r}{0.15\linewidth}
  \vspace{-20pt}
  \includegraphics[scale=0.12]{cap_despliegue/images/jaeger-logo-x}
  \vspace{-20pt}
\end{wrapfigure}

Jaeger\footnote{Página oficial: \url{https://www.jaegertracing.io}} es un sistema para la captura de trazas distribuidas. Fue desarrollado originalmente por Uber Technologies. Todos los servicios instrumentados enviarán allí los fragmentos correspondientes a las actividades en las que participan (los \foreign{english}{spans}). A partir de ellas y el identificador común de la traza, es capaz de reconstruir la traza completa de la petición.

Gracias a las trazas distribuidas, podemos ver todas las actividades que desencadenó una petición concreta. Podemos ver sus nombres, su duración e incluso las sub-actividades en las que derivan. En nuestro caso, mostramos en la figura \ref{fig:jaeger-traza-distribuida} un fragmento de la traza del reporte de una medición de temperatura. Vemos que esta inicia con la actividad de reporte y que acaba desencadenando actividades en seis servicios distintos.

\begin{figure}[htb]
  \centering
  \includegraphics[scale=0.35]{cap_despliegue/images/jaeger-traza-distribuida}
  \caption{Ejemplo de una traza distribuida de Jaeger. Representa las actividades que desencadena el reporte de una medición de tempertura.}
  \label{fig:jaeger-traza-distribuida}
\end{figure}

A partir de las trazas, Jaeger también es capaz de inferir la arquitectura de nuestra aplicación. En la figura \ref{fig:jaeger-arquitectura-inferida} mostramos la arquitectura inferida de nuestro sistema de climatización. Podemos comprobar que la implementación respeta la jerarquía de los microservicios definida en este trabajo. Aunque, este diagrama no muestra el mecanismo de comunicación usado entre cada uno.

\begin{figure}[htb]
  \hspace{1.25cm}
  \includegraphics[scale=0.3]{cap_despliegue/images/Jaeger-arquitectura-climatizacion}
  \caption{Arquitectura inferida por Jaeger de nuestro sistema de climatización a partir de las trazas capaturadas.}
  \label{fig:jaeger-arquitectura-inferida}
\end{figure}

\subsubsection{Grafana: Visualización}

\begin{wrapfigure}{r}{0.15\linewidth}
  \vspace{-20pt}
  \includegraphics[scale=0.10]{cap_despliegue/images/Grafana_logo}
  \vspace{-20pt}
\end{wrapfigure}

La última pieza del puzzle de observabilidad es Grafana. Desarrollado también por Grafana Labs, es una herramienta para la monitorización y visualización de datos. Gracias a su sistema de \foreign{english}{plugins}, es compatible con una gran variedad de fuentes de información: bases de datos, servicios web, servicios de métricas, etc.

Nos permite \textbf{explorar los datos} a través de consultas sobre las fuentes de información. Todas ellas se hacen usando los mecanismos ofrecidos por cada plataforma. Este sería el caso de Prometheus, donde podemos usar su lenguaje de consultas \texttt{PromQL}. Con él, podremos consultar y visualizar métricas. Esto nos permitirá explotar al máximo las capacidades de cada una.

Incluso, podemos ir más allá y \textbf{definir relaciones entre datos de fuentes distintas}. Retrocedamos a la figura \ref{fig:loki-ejemplo-logs}, que representa los \foreign{english}{logs} de Loki. Si nos fijamos, en el campo \texttt{TracerId} aparece un enlace a Jaeger. Al hacer clic en él, se desplegará en un panel lateral la traza de la petición a la que pertenece el registro. Todo esto nos será de gran ayuda a la hora de \textbf{investigar} los motivos de las fluctuaciones en el funcionamiento.

También se pueden aprovechar las consultas de las fuentes de datos para crear \textbf{paneles de monitorización}. Esto nos permitirá agregar en solo lugar a los \foreign{english}{logs}, las métricas y las trazas. A partir de ellos podemos crear todo tipo de visualizaciones útiles. Por ejemplo, del estado de las peticiones concurrentes, del número de errores, etc. En la figura \ref{fig:grafana-panel-monitorizacion} enseñamos nuestro panel de monitorización. Este nos muestra parámetros como la temperatura actual de la habitación (gráfica superior izquierda), un listado de las adaptaciones ejecutadas (panel de \foreign{english}{logs} bajo las temperaturas) o información técnica de los servicios (consumo de RAM, tiempo de CPU...). En la siguiente sección las explicaremos en más detalle.

\begin{landscape}

  \begin{figure}[htb]
    \centering
    \includegraphics[scale=0.37]{cap_despliegue/images/Grafana-panel-monitorizacion}
    \caption{Panel de monitorización para la solución autoadaptativa de climatización.}
    \label{fig:grafana-panel-monitorizacion}
  \end{figure}

\end{landscape}


\section{Pruebas}

Finalmente, llegó el momento de poner a prueba nuestro sistema. Queríamos determinar si la arquitectura diseñada era viable o requería de algún refinamiento. Recordemos que el objetivo es aplicarla en el bucle MAPE-K \foreign{english}{Lite} original mediante una refactorización. Con esto en mente, diseñamos distintas pruebas para verificar su funcionamiento. Nos resultó de gran ayuda nuestra plataforma de observabilidad, que nos permitirá investigar distintas áreas de la solución.

Las primeras pruebas que ejecutamos fueron las relacionadas con el funcionamiento del sistema de climatización. Hasta que este no se estabilizará, no podíamos emitir ningún juicio sobre la arquitectura. En estos tests verificamos que, a partir de las mediciones de temperatura, debe ser capaz de completar el proceso de adaptación. Esto implica que todas las etapas del bucle MAPE-K se ejecutan correctamente.

Completadas estas pruebas pasamos a verificar, ahora sí, aspectos de la arquitectura. Gracias a la telemetría que capturamos, pudimos responder a distintas preguntas sobre ella. Por ejemplo, si es correcta la división funcional que hemos elegido o si los mecanismos de comunicación son los adecuados. En base a las respuestas, pautamos una serie de correcciones que se podrían aplicar.

\subsection{Pruebas sobre el sistema de climatización}

\begin{wrapfigure}{r}{0.38\linewidth}
  \vspace{-15pt}
  \centering
  \includegraphics[scale=0.50]{cap_despliegue/images/pruebas-logs-error}
  \caption{Niveles de los \foreign{english}{logs} registrados durante la inicialización del sistema.}
  \label{fig:pruebas-logs-inicializacion}
  \vspace{-15pt}
\end{wrapfigure}

La primera prueba del sistema de climatización consistió en comprobar su \textbf{correcto despliegue}. Para ello, verificamos que después de este, todos los servicios estén operativos. También analizamos los \foreign{english}{logs} en busca de registros de error. Detectamos que durante la inicialización aparecen algunos (primera barra en la figura \ref{fig:pruebas-logs-inicializacion}). Una vez que se completa, el sistema se estabiliza y estos errores desaparecen (resto de barras). Todos provenían de servicios que dependen del \foreign{english}{broker} de mensajería (figura \ref{fig:prueba-logs-error-rabbitmq}). Como deben establecer una conexión con él durante el arranque, si no ha completado su despliegue todavía, fallarán.

Según el desarrollador de Rebus, estos errores podrían solucionarse implementado un \textbf{servicio en segundo plano que gestione la conexión}.\footnote{\url{https://github.com/rebus-org/Rebus.ServiceProvider\#delayed-start-of-the-bus}} De esta forma, el servicio no fallará durante el arranque e intentará periódicamente conectarse con RabbitMQ. Debido a restricciones de tiempo, en lugar de implementarlo así, optamos por definir una estrategia de reintentos en el fichero de Docker Compose. Así, los afectados se reiniciarán hasta que puedan establecer la conexión correctamente.

\begin{figure}[htb]
  \centering
  \includegraphics[scale=0.45]{cap_despliegue/images/Logs-fallo-RabbitMQ}
  \caption{Ejemplo de registro relacionado con el fallo al contactar con RabbitMQ durante el arranque.}
  \label{fig:prueba-logs-error-rabbitmq}
\end{figure}

A continuación, procedimos a realizar \textbf{pruebas sobre la funcionalidad}. En estas, nos centramos en verificar que el sistema \textbf{regula correctamente la temperatura} de la habitación. Recordemos que el servicio de aire acondicionado cuenta con un termómetro simulado. Cada 15 segundos, este reporta una medición de temperatura ficticia al monitor de climatización. Esta variará dependiendo del modo activo del aire acondicionado. En base a la medición, el bucle evaluará las reglas y pautará adaptaciones si lo considera necesario.

Comprobaremos entonces que se aplican las cuatro reglas definidas en la tabla \ref{tab:adaption-rules-climatisation}. Todas ellas activan o desactivan un modo del aire acondicionado cuando la temperatura alcanza un determinado umbral. Por ejemplo, si es muy alta, se debería activar el modo de refrigeración. Cuando se alcance la temperatura de confort, otra regla lo apagará.

Contamos con varias formas de verificarlo. La primera de ellas es mediante la visualización de la temperatura de nuestro panel de monitorización. En la figura \ref{fig:pruebas-temperatura} mostramos la aplicación de las cuatro reglas. Cuando se alcanza uno de los umbrales (las líneas horizontales), el aire acondicionado se activa o desactiva según corresponda.

\begin{figure}[h]
  \hspace{-1.2cm}
  \includegraphics[scale=0.42]{cap_despliegue/images/pruebas-temperatura-calentar}
  \includegraphics[scale=0.42]{cap_despliegue/images/pruebas-temperatura-enfriar}
  \caption{Gráficas extraídas de Grafana que muestran el funcionamiento de las adaptaciones. Izq.: Encender y apagar la calefacción. Der.: Encender y apagar la refrigeración.}
  \label{fig:pruebas-temperatura}
\end{figure}

Otras visualizaciones que nos pueden ser de utilidad son \foreign{english}{logs} de los ejecutores de la solución. En el panel de monitorización incluimos aquellos que registran las adaptaciones (figura \ref{fig:prueba-logs-adaptaciones}). En base a ellos podemos determinar que efectivamente se están pautando las adaptaciones correspondientes. Se intenta cambiar el parámetro \texttt{Mode} a \texttt{Cooling} o \texttt{Off} según corresponda. Para verificar que realmente se está siguiendo el flujo esperado, podemos consultar la traza asociada al registro. En la figura \ref{fig:prueba-logs-adaptaciones} mostramos su identificador resaltado.

\begin{figure}[htb]
  \hspace{-0.9cm}
  \includegraphics[scale=1.85]{cap_despliegue/images/Pruebas-logs-adaptaciones}
  \caption{\foreign{english}{Logs} de los ejecutores de la solución que confirman las adaptaciones pautadas.}
  \label{fig:prueba-logs-adaptaciones}
\end{figure}

\pagebreak

\subsection{Verificación de la arquitectura}

Una vez confirmado el correcto funcionamiento del sistema, pudimos proceder a verificar distintos aspectos de la arquitectura. Nos centramos especialmente en la comunicación entre servicios. Gracias a ella, podríamos determinar si era apropiada la división funcional en microservicios que definimos. También nos permitió estudiar el comportamiento de los mecanismos de comunicación elegidos.

Recurrimos en primer lugar a la arquitectura inferida por Jaeger. Recuperamos para ello la figura \ref{fig:jaeger-arquitectura-inferida}. Esta describe el resultado de trazar 10 reportes de mediciones de temperatura. De ellas, 8 pasaron el filtro del monitor, y solo 1 ha provocado la adaptación del sistema. Muestra todos los microservicios que intervinieron y el número de mensajes enviados entre cada uno.

Nuestra primera comprobación fue respecto a la jerarquía de componentes y la dirección de la comunicación. Verificamos que la estructura de este diagrama coincide con la nuestro diseño (figura \textcolor{red}{REFERENCIA FIGURA DISEÑO FINAL}). Ningún microservicio aparece conectado con otro no contemplado. También coincide la dirección de la comunicaciones entre ellos.

La siguiente prueba fue sobre el número de mensajes. Si existe un intercambio elevado de mensajes entre dos o más servicios, se puede considerar que son muy ''habladores'' (\foreign{english}{chatty} en inglés). Esto puede ser un indicador de que están muy acoplados. \cite{singjaiPatternsDerivingAPIs2021} De ser así, pueden convertirse en \textbf{puntos de congestión} o \textbf{impedir que el dependiente funcione} correctamente si falla el otro. Detectamos dos casos así en el diagrama y los presentamos en la figura \ref{fig:pruebas-congestion}. En verde, aparecen marcadas las conexiones entre los monitores con el servicio de monitorización. Por otro lado, en rojo marcamos las reglas y el servicio de análisis. Como veremos a continuación, son casos muy similares.

\begin{figure}[htb]
  \hspace{1.25cm}
  \includegraphics[scale=0.3]{cap_despliegue/images/Pruebas-congestion}
  \caption{Puntos de congestión visibles en la arquitectura inferida por Jaeger. Marcados en verde y en rojo.}
  \label{fig:pruebas-congestion}
\end{figure}

El más evidente es el del microservicio de reglas. Realiza 85 peticiones síncronas al módulo de análisis. Este último, a su vez, redirige 84 de ellas al servicio de conocimiento. La faltante podemos asumir que es una petición de cambio de configuración de sistema, resultado de la ejecución de una regla. Presenta entonces muestras muy claras de acoplamiento. En la sección \ref{sec:implementacion-modulo-reglas}, ya comentamos que el módulo de análisis actuaría como intermediario entre todas las comunicaciones con las capas inferiores.

\pagebreak

El siguiente paso fue comprobar el impacto de esta dependencia mediante \textbf{pruebas de carga}. Queríamos ver cómo se comportaba el sistema en casos extremos. Para ello, usamos la librería \texttt{NBomber}\footnote{Página oficial: \url{https://github.com/PragmaticFlow/NBomber}.} e implementamos un test sencillo. Durante un minuto, saturará el sistema enviando mediciones falsas de temperatura al monitor. Esto provocará que el bucle esté constantemente ejecutando adaptaciones. Mediremos su impacto mediante el \textbf{tiempo medio de adaptación}. Con tiempo de adaptación nos referimos al intervalo que transcurre desde que se reporta la medición hasta que se aplica y confirma una adaptación. Lo calcularemos a partir de la duración de las trazas distribuidas.

En la figura \ref{fig:pruebas-carga} presentamos el resultado. A la izquierda aparece nuestro marco de referencia. Fue tomada con la carga habitual del sistema, recibiendo una medición de temperatura cada 15 segundos. Presenta un tiempo medio de 73ms, con picos cercanos a los 150ms. A la derecha se muestra el resultado de una carga extrema: en el espacio de un minuto se enviaron 2312 mediciones. Esto provocó que el tiempo medio aumente hasta los 3.02s, con picos superado los 15s. Tras repetidas ejecuciones confirmamos que los resultados eran muy similares. Confirmamos así nuestras sospechas de la existencia de un punto de congestión.

\begin{figure}[h]
  \hspace{-1.2cm}
  \includegraphics[scale=0.42]{cap_despliegue/images/pruebas-carga-baseline}
  \includegraphics[scale=0.42]{cap_despliegue/images/pruebas-carga-extremo}
  \caption{Comparación del tiempo medio de adaptación según el nivel de carga del sistema. Izq.: Carga habitual Der.: Carga extrema}
  \label{fig:pruebas-carga}
\end{figure}

Para confirmar dónde se encuentra el punto de congestión, acudimos a las trazas. Analizamos varias peticiones del pico de tiempo medio. En la figura \ref{fig:prueba-carga-traza} mostramos una de ellas con una duración de 16.39s. La mayor parte de este tiempo se encuentra en el intervalo (0.042s - 16.04s), en el que no se ejecuta ninguna actividad. La notificación de cambio de la propiedad temperatura está encolada, a la espera de que el servicio de reglas la procese. Esto confirmó que \textbf{se satura y ralentiza el proceso de adaptación}.

\begin{figure}[htb]
  \hspace{-0.2cm}
  \includegraphics[scale=0.45]{cap_despliegue/images/pruebas-carga-traza}
  \caption{Traza distribuida de una adaptación cuando el sistema se encuentra bajo carga extrema.}
  \label{fig:prueba-carga-traza}
\end{figure}



\section{Propuestas de mejora}

Para reducir este riesgo, proponemos dos estrategias complementarias. La más sencilla, es reducir el número de peticiones al conocimiento. Actualmente, para consultar una propiedad de adaptación o configuración, es necesario emitir una petición independiente. Podríamos agruparlas y reducir considerablemente el número de llamadas.


Para solucionarlo



comprobar que dos servicios están muy acoplados. El de reglas y el de análisis.
Por otro lado, vemos también la misma dependencia entre el servicio de análisis y el conocimiento. EN este caso, la división funcional si que es clara. AParte de que el cocnocimiento está compartido entre distintos servicios del nivel del bucle. Por tanto, una mejor solución sería optimizar esta comunicación. Podríamos agrupar en una misma petición la solicitud de varias propiedades del conocimiento a la vez.

Las modificaciones que se podrían efectuar son las siguientes: agrupar en un mismo servcio el  monitor con el monitor de la solución y las reglas con el módulo de análisis. Esto nos facilitará también el funcionamiento plug \& play que buscábamos.


Obviamente, los servicios que dependan de peticiones síncronas no podrán funcionar sin aquel del que dependen. Por ejemplo, el conocimiento. Por ello, deberemos gestionarlos mediante técnicas de replicación para adelantarnos a cualquier fallo sobre él.


%%%%%%%%%%%%%%%%%%%%%%%%%%%%%%%%%%%%%%%%%%%%%%%%%%%%%%%%%%%%%%%%%%%%%%%%%%%%%%%
%                                 CONCLUSIONS                                 %
%%%%%%%%%%%%%%%%%%%%%%%%%%%%%%%%%%%%%%%%%%%%%%%%%%%%%%%%%%%%%%%%%%%%%%%%%%%%%%%

\chapter{Conclusiones}
\label{chap:conclusiones}

En este capítulo se realiza una retrospectiva del desarrollo del trabajo y presentaremos algunas conclusiones sobre el mismo. Por ejemplo, el grado de realización de los objetivos marcados o la descripción de vertientes todavía abiertas.

Al inicio del trabajo se presentaron una serie de objetivos que se quería alcanzar.


Para el desarrollo del trabajo nos planteamos los siguientes objetivos:

\begin{enumerate}
  \item Diseñar una arquitectura para soluciones autoadaptativas preparadas para desplegarse nativamente como microservicios en la nube. Esto implica determinar los componentes en los que dividiremos la funcionalidad del bucle y los mecanismos de comunicación para conectarlos.

  \item Definir directrices para la implementación de los diferentes componentes adaptativos específicos de una solución: monitores, sondas, efectores\dots

  \item Desarrollar un caso práctico para demostrar la viabilidad y aplicabilidad de nuestra propuesta.
\end{enumerate}

\section{Relación con asignaturas cursadas}

El trabajo desarrollado tiene relación con varias asignaturas cursadas durante el máster. Entre ellas, podemos destacar:

\begin{itemize}
  \item \textbf{Diseño de Sistemas Ubicuos y Adaptativos} (SUA): Es la asignatura que más relación guarda con el trabajo. En ella se tratan la computación autónoma y los sistemas autoadaptativos. Mediante el desarrollo de un prototipo de coche autónomo, se presentó el bucle MAPE-K y sus distintas fases.

  \item \textbf{Internet de los Servicios (IoS) y de las Cosas (IoT)} (ISC): En esta asignatura se presentan conceptos relacionados con los servicios web. Se introducen patrones de diseño como las APIs REST y arquitecturas adaptadas a entornos \foreign{english}{cloud}. Además, el campo del internet de las cosas se beneficia también de los sistemas autoadaptativos. \cite{savaglioAgentbasedInternetThings2020}.

  \item \textbf{\foreign{english}{Data Science}} (DAS) y \textbf{Extracción de información desde la red social} (ERS): En ambas asignaturas se trata la extracción de conocimiento a partir de los datos. Mediante técnicas de obtención, procesamiento y visualización, podemos interpretarlos y ''contar una historia'' con ellos. Ambas tuvieron una gran influencia en el desarrollo de la plataforma de observabilidad y las visualizaciones implementadas (capitulo \ref{chap:despliegue}).

\end{itemize}

\section{Trabajos futuros}

En cuanto a trabajos futuros, el más evidente es aplicar la refactorización sobre el bucle MAPE-K \foreign{english}{Lite} de FaDA. Aunque el prototipo pretendía ser lo más fiel al sistema original, es posible que surjan nuevas dificultades no contempladas. Deberá definirse una estrategia para atacarla e ir implementándola gradualmente. Para ello, podrán aprovecharse las interfaces definidas de los servicios y sus especificaciones en lenguajes estándares como OpenAPI. Mediante la generación de código, tanto de clientes como de servidores, se podrá reducir el tiempo y esfuerzo necesarios para la implementación. abo.

Todavía quedan algunas vertientes abiertas que se podrían explorar. Entre ellas, la implementación de \foreign{english}{multitenancy} (multicliente). \cite{aljahdaliMultitenancyCloudComputing2014} Es decir, permitir que varias soluciones autoadaptativas empleen la misma infraestructura del bucle; pero de forma segregada, sin poder interferir entre ellas o acceder a los datos de otras. Se excluyó de este trabajo para reducir el alcance del proyecto. Para implementarla, se debería desarrollar mecanismos de autenticación y autorización, que permitan identificar a cada aplicación y limitar sus permisos. Además deberá estudiarse cómo proteger el acceso a la información de los distintos clientes.

Por otro lado, se podrían investigar más maneras de explotar la telemetría recogida por la plataforma de observabilidad. Por ejemplo, para informar al proceso del bucle MAPE-K. De esta forma, se podrían extraer propiedades de adaptación y ampliar nuestro conocimiento del estado del sistema. En base a estas, se podrían definir nuevas reglas que nos permitan adaptar nuestro sistema a distintas situaciones. Un caso interesante sería analizar las métricas de peticiones concurrentes para desplegar nuevas instancias de los servicios.


%%%%%%%%%%%%%%%%%%%%%%%%%%%%%%%%%%%%%%%%%%%%%%%%%%%%%%%%%%%%%%%%%%%%%%%%%%%%%%%
%                                BIBLIOGRAFIA                                 %
%%%%%%%%%%%%%%%%%%%%%%%%%%%%%%%%%%%%%%%%%%%%%%%%%%%%%%%%%%%%%%%%%%%%%%%%%%%%%%%

\bibliography{bibliography}

\cleardoublepage

%%%%%%%%%%%%%%%%%%%%%%%%%%%%%%%%%%%%%%%%%%%%%%%%%%%%%%%%%%%%%%%%%%%%%%%%%%%%%%%
%                           APÈNDIXS                            %
%%%%%%%%%%%%%%%%%%%%%%%%%%%%%%%%%%%%%%%%%%%%%%%%%%%%%%%%%%%%%%%%%%%%%%%%%%%%%%%

\APPENDIX

\chapter{APIs del Sistema}
\label{anx:apis}

En este anexo incluimos la definición de todas las APIs de los microservicios del sistema. Esto incluye los \foreign{english}{endponts} HTTP, las notificaciones y las peticiones asíncronas. Se listan en orden de intervención en el proceso de adaptación.

\section{Bucle de adaptación}

\subsection{Monitorización}

\subsubsection{Peticiones síncronas}

Su especificación OpenAPI puede encontrarse \href{https://github.com/Starkie/TFM-DistributedAutoadaptiveSystems/blob/1db95346290cb55edbfd5efb717785bcd06def79/src/AutoAdaptativeSystem/AdaptionLoop/Monitoring/Monitoring.Service-OpenAPISpec.json}{aquí}.

\begin{figure}[h!]
  \hspace{-0.25cm}
  \includegraphics[scale=0.45]{anx_apis/images/apis-monitoring}
  \caption{\foreign{english}{Endponts} HTTP que expone el servicio de monitorización.}
\end{figure}

\subsection{Conocimiento}

\subsubsection{Peticiones síncronas}

Su especificación OpenAPI puede encontrarse \href{https://github.com/Starkie/TFM-DistributedAutoadaptiveSystems/blob/1db95346290cb55edbfd5efb717785bcd06def79/src/AutoAdaptativeSystem/AdaptionLoop/Knowledge/Knowledge.Service-OpenAPISpec.json}{aquí}.

\begin{figure}[h!]
  \hspace{-0.25cm}
  \includegraphics[scale=0.45]{anx_apis/images/apis-knowledge}
  \caption{\foreign{english}{Endponts} HTTP que expone el servicio de conocimiento.}
\end{figure}

\pagebreak

\subsubsection{Notificaciones}

\newsavebox\configurationchangedeventbox
\begin{lrbox}{\configurationchangedeventbox}
  \begin{minipage}[t]{2in}
    \begin{verbatim}
{
  "ServiceName":"Climatisation.AirConditioner.Service",
  "ConfigurationName":"TargetTemperature",
}
        \end{verbatim}
  \end{minipage}
\end{lrbox}

\begin{longtable}{|m{2.3cm}|p{3cm}|p{2.6cm}|b{1.5cm}|b{1cm}|}
  \hline

  \textbf{Evento} & \multicolumn{4}{|b{0.7\linewidth}|}{\emph{PropertyChangedIntegrationEvent }} \\
  \hline

  \textbf{\emph{Exchange}} & \multicolumn{4}{|b{0.7\linewidth}|}{\emph{AdaptionLoop.Knowledge}} \\
  \hline

  \textbf{Tema} & \multicolumn{4}{|b{0.7\linewidth}|}{\emph{PropertyChangedIntegrationEvent}} \\
  \hline

  \textbf{Descripción} & \multicolumn{4}{|b{0.6\linewidth}|}{Evento de integración que notifica sobre el cambio de una propiedad adaptación.} \\
  \hline

  \textbf{Propiedades}
        & \emph{propertyName} & \multicolumn{3}{|b{0.6\linewidth}|}{Nombre de la propiedad que ha cambiado.} \\
  \hline

  \textbf{Ejemplo} & \multicolumn{4}{|b{0.7\linewidth}|}{Evento que notifica del cambio de la propiedad \emph{Temperature}:\linebreak
  \usebox\propertychangedeventbox} \\

  \hline
  \hline

  \textbf{Evento} & \multicolumn{4}{|b{0.7\linewidth}|}{\emph{ConfigurationChangedIntegrationEvent}} \\
  \hline

  \textbf{\emph{Exchange}} & \multicolumn{4}{|b{0.7\linewidth}|}{\emph{AdaptionLoop.Knowledge}}  \\
  \hline

  \textbf{Tema} & \multicolumn{4}{|b{0.7\linewidth}|}{\emph{ConfigurationChangedIntegrationEvent}} \\
  \hline

  \textbf{Descripción} & \multicolumn{4}{|b{0.6\linewidth}|}{Evento de integración que notifica sobre el cambio de una clave de configuración.} \\
  \hline

  \textbf{Propiedades}
        & \emph{serviceName} & \multicolumn{3}{|b{0.6\linewidth}|}{Nombre del servicio al que pertenece.} \\

        \cline{2-5}

        & \emph{configurationName} & \multicolumn{3}{|b{0.6\linewidth}|}{Nombre de la clave de configuración que ha cambiado.} \\
  \hline

  \textbf{Ejemplo} & \multicolumn{4}{|b{0.7\linewidth}|}{Evento que notifica del cambio de la propiedad de configuración \emph{TargetTemperature}:\linebreak
  \usebox\configurationchangedeventbox} \\

  \hline

  \caption{Especificación de las notificaciones que publica el servicio de conocimiento.}
\end{longtable}


\subsection{Análisis}

\subsubsection{Peticiones síncronas}

Su especificación OpenAPI puede encontrarse \href{https://github.com/Starkie/TFM-DistributedAutoadaptiveSystems/blob/1db95346290cb55edbfd5efb717785bcd06def79/src/AutoAdaptativeSystem/AdaptionLoop/Analysis/Analysis.Service-OpenAPISpec.json}{aquí}.

\begin{figure}[h!]
  \hspace{-0.25cm}
  \includegraphics[scale=0.45]{anx_apis/images/apis-analysis}
  \caption{\foreign{english}{Endponts} HTTP que expone el servicio de análisis.}
\end{figure}

\subsubsection{Notificaciones}

\begin{longtable}{|m{2.3cm}|p{3cm}|p{2.6cm}|b{1.5cm}|b{1cm}|}
  \hline

  \textbf{Evento} & \multicolumn{4}{|b{0.7\linewidth}|}{\emph{PropertyChangedIntegrationEvent }} \\
  \hline

  \textbf{\emph{Exchange}} & \multicolumn{4}{|b{0.7\linewidth}|}{\emph{AdaptionLoop.Analysis}} \\
  \hline

  \textbf{Tema} & \multicolumn{4}{|b{0.7\linewidth}|}{Nombre de la propiedad. Ej: \emph{Temperature}} \\
  \hline

  \textbf{Descripción} & \multicolumn{4}{|b{0.6\linewidth}|}{Evento de integración que notifica sobre el cambio de una propiedad adaptación.} \\
  \hline

  \textbf{Propiedades}
        & \emph{propertyName} & \multicolumn{3}{|b{0.6\linewidth}|}{Nombre de la propiedad que ha cambiado.} \\
  \hline

  \textbf{Ejemplo} & \multicolumn{4}{|b{0.7\linewidth}|}{Evento que notifica del cambio de la propiedad \emph{Temperature}:\linebreak
  \usebox\propertychangedeventbox} \\

  \hline
  \hline

  \textbf{Evento} & \multicolumn{4}{|b{0.7\linewidth}|}{\emph{ConfigurationChangedIntegrationEvent}} \\
  \hline

  \textbf{\emph{Exchange}} & \multicolumn{4}{|b{0.7\linewidth}|}{\emph{AdaptionLoop.Analysis}}  \\
  \hline

  \textbf{Tema} & \multicolumn{4}{|b{0.7\linewidth}|}{Nombre del servicio y la propiedad. Ej: \emph{Climatisation.AirConditioner.TargetTemperature}} \\
  \hline

  \textbf{Descripción} & \multicolumn{4}{|b{0.6\linewidth}|}{Evento de integración que notifica sobre el cambio de una clave de configuración.} \\
  \hline

  \textbf{Propiedades}
        & \emph{serviceName} & \multicolumn{3}{|b{0.6\linewidth}|}{Nombre del servicio al que pertenece.} \\

        \cline{2-5}

        & \emph{configurationName} & \multicolumn{3}{|b{0.6\linewidth}|}{Nombre de la clave de configuración que ha cambiado.} \\
  \hline

  \textbf{Ejemplo} & \multicolumn{4}{|b{0.7\linewidth}|}{Evento que notifica del cambio de la propiedad de configuración \emph{TargetTemperature}:\linebreak
  \usebox\configurationchangedeventbox} \\

  \hline

  \caption{Especificación de las notificaciones que publica el servicio de análisis.}
\end{longtable}

\subsection{Planificador}

\subsubsection{Peticiones asíncronas}

\begin{longtable}{|m{2cm}|m{2.3cm}|m{10cm}|b{0.85cm}|b{2.75cm}|}
  \hline

  \textbf{Peticion} & \multicolumn{4}{|b{0.7\linewidth}|}{\emph{SystemConfigurationChangeRequest}} \\
  \hline

  \textbf{\emph{Cola}} & \multicolumn{4}{|b{0.7\linewidth}|}{\emph{AdaptionLoop.Planification.Requests}} \\
  \hline

  \textbf{Descripción} & \multicolumn{4}{|b{0.82\linewidth}|}{Petición que representa una propuesta de cambio de la configuración del sistema.} \\
  \hline

  \textbf{Propiedades}
    & \emph{Timestamp} & \multicolumn{3}{|m{0.67\linewidth}|}{Fecha y hora de la petición de cambio.} \\
    \cline{2-5}
    & \emph{Symptoms} & \multicolumn{3}{|m{0.67\linewidth}|}{Colección de síntomas que la han desencadenado.} \\
    \cline{2-5}
    & \emph{Configuration Requests} & \multicolumn{3}{|m{0.67\linewidth}|}{Colección peticiones de configuración de la propuesta de cambio. Cada una de estas está compuesta por:
    \begin{itemize}[noitemsep]
      \item \textbf{\emph{ServiceName}}: Identificador del servicio cuya configuración queremos cambiar.
      \item \textbf{\emph{IsDeployed}}: Indica si el servicio debe estar desplegado o no en la siguiente configuración.
      \item \textbf{\emph{Bindings}}: Colección de conexiones que indican a qué otros servicios debe estar conectado (o no) en la siguiente configuración.
      \item \textbf{\emph{ConfigurationProperties}}: Colección de pares clave-valor que representan valores de su configuración que queremos actualizar.
    \end{itemize}} \\
  \hline

  \textbf{Ejemplo} & \multicolumn{4}{|b{0.82\linewidth}|}{Solicitud de cambio del modo de un aire acondicionado a modo calefacción (\emph{heating}). Los síntomas indican que fue desencadenada porque la temperatura era menor que un umbral determinado:\linebreak
  \usebox\systemconfigurationchangerequestbox} \\

  \hline

  \caption{Especificación de la petición asíncrona que expone el planificador.}
\end{longtable}

\subsection{Ejecutor}

\subsubsection{Peticiones asíncronas}

\newsavebox\executechangeplanrequestbox
\begin{lrbox}{\executechangeplanrequestbox}
  \begin{minipage}[t]{2in}
    \begin{verbatim}
{
  "Timestamp": "2022-06-19T16:38:30.6092751Z",
  "Symptoms":[
    {
      "Name": "temperature-lesser-than-cold-threshold",
      "Value": "true"
    }
  ],
  "ChangePlan":  [
    {
      "ServiceName": "Climatisation.AirConditioner.Service",
      "Type": "SetParameter",
      "PropertyName": "Mode",
      "PropertyValue": "Heating"
    }
  ]
}
        \end{verbatim}
  \end{minipage}
\end{lrbox}

\begin{longtable}{|m{2cm}|m{2.3cm}|m{10cm}|b{0.85cm}|b{2.75cm}|}
  \hline

  \textbf{Peticion} & \multicolumn{4}{|b{0.7\linewidth}|}{\emph{ExecuteChangePlanRequest}} \\
  \hline

  \textbf{\emph{Cola}} & \multicolumn{4}{|b{0.7\linewidth}|}{\emph{AdaptionLoop.Execution.Requests}} \\
  \hline

  \textbf{Descripción} & \multicolumn{4}{|b{0.82\linewidth}|}{Petición para ejecutar el plan de cambio creado por el planificador.} \\
  \hline

  \textbf{Propiedades}
    & \emph{Timestamp} & \multicolumn{3}{|m{0.67\linewidth}|}{Fecha y hora en la que se generó el plan de cambio.} \\
    \cline{2-5}
    & \emph{Symptoms} & \multicolumn{3}{|m{0.67\linewidth}|}{Colección de síntomas que lo han desencadenado.} \\
    \cline{2-5}
    & \emph{ChangePlan} & \multicolumn{3}{|m{0.67\linewidth}|}{Colección de acciones de adaptación que deben ejecutarse para completar la adaptación.
    Dependiendo del tipo de acción, tendrá una estructura similar a:
    \begin{itemize}
      \item \textbf{\emph{ServiceName}}: Identificador del servicio afectado.
      \item \textbf{\emph{Type}}: Indica el tipo de acción de adaptación. Puede ser: \emph{Deploy}, \emph{Undeploy}, \emph{SetParameter}, \emph{Bind}, \emph{Unbind}.
      \item \textbf{\emph{TargetService}}: (\emph{BindingAction}) Servicio con el que establecer una conexión.
      \item \textbf{\emph{PropertyName}}: (\emph{SetParameterAction}) Nombre de la propiedad a actualizar.
      \item \textbf{\emph{PropertyValue}}: (\emph{SetParameterAction}) Valor a asignar a la propiedad.
    \end{itemize}} \\
  \hline

  \textbf{Ejemplo} & \multicolumn{4}{|b{0.82\linewidth}|}{Petición de ejecución de plan de cambio que contiene una acción de adaptación. Esta cambia el modo de un aire acondicionado a modo calefacción (\emph{heating}). Los síntomas indican que fue desencadenada porque la temperatura era menor que un umbral determinado:\linebreak
  \usebox\executechangeplanrequestbox} \\

  \hline

  \caption{Especificación de la petición asíncrona que expone el ejecutor.}
\end{longtable}

\pagebreak

\subsubsection{Notificaciones}

\newsavebox\executenotificationbox
\begin{lrbox}{\executenotificationbox}
  \begin{minipage}[t]{2in}
    \begin{verbatim}
{
  "ServiceName": "Climatisation.AirConditioner.Service",
  "Timestamp": "2022-06-19T16:38:30.6092751Z",
  "Symptoms":[
    {
      "Name": "temperature-lesser-than-cold-threshold",
      "Value": "true"
    }
  ],
  "Actions":  [
    {
      "ServiceName": "Climatisation.AirConditioner.Service",
      "Type": "SetParameter",
      "PropertyName": "Mode",
      "PropertyValue": "Heating"
    }
  ]
}
        \end{verbatim}
  \end{minipage}
\end{lrbox}

\begin{longtable}{|m{2.3cm}|p{3cm}|p{2.6cm}|b{1.5cm}|b{1cm}|}
  \hline

  \textbf{Evento} & \multicolumn{4}{|b{0.7\linewidth}|}{\emph{ExecutionRequestedIntegrationEvent}} \\
  \hline

  \textbf{\emph{Exchange}} & \multicolumn{4}{|b{0.7\linewidth}|}{\emph{AdaptionLoop.Execute}}  \\
  \hline

  \textbf{Tema} & \multicolumn{4}{|b{0.7\linewidth}|}{Nombre del servicio afectado. Ej: \emph{Climatisation.AirConditioner}} \\
  \hline

  \textbf{Descripción} & \multicolumn{4}{|b{0.6\linewidth}|}{Evento de integración que notifica sobre la solicitud de ejecución de acciones de adaptación para un servicio determinado. Se publica como notificación porque no sabemos cuantos ejecutores tendrá un determinado servicio.} \\
  \hline

  \textbf{Propiedades}
        & \emph{ServiceName} & \multicolumn{3}{|b{0.6\linewidth}|}{Nombre del servicio afectado.} \\

        \cline{2-5}

        & \emph{Timestamp} & \multicolumn{3}{|b{0.6\linewidth}|}{Fecha y hora en la que se solicitó la ejecución.} \\

        \cline{2-5}

        & \emph{Symptoms} & \multicolumn{3}{|b{0.6\linewidth}|}{Conjunto de síntomas que han desencadenado la adaptación.} \\

        \cline{2-5}

        & \emph{Actions} & \multicolumn{3}{|b{0.6\linewidth}|}{Conjunto de acciones de adaptación a ejecutar.} \\

        \cline{2-5}

  \hline

  \textbf{Ejemplo} & \multicolumn{4}{|b{0.7\linewidth}|}{Evento que notifica de la solicitud de ejecución de una acción de adaptación sobre el servicio \emph{Climatisation.AirConditioner}. Esta solicita que se cambie el parámetro \emph{Mode} a \emph{Heating}:\linebreak
  \usebox\executenotificationbox} \\

  \hline

  \caption{Especificación de las notificaciones que publica el servicio ejecución.}
\end{longtable}

\section{Sistema de climatización}

\subsection{Recurso manejado: Aire acondicionado}

\subsubsection{Peticiones síncronas}

Su especificación OpenAPI puede encontrarse \href{https://github.com/Starkie/TFM-DistributedAutoadaptiveSystems/blob/1db95346290cb55edbfd5efb717785bcd06def79/src/AutoAdaptativeSystem/Climatisation/AirConditioner/Service/Climatisation.AirConditioner.Service-OpenAPISpec.json}{aquí}.

\begin{figure}[h!]
  \hspace{-0.25cm}
  \includegraphics[scale=0.45]{anx_apis/images/apis-airconditioner}
  \caption{\foreign{english}{Endponts} HTTP que expone el recurso manejado: el aire acondicionado.}
  \label{fig:apis-eps-aireacondicionado}
\end{figure}

\subsection{Monitor}

\subsubsection{Peticiones síncronas}

Su especificación OpenAPI puede encontrarse \href{https://github.com/Starkie/TFM-DistributedAutoadaptiveSystems/blob/1db95346290cb55edbfd5efb717785bcd06def79/src/AutoAdaptativeSystem/Climatisation/Monitor/Climatisation.Monitor.Service-OpenAPISpec.json}{aquí}.

\begin{figure}[h!]
  \hspace{-0.25cm}
  \includegraphics[scale=0.45]{anx_apis/images/apis-room-monitor}
  \caption{\foreign{english}{Endponts} HTTP que expone el servicio del monitor del sistema de climatización.}
\end{figure}


%%%%%%%%%%%%%%%%%%%%%%%%%%%%%%%%%%%%%%%%%%%%%%%%%%%%%%%%%%%%%%%%%%%%%%%%%%%%%%%
%                              FI DEL DOCUMENT                                %
%%%%%%%%%%%%%%%%%%%%%%%%%%%%%%%%%%%%%%%%%%%%%%%%%%%%%%%%%%%%%%%%%%%%%%%%%%%%%%%

\end{document}
