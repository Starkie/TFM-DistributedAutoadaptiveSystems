\chapter{Caso de estudio: Sistema de climatización}
\label{chap:caso_estudio}

Para verificar la arquitectura definida, decidimos implementar un pequeño sistema autoadaptativo. Se trata de un sistema de climatización, que gestiona la temperatura de una habitación. Para ello, dispondremos de un aire acondicionado, que calentará o enfriará la habitación según corresponda.

\section{Análisis}

El primer paso es capturar los requisitos del sistema a implementar. Cómo hemos comentado, queremos desarrollar un sistema de climatización. Este sistema regulará la temperatura de una habitación mediante el uso de un aparato de aire acondicionado.

El aparato de aire acondicionado ofrece tres modos de funcionamiento: un modo para calentar la estancia, otro para enfriarla, y un estado neutral (apagado). Además, lo hemos dotado con un termómetro interno que nos reporta la temperatura periódicamente.

Para poder climatizar la habitación, necesitamos que el usuario defina su temperatura objetivo: la temperatura de confort. Cambios en la temperatura deberán activar o desactivar el aparato para mantenerla.

Además, nos interesa evitar que el aire acondicionado se encienda y se apague constantemente cuando se alcance o sobrepase esta temperatura. Por ello, definimos unas temperaturas umbrales, tanto de frío como de calor, a partir de las cuales se encenderá el aparato.

\section{Diseño}

Del análisis anterior ya podemos deducir la existencia de dos componentes: un aparato de aire acondicionado (el sistema gestionado) y un termómetro (la sonda). Aparte de ellos, deberemos implementar la infraestructura necesaria para comunicarse con nuestro bucle MAPE-K: monitores, módulos de reglas y efectores que nos permitan interactuar con el sistema manejado.

Para describir el diseño usaremos la notación de sistemas autoadaptativos descrita en \cite{fonsEspecificacionSistemasAutoadaptativos2021}.

\subsection{Sondas:}

Para implementar el sistema, requerimos de las siguientes sondas:

\begin{table}[htb]
  \centering

  \begin{tabular}{|r p{11.5cm}|}
    \hline
    \textbf{Sonda:} & \emph{thermometer}  \\
    \textbf{Descripción:} & Reporta la temperatura actual de la habitación (en ºc). \\
    \textbf{Monitor:} & \emph{Climatisation.Monitor} \\
    \textbf{Datos:} & \emph{temperature} \\
    \hline
    \textbf{Sonda:} & \emph{airconditioner-mode-changed-probe}  \\
    \textbf{Descripción:} & Reporta el modo de funcionamiento del aire acondicionado cuando este cambia. \\
    \textbf{Monitor:} & \emph{Climatisation.Monitor} \\
    \textbf{Datos:} & \emph{airconditioner-mode} \\
    \hline
    \textbf{Sonda:} & \emph{airconditioner-adaption-loop-registration}  \\
    \textbf{Descripción:} & Cuando arranca el servicio de aire acondicionado, registra la configuración inicial del sistema. \\
    \textbf{Monitor:} & \emph{Climatisation.Monitor} \\
    \textbf{Datos:} & \emph{airconditioner.is-deployed}, \emph{airconditioner-mode}, \emph{target-temperature}, \emph{cold-temperature-threshold}, \emph{hot-temperature-threshold} \\
    \hline
  \end{tabular}
\end{table}

\subsection{Efectores:}

Para implementar el sistema, requerimos de los siguientes efectores. Son necesarios para cambiar el modo de funcionamiento del aire acondicionado:

\begin{table}[htb]
  \centering

  \begin{tabular}{|r p{11.5cm}|}
    \hline
    \textbf{Efector:} & \emph{airconditioner.heat}  \\
    \textbf{Descripción:} & Activa el modo calentar del aire acondicionado. \\
    \hline
    \textbf{Efector:} & \emph{airconditioner.cool}  \\
    \textbf{Descripción:} & Activa el modo enfriar del aire acondicionado. \\
    \hline
    \textbf{Efector:} & \emph{airconditioner.turn-off}  \\
    \textbf{Descripción:} & Apaga el aire acondicionado. \\
    \hline
  \end{tabular}
\end{table}

\subsection{Propiedades de adaptación:}

También podemos deducir cuáles son nuestras propiedades de adaptación:

\begin{table}[htb]
  \centering

  \begin{tabular}{|r p{11.5cm}|}
    \hline
    \textbf{Propiedad:} & \emph{temperature}  \\
    \textbf{Descripción:} & Representa la temperatura actual de la habitación (en ºC).  \\
    \textbf{Tipo de dato:} & \emph{float} \\
    \hline
    \textbf{Propiedad:} & \emph{target-temperature}  \\
    \textbf{Descripción:} & La temperatura de confort definida por el usuario. El sistema deberá adaptarse para alcanzarla.  \\
    \textbf{Tipo de dato:} & \emph{float} \\
    \hline
    \textbf{Propiedad:} & \emph{cold-temperature-threshold}  \\
    \textbf{Descripción:} & La temperatura umbral de frío (en ºc). Si la temperatura baja por debajo de este umbral, deberá calentarse la habitación. \\
    \textbf{Tipo de dato:} & \emph{float} \\
    \hline
    \textbf{Propiedad:} & \emph{hot-temperature-threshold}  \\
    \textbf{Descripción:} & La temperatura umbral de calor (en ºc). Si la temperatura sube por encima de este umbral, deberá enfriarse la habitación. \\
    \textbf{Tipo de dato:} & \emph{float} \\
    \hline
    \textbf{Propiedad:} & \emph{airconditioner.is-deployed}  \\
    \textbf{Descripción:} & Indica si el servicio de aire acondicionado está desplegado y en funcionamiento.  \\
    \textbf{Tipo de dato:} & \emph{bool} \\
    \hline
    \textbf{Propiedad:} & \emph{airconditioner-mode}  \\
    \textbf{Descripción:} & Representa el modo de operación actual del aire acondicionado: \emph{Off} = 0, \emph{Cooling} = 1, \emph{Heating} = 2  \\
    \textbf{Tipo de dato:} & Enumerado \\
    \hline
  \end{tabular}

  \caption{Propiedades de adaptación del sistema de climatización.}
  \label{tab:adaption-properties-climatisation}
\end{table}

Describir el Climatisation.Monitor. Descarta medidas erróneas.

Describir las reglas con la notación SAS.

Necesitaremos por tanto dos tres elementos para realizar la adaptación: un aire acondicionado (sistema gestionado), un termómetro (sonda) y unos efectores, para ejectura las acciones. El termómetro nos reportará periódicamente la temperatura de la habitación.

Para evitar falsos positivos, y que se lleve a cabo adaptaciones provocadas por errores de medición, deberemos filtrar estos datos. Por ejemplo, podría darse el caso que el termómetro nos provea una temperatura que fluctúa más de los esperado o algo por el estilo. Descartaremos medidas erróneas usando un monitor, que filtrará los datos.

En base a cambios de la temperatura local, deberemos decidir si es necesario llevar a cabo una acción correctiva. Por ejemplo, que si la temperatura es inferior al umbral de temperatura fría, el aparato se enciende en modo calentador. Para ello, deberemos definir una serie de reglas que se disparen cuando cambie una de nuestras propiedades de adaptación. En este caso, la temperatura.

Definimos 4 reglas distintas:
\begin{itemize}
  \item Una para activar el aire condicionado en modo enfriar cuando se supere el umbral de calor.
  \item Una para desactivar el aire acondicionado cuando esté en modo enfriar y se alcance la temperatura de confort.
  \item Una para activar el aire acondicionad en modo calor cuando la temperatura sea inferior al umbral de frío.
  \item Una para desactivar el aire acondicionado cuando esté en modo calentar y
\end{itemize}

Como comentamos en el capitulo anterior, en nuestro ejemplo de bucle MAPE-K, nos limitamos a implementar las adaptaciones de tipo set-parameter. Por tanto, no tendremos reglas de despliegue o de binding.

Una vez se disparan estas reglas, solicitamos un cambio en la configuración del sistema. El módulo de planificación comprobará contra el conocimiento y el estado actual del sistema cuáles de los cambios solicitados es necesario aplicar. Si por ejemplo la propiedad ya tiene el valor solicitado, no hará falta ejecutarla.

El modulo de ejecución recibirá la petición y se la redirigirá a los efectores del sistema de climatización. En este caso, cambiarán el modo del aire acondicionado según corresponda.

Hecho esto, el sistema se adapta a a la nueva situación, y reportará una nueva temperatura en cuanto corresponda. La temperatura variará dependiendo de si está apagado o no.

Para nuestra implementación, hemos decidido implementar un aire acondicionado ficticio, que reporta las medidas periodicamente. Cuando está apagado, la temperatura aumenta o

\section{Implementación}
