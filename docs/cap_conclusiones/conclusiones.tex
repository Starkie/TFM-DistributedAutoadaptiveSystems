\chapter{Conclusiones}
\label{chap:conclusiones}

En este capítulo se realiza una retrospectiva del desarrollo del trabajo y presentaremos algunas conclusiones sobre el mismo. Por ejemplo, el grado de realización de los objetivos marcados o la descripción de vertientes todavía abiertas.

Al inicio del trabajo se presentaron una serie de objetivos que se quería alcanzar.


Para el desarrollo del trabajo nos planteamos los siguientes objetivos:

\begin{enumerate}
  \item Diseñar una arquitectura para soluciones auto-adaptativas preparadas para desplegarse nativamente como microservicios en la nube. Esto implica determinar los componentes en los que dividiremos la funcionalidad del bucle y los mecanismos de comunicación para conectarlos.

  \item Definir directrices para la implementación de los diferentes componentes adaptativos específicos de una solución: monitores, sondas, efectores\dots

  \item Desarrollar un caso práctico para demostrar la viabilidad y aplicabilidad de nuestra propuesta.
\end{enumerate}

\section{Relación con asignaturas cursadas}

El trabajo desarrollado tiene relación con varias asignaturas cursadas durante el máster. Entre ellas, podemos destacar:

\begin{itemize}
  \item \textbf{Diseño de Sistemas Ubicuos y Adaptativos} (SUA): Es la asignatura que más relación guarda con el trabajo. En ella se tratan la computación autónoma y los sistemas autoadaptativos. Mediante el desarrollo de un prototipo de coche autónomo, se presentó el bucle MAPE-K y sus distintas fases.

  \item \textbf{Internet de los Servicios (IoS) y de las Cosas (IoT)} (ISC): En esta asignatura se presentan conceptos relacionados con los servicios web. Se introducen patrones de diseño como las APIs REST y arquitecturas adaptadas a entornos \foreign{english}{cloud}. Además, el campo del internet de las cosas se beneficia también de los sistemas autoadaptativos. \cite{savaglioAgentbasedInternetThings2020}.

  \item \textbf{\foreign{english}{Data Science}} (DAS) y \textbf{Extracción de información desde la red social} (ERS): En ambas asignaturas se trata la extracción de conocimiento a partir de los datos. Mediante técnicas de obtención, procesamiento y visualización, podemos interpretarlos y ''contar una historia'' con ellos. Ambas tuvieron una gran influencia en el desarrollo de la plataforma de observabilidad y las visualizaciones implementadas (capitulo \ref{chap:despliegue}).

\end{itemize}

\section{Trabajos futuros}

En cuanto a trabajos futuros, el más evidente es aplicar la refactorización sobre el bucle MAPE-K \foreign{english}{Lite} de FaDA. Aunque el prototipo pretendía ser lo más fiel al sistema original, es posible que surjan nuevas dificultades no contempladas. Deberá definirse una estrategia para atacarla e ir implementándola gradualmente. Para ello, podrán aprovecharse las interfaces definidas de los servicios y sus especificaciones en lenguajes estándares como OpenAPI. Mediante la generación de código, tanto de clientes como de servidores, se podrá reducir el tiempo y esfuerzo necesarios para la implementación. abo.

Todavía quedan algunas vertientes abiertas que se podrían explorar. Entre ellas, la implementación de \foreign{english}{multitenancy} (multicliente). \cite{aljahdaliMultitenancyCloudComputing2014} Es decir, permitir que varias soluciones autoadaptativas empleen la misma infraestructura del bucle; pero de forma segregada, sin poder interferir entre ellas o acceder a los datos de otras. Se excluyó de este trabajo para reducir el alcance del proyecto. Para implementarla, se debería desarrollar mecanismos de autenticación y autorización, que permitan identificar a cada aplicación y limitar sus permisos. Además deberá estudiarse cómo proteger el acceso a la información de los distintos clientes.

Por otro lado, se podrían investigar más maneras de explotar la telemetría recogida por la plataforma de observabilidad. Por ejemplo, para informar al proceso del bucle MAPE-K. De esta forma, se podrían extraer propiedades de adaptación y ampliar nuestro conocimiento del estado del sistema. En base a estas, se podrían definir nuevas reglas que nos permitan adaptar nuestro sistema a distintas situaciones. Un caso interesante sería analizar las métricas de peticiones concurrentes para desplegar nuevas instancias de los servicios.
