%%%%%%%%%%%%%%%%%%%%%%%%%%%%%%%%%%%%%%%%%%%%%%%%%%%%%%%%%%%%%%%%%%%%%%%%%%%%%%%
%                                  INTRODUCCIO                                %
%%%%%%%%%%%%%%%%%%%%%%%%%%%%%%%%%%%%%%%%%%%%%%%%%%%%%%%%%%%%%%%%%%%%%%%%%%%%%%%

\chapter{Introducción}
\label{chap:introduccion}

\textcolor{red}{La revolución digital \emph{citation needed} ha permeado en todos los aspectos de nuestras vidas. En nuestro día a día usamos una gran variedad de aplicaciones: redes sociales, aplicaciones de ofimática, comercios en línea\dots. A su vez, gracias a las posibilidades que ofrecen los avances en las tecnologías, cada vez ofrecen distintas funcionalidades y están en constante evolución. \emph{citation needed} Esto deriva en mayor complejidad de estos programas. \emph{citation needed}}

\textcolor{red}{Las aplicaciones están en constante evolución. Hay muchas partes móviles}

\textcolor{red}{Tambien el internet de las cosas \emph{citation needed}}

\textcolor{red}{Uno de los requisitos claves es la disponibilidad: nuestros servicios deben estar en funcionamiento en todo momento para atender a nuestros usuarios. Tomemos por ejemplo el caso de una tienda \foreign{english}{on-line}. Necesitamos asegurar que esté disponible el mayor tiempo posible. Si surgiera una incidencia y se degrada la capacidad de atender a clientes, o directamente no podemos atender a ninguno, perderemos ingresos.}

\textcolor{red}{Debido a esto querremos que nuestro sistema sea capaz de adaptarse a picos de demanda, aumentando su capacidad de cómputo cuando tengamos mayor afluencia de clientes. Por ejemplo, en temporadas de rebajas como \emph{black friday}. Operar sistemas capaces de escalar, deriva en sistemas complejos. Como no es viable tener a operarios pendientes del estado del sistema para llevar a cabo estas adaptaciones. Deben hacerse automáticamente.}

\textcolor{red}{En el ámbito de la computación autónoma encontramos el concepto de sistemas \textbf{autoadaptativos}: aquellos capaces de ajustar su propio comportamiento en base a cambios en su entorno de operación. Se caracteriza por dotarlos con capacidades para razonar sobre su estado de operación y su entorno. En base a estos parámetros, el sistema puede intuir que debe reconfigurarse para cumplir con los objetivos que tiene marcados. Para ello, en base a una serie de estrategias predefinidas, es capaz de elegir su siguiente configuración. \cite{garlanIncreasingSystemDependability2003}. Esto conlleva mover a tiempo de ejecución las decisiones de arquitectura y funcionalidad. Con ello, buscamos permitir un comportamiento dinámico del sistema. \cite{brunEngineeringSelfAdaptiveSystems2009}.}

\section{Motivación}

En este trabajo se quiere abordar la división de un servicio monolítico y adaptarlo para su funcionamiento en entornos en la nube. Para ello, se quiere extraer su funcionalidad en distintos microservicios. Es decir, se quiere \textbf{cambiar la topología} de la solución. Se trata de un cambio importante en la arquitectura de la solución.

En concreto, se trata de un servicio que implementa un bucle de control MAPE-K \cite{ibmcorporationArchitecturalBlueprintAutonomic2006, fonsServiciosAdaptivereadyPara2021}, una para la implementación de sistemas autónomos propuesta inicialmente por IBM. El bucle se encarga de gestionar un \textbf{recurso manejado} en base a unas \textbf{políticas} definidas por el administrador del sistema. Las políticas

%% TODO: ¿Multi-tennant? ¿Solución inicial muy acoplada y ad-hoc a una solución concreta? Se quiere independizar del programa.

\textcolor{red}{La idea es separar cada una de sus etapas en microservicios individuales. De esta forma, podemos desarrollarlas de forma independiente entre ellas, replicarlas para mejorar su escalabilidad, o sustituirlas por }implementaciones distintas, etc.

Para desarrollar el trabajo, propusimos el siguiente plan:
\begin{itemize}
  \item Cada etapa del bucle será un microservicio distinto. Extraeremos cuatro microservicios distintos: Planificador, Analizador,
\end{itemize}

\textcolor{red}{Por tanto, los conectores elegidos para comunicar los microservicios han sido más centrados en comunicar con las APIs públicas que expone cada uno.}

-----------------------------------------------------------------

\section{Motivación}

????? ????????????? ????????????? ????????????? ????????????? ?????????????

\section{Objetivos}

????? ????????????? ????????????? ????????????? ????????????? ?????????????

\section{Estructura de la memoria}

????? ????????????? ????????????? ????????????? ????????????? ?????????????

%\section{Notes bibliografiques} %%%%% Opcional

%????? ????????????? ????????????? ????????????? ????????????? ?????????????
