%%%%%%%%%%%%%%%%%%%%%%%%%%%%%%%%%%%%%%%%%%%%%%%%%%%%%%%%%%%%%%%%%%%%%%%%%%%%%%%
%                                  INTRODUCCIO                                %
%%%%%%%%%%%%%%%%%%%%%%%%%%%%%%%%%%%%%%%%%%%%%%%%%%%%%%%%%%%%%%%%%%%%%%%%%%%%%%%

\chapter{Introducción}
\label{chap:introduccion}

La Computación Autónoma (\foreign{english}{Autonomic Computing}) promueve la ingeniería, diseño y desarrollo de sistemas con capacidades de auto-adaptación, a través del uso de bucles de control. Estas capacidades le confieren a estos sistemas la posibilidad de adaptarse a entornos cambiantes, a conflictos operacionales e incluso a la optimización dinámica en su ejecución. Por otra parte, en la última década, la computación en el cloud y las arquitecturas basadas en microservicios se han postulado como una infraestructura muy flexible y dinámica para desplegar soluciones altamente disponibles y eficientes. Hay una tendencia clara a aplicar este tipo de infraestructuras, gracias a los múltiples beneficios que aporta.

\textcolor{red}{La revolución digital \emph{citation needed} ha permeado en todos los aspectos de nuestras vidas. En nuestro día a día usamos una gran variedad de aplicaciones: redes sociales, aplicaciones de ofimática, comercios en línea\dots. A su vez, gracias a las posibilidades que ofrecen los avances en las tecnologías, cada vez ofrecen distintas funcionalidades y están en constante evolución. \emph{citation needed} Esto deriva en mayor complejidad de estos programas. \emph{citation needed}}

\textcolor{red}{Las aplicaciones están en constante evolución. Hay muchas partes móviles}

\textcolor{red}{Tambien el internet de las cosas \emph{citation needed}}

\textcolor{red}{Uno de los requisitos claves es la disponibilidad: nuestros servicios deben estar en funcionamiento en todo momento para atender a nuestros usuarios. Tomemos por ejemplo el caso de una tienda \foreign{english}{on-line}. Necesitamos asegurar que esté disponible el mayor tiempo posible. Si surgiera una incidencia y se degrada la capacidad de atender a clientes, o directamente no podemos atender a ninguno, perderemos ingresos.}

\textcolor{red}{Debido a esto querremos que nuestro sistema sea capaz de adaptarse a picos de demanda, aumentando su capacidad de cómputo cuando tengamos mayor afluencia de clientes. Por ejemplo, en temporadas de rebajas como \emph{black friday}. Operar sistemas capaces de escalar, deriva en sistemas complejos. Como no es viable tener a operarios pendientes del estado del sistema para llevar a cabo estas adaptaciones. Deben hacerse automáticamente.}

\textcolor{red}{En el ámbito de la computación autónoma encontramos el concepto de sistemas \textbf{autoadaptativos}: aquellos capaces de ajustar su propio comportamiento en base a cambios en su entorno de operación. Se caracteriza por dotarlos con capacidades para razonar sobre su estado de operación y su entorno. En base a estos parámetros, el sistema puede intuir que debe reconfigurarse para cumplir con los objetivos que tiene marcados. Para ello, en base a una serie de estrategias predefinidas, es capaz de elegir su siguiente configuración. \cite{garlanIncreasingSystemDependability2003}. Esto conlleva mover a tiempo de ejecución las decisiones de arquitectura y funcionalidad. Con ello, buscamos permitir un comportamiento dinámico del sistema. \cite{brunEngineeringSelfAdaptiveSystems2009}.}

\section{Motivación}

En este trabajo se quiere explorar el diseño de soluciones que, aplicando los conceptos de los bucles de control (AC), estén preparadas para desplegarse en la nube. Para ello se tomó como punto de partida la infraestructura FaDA\footnote{Página oficial: \url{http://fada.tatami.webs.upv.es/}} (desarrollada por el grupo PROS/Tatami\footnote{Página oficial: \url{http://www.pros.webs.upv.es/}} del instituto VRAIN/UPV\footnote{Página oficial: \url{https://vrain.upv.es/}}). Esta propone una estrategia para la ingeniería de sistemas auto-adaptativos usando bucles de control MAPE-K\cite{ibmcorporationArchitecturalBlueprintAutonomic2006, fonsServiciosAdaptivereadyPara2021}.

Actualmente, el bucle de control de FaDA está implementado como un servicio monolítico. Todos sus componentes operan dentro del mismo proceso, incluidos los específicos a sistemas manejados (sondas, monitores\dots). Se trata por tanto de una implementación muy rígida. En caso de querer modificar algún componente, hay que redesplegarlo entero.

Por ello, se buscó dividir su funcionalidad en microservicios. Es decir, \textbf{cambiar la topología} de la solución. Con ello, lograríamos independizar los componentes y su despliegue individual. Además, facilitaría escalar horizontalmente la capacidad del sistema en base a la carga de sus componentes.\textcolor{red}{cita sam newman}.

\section{Objetivos}

Para el desarrollo del trabajo nos planteamos los siguientes objetivos:

\begin{enumerate}
  \item Diseñar una arquitectura para soluciones auto-adaptativas preparadas para desplegarse nativamente como microservicios en la nube. Esto implica determinar los componentes en los que dividiremos la funcionalidad del bucle y los mecanismos de comunicación para conectarlos.

  \item Definir directrices para la implementación de los diferentes componentes adaptativos específicos de una solución (monitores, sondas, efectores...).

  \item Desarrollar un caso práctico para demostrar la viabilidad y aplicabilidad de nuestra propuesta.
\end{enumerate}

\section{Estructura de la memoria}

El trabajo se puede dividir en tres grandes partes:

En los capítulos \ref{chap:contexto_tecnologico}-\textcolor{red}{x} hacemos una introducción a algunos conceptos de la computación autónoma y los bucles de control. Describiremos la arquitectura del bucle MAPE-K que nos ocupa.

La segunda parte de este trabajo describe la migración del sistema existente a una arquitectura basada en microservicios. Paara ello, comenzaremos describiendo el sistema actual en el capítulo \ref{chap:sistema_original}. En base a este, en el capítulo \ref{chap:diseño} describiremos nuestra propuesta arquitectónica. Aquí se describirá los distintos componentes que conforman nuestra solución y se describirá los mecanismos de comunicación por los que optamos.

Finalmente, presentamos el caso de estudio (capítulo \ref{chap:caso_estudio}). En él implementamos un sistema auto-adaptativo básico para un sistema de climatización.

Cerramos el trabajo presentando las conclusiones (capítulo \ref{chap:conclusiones}). Describiremos también las vertientes por las que se podría continuar ampliando el trabajo presentado.

%\section{Notes bibliografiques} %%%%% Opcional

%????? ????????????? ????????????? ????????????? ????????????? ?????????????

\textcolor{red}{Añadir diagrama de gant con los 7 hitos}
