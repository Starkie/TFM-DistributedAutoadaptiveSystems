%%%%%%%%%%%%%%%%%%%%%%%%%%%%%%%%%%%%%%%%%%%%%%%%%%%%%%%%%%%%%%%%%%%%%%%%%%%%%%%
%                                  INTRODUCCIO                                %
%%%%%%%%%%%%%%%%%%%%%%%%%%%%%%%%%%%%%%%%%%%%%%%%%%%%%%%%%%%%%%%%%%%%%%%%%%%%%%%

\chapter{Introducción}
\label{chap:introduccion}

La Computación Autónoma (\foreign{english}{Autonomic Computing}) promueve la ingeniería, diseño y desarrollo de sistemas con capacidades de auto-adaptación, a través del uso de bucles de control. Estas capacidades le confieren a estos sistemas la posibilidad de adaptarse a entornos cambiantes, a conflictos operacionales e incluso a la optimización dinámica en su ejecución. Por otra parte, en la última década, la computación en el cloud y las arquitecturas basadas en microservicios se han postulado como una infraestructura muy flexible y dinámica para desplegar soluciones altamente disponibles y eficientes. Hay una tendencia clara a aplicar este tipo de infraestructuras, gracias a los múltiples beneficios que aporta.

\textcolor{red}{La revolución digital \emph{citation needed} ha permeado en todos los aspectos de nuestras vidas. En nuestro día a día usamos una gran variedad de aplicaciones: redes sociales, aplicaciones de ofimática, comercios en línea\dots. A su vez, gracias a las posibilidades que ofrecen los avances en las tecnologías, cada vez ofrecen distintas funcionalidades y están en constante evolución. \emph{citation needed} Esto deriva en mayor complejidad de estos programas. \emph{citation needed}}

\textcolor{red}{Las aplicaciones están en constante evolución. Hay muchas partes móviles}

\textcolor{red}{Tambien el internet de las cosas \emph{citation needed}}

\textcolor{red}{Uno de los requisitos claves es la disponibilidad: nuestros servicios deben estar en funcionamiento en todo momento para atender a nuestros usuarios. Tomemos por ejemplo el caso de una tienda \foreign{english}{on-line}. Necesitamos asegurar que esté disponible el mayor tiempo posible. Si surgiera una incidencia y se degrada la capacidad de atender a clientes, o directamente no podemos atender a ninguno, perderemos ingresos.}

\textcolor{red}{Debido a esto querremos que nuestro sistema sea capaz de adaptarse a picos de demanda, aumentando su capacidad de cómputo cuando tengamos mayor afluencia de clientes. Por ejemplo, en temporadas de rebajas como \emph{black friday}. Operar sistemas capaces de escalar, deriva en sistemas complejos. Como no es viable tener a operarios pendientes del estado del sistema para llevar a cabo estas adaptaciones. Deben hacerse automáticamente.}

\textcolor{red}{En el ámbito de la computación autónoma encontramos el concepto de sistemas \textbf{autoadaptativos}: aquellos capaces de ajustar su propio comportamiento en base a cambios en su entorno de operación. Se caracteriza por dotarlos con capacidades para razonar sobre su estado de operación y su entorno. En base a estos parámetros, el sistema puede intuir que debe reconfigurarse para cumplir con los objetivos que tiene marcados. Para ello, en base a una serie de estrategias predefinidas, es capaz de elegir su siguiente configuración. \cite{garlanIncreasingSystemDependability2003}. Esto conlleva mover a tiempo de ejecución las decisiones de arquitectura y funcionalidad. Con ello, buscamos permitir un comportamiento dinámico del sistema. \cite{brunEngineeringSelfAdaptiveSystems2009}.}

\section{Motivación}

En este trabajo se exploró cómo diseñar soluciones que, aplicando los conceptos de los bucles de control (AC), estén preparadas para desplegarse en la nube. Para ello se tomó como punto de partida la infraestructura FaDA\footnote{Página oficial: \url{http://fada.tatami.webs.upv.es/}} (desarrollada por el grupo PROS/Tatami\footnote{Página oficial: \url{http://www.pros.webs.upv.es/}} del instituto VRAIN/UPV\footnote{Página oficial: \url{https://vrain.upv.es/}}). Esta propone una estrategia para realizar la ingeniería de sistemas auto-adaptativos usando bucles de control MAPE-K\cite{ibmcorporationArchitecturalBlueprintAutonomic2006, fonsServiciosAdaptivereadyPara2021}.

Actualmente, el bucle de control está implementado como un servicio monolítico. Todos sus componentes operan dentro del mismo proceso, incluidos los específicos a sistemas manejados (sondas, monitores\dots). Se trata por tanto de una implementación muy rígida. En caso de querer modificar algún componente, hay que redesplegarlo entero.

Por ello, se buscó dividir su funcionalidad en microservicios. Es decir, \textbf{cambiar la topología} de la solución. Con ello, podríamos independizar los compomentes y mejoraríamos su despliegue y su escalabilidad\textcolor{red}{cita sam newman}.

\section{Objetivos}

Como resultado de este TFM se espera obtener la definición arquitectónica de soluciones auto-adaptativas (incluyendo tanto al bucle de control MAPE-K como directrices para la implementación de los diferentes componentes adaptativos de la solución) diseñadas para desplegarse nativamente como microservicios en la nube. Por último, se aplicará la propuesta realizada al desarrollo de un caso práctico para demostrar su viabilidad y aplicabilidad.

Para el desarrollo del trabajo nos planteamos los siguientes objetivos objetivos:

\begin{enumerate}
  \item  Definición de las piezas necesarias para componer una solución completa, y que permita ir extendiendo el bucle (en nuestro caso, las piezas eran M, A, P, E i K, pero también los Monitores, Sondas, Reglas, etc. que se convertirían en estos microservicios);

  \item La definición de las APIs (REST) para permitir la comunicación entre las piezas
\end{enumerate}

\textcolor{red}{La idea es separar cada una de sus etapas en microservicios individuales. De esta forma, podemos desarrollarlas de forma independiente entre ellas, replicarlas para mejorar su escalabilidad, o sustituirlas por }implementaciones distintas, etc.

Para desarrollar el trabajo, propusimos el siguiente plan:
\begin{itemize}
  \item Cada etapa del bucle será un microservicio distinto. Extraeremos cuatro microservicios distintos: Planificador, Analizador,
\end{itemize}

\textcolor{red}{Por tanto, los conectores elegidos para comunicar los microservicios han sido más centrados en comunicar con las APIs públicas que expone cada uno.}



\section{Estructura de la memoria}

????? ????????????? ????????????? ????????????? ????????????? ?????????????

%\section{Notes bibliografiques} %%%%% Opcional

%????? ????????????? ????????????? ????????????? ????????????? ?????????????

\textcolor{red}{Añadir diagrama de gant con los 7 hitos}
