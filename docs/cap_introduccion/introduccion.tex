%%%%%%%%%%%%%%%%%%%%%%%%%%%%%%%%%%%%%%%%%%%%%%%%%%%%%%%%%%%%%%%%%%%%%%%%%%%%%%%
%                                  INTRODUCCIO                                %
%%%%%%%%%%%%%%%%%%%%%%%%%%%%%%%%%%%%%%%%%%%%%%%%%%%%%%%%%%%%%%%%%%%%%%%%%%%%%%%

\chapter{Introducción}
\label{chap:introduccion}

Debido al avance de las tecnologías en las últimas décadas, y a la penetración del \emph{software} en todos los ámbitos de nuestras vidas, cada vez tenemos sistemas más complejos y con requisitos de disponibilidad más altos. Por ejemplo, en caso de tener una tienda online, necesitamos asegurar que la tienda está disponible el mayor tiempo posible. Cuanto más tiempo pase ''caída'', menos potenciales clientes nos comprarán, y perderemos ingresos.

Por otro lado, queremos también querremos que nuestro sistema sea capaz de adaptarse a picos de demanda, aumentando su capacidad de cómputo cuando tengamos mayor afluencia de clientes. Por ejemplo, en temporadas de rebajas como \emph{black friday}. Operar sistemas capaces de escalar, deriva en sistemas complejos. Como no es viable tener a operarios pendientes del estado del sistema para llevar a cabo estas adaptaciones. Deben hacerse automáticamente.

En el ámbito de la computación autónoma encontramos el concepto de sistemas \textbf{autoadaptativos}: aquellos capaces de ajustar su propio comportamiento en base a cambios en su entorno de operación. Se caracteriza por dotarlos con capacidades para razonar sobre su estado de operación y su entorno. En base a estos parámetros, el sistema puede intuir que debe reconfigurarse para cumplir con los objetivos que tiene marcados. Para ello, en base a una serie de estrategias predefinidas, es capaz de elegir su siguiente configuración. \cite{garlanIncreasingSystemDependability2003}. Esto conlleva mover a tiempo de ejecución las decisiones de arquitectura y funcionalidad. Con ello, buscamos permitir un comportamiento dinámico del sistema. \cite{brunEngineeringSelfAdaptiveSystems2009}.

En este trabajo se quiere abordar la división de un servicio monolítico y adaptarlo para su funcionamiento en entornos en la nube. Para ello, se quiere extraer su funcionalidad en distintos microservicios. Es decir, se quiere \textbf{cambiar la topología} de la solución. Se trata de un cambio importante en la arquitectura de la solución.

En concreto, se trata de un servicio que implementa un bucle de control MAPE-K \cite{ibmcorporationArchitecturalBlueprintAutonomic2006, fonsServiciosAdaptivereadyPara2021}, una para la implementación de sistemas autónomos propuesta inicialmente por IBM. El bucle se encarga de gestionar un \textbf{recurso manejado} en base a unas \textbf{políticas} definidas por el administrador del sistema. Las políticas

%% TODO: ¿Multi-tennant? ¿Solución inicial muy acoplada y ad-hoc a una solución concreta? Se quiere independizar del programa.

La idea es separar cada una de sus etapas en microservicios individuales. De esta forma, podemos desarrollarlas de forma independiente entre ellas, replicarlas para mejorar su escalabilidad, o sustituirlas por implementaciones distintas, etc.

Para desarrollar el trabajo, propusimos el siguiente plan:
\begin{itemize}
  \item Cada etapa del bucle será un microservicio distinto. Extraeremos cuatro microservicios distintos: Planificador, Analizador,
\end{itemize}

Por tanto, los conectores elegidos para comunicar los microservicios han sido más centrados en comunicar con las APIs públicas que expone cada uno.

-----------------------------------------------------------------

\section{Motivación}

????? ????????????? ????????????? ????????????? ????????????? ?????????????

\section{Objetivos}

????? ????????????? ????????????? ????????????? ????????????? ?????????????

\section{Estructura de la memoria}

????? ????????????? ????????????? ????????????? ????????????? ?????????????

%\section{Notes bibliografiques} %%%%% Opcional

%????? ????????????? ????????????? ????????????? ????????????? ?????????????
